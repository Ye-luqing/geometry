\documentclass[a4paper]{article}
\usepackage{amsmath,amsfonts,amsthm,amssymb}
\usepackage{bm}
\usepackage{hyperref}
\usepackage{geometry}
\usepackage{yhmath}
\usepackage{pstricks-add}
\usepackage{framed,mdframed}
\usepackage{graphicx,color} 
\usepackage{mathrsfs,xcolor} 
\usepackage[all]{xy}
\usepackage{fancybox} 
\usepackage{xeCJK}
\newtheorem*{theo}{定理}
\newtheorem*{hypo}{猜想}
\newmdtheoremenv{lemma}{引理}
\newmdtheoremenv{corollary}{推论}
\newmdtheoremenv{exercise}{习题}
\newmdtheoremenv{example}{例}
\newenvironment{theorem}
{\bigskip\begin{mdframed}\begin{theo}}
    {\end{theo}\end{mdframed}\bigskip}
\newenvironment{hypothethis}
{\bigskip\begin{mdframed}\begin{hypo}}
    {\end{hypo}\end{mdframed}\bigskip}
\geometry{left=2.5cm,right=2.5cm,top=2.5cm,bottom=2.5cm}
\setCJKmainfont[BoldFont=FZHei-B01S]{FZFangSong-Z02S}
\renewcommand{\today}{\number\year 年 \number\month 月 \number\day 日}
\newcommand{\D}{\displaystyle}\newcommand{\ri}{\Rightarrow}
\newcommand{\ds}{\displaystyle} \renewcommand{\ni}{\noindent}
\newcommand{\pa}{\partial} \newcommand{\Om}{\Omega}
\newcommand{\om}{\omega} \newcommand{\sik}{\sum_{i=1}^k}
\newcommand{\vov}{\Vert\omega\Vert} \newcommand{\Umy}{U_{\mu_i,y^i}}
\newcommand{\lamns}{\lambda_n^{^{\scriptstyle\sigma}}}
\newcommand{\chiomn}{\chi_{_{\Omega_n}}}
\newcommand{\ullim}{\underline{\lim}} \newcommand{\bsy}{\boldsymbol}
\newcommand{\mvb}{\mathversion{bold}} \newcommand{\la}{\lambda}
\newcommand{\La}{\Lambda} \newcommand{\va}{\varepsilon}
\newcommand{\be}{\beta} \newcommand{\al}{\alpha}
\newcommand{\dis}{\displaystyle} \newcommand{\R}{{\mathbb R}}
\newcommand{\N}{{\mathbb N}} \newcommand{\cF}{{\mathcal F}}
\newcommand{\gB}{{\mathfrak B}} \newcommand{\eps}{\epsilon}
\renewcommand\refname{参考文献}\renewcommand\figurename{图}
\usepackage[]{caption2} 
\renewcommand{\captionlabeldelim}{}
\begin{document}
\title{\huge{\bf{Ptolemy定理及其逆定理}}} \author{\small{叶卢
    庆\footnote{叶卢庆(1992---),男,杭州师范大学理学院数学与应用数学专业
      本科在读,E-mail:yeluqingmathematics@gmail.com}}}\date{}
\maketitle
\begin{theorem}[Ptolemy定理]
  如果一个四边形内接于一个圆,则四边形对角线的长度乘积等于四边形的两对
  对边长度乘积的和.
\end{theorem}
\begin{proof}[\textbf{证明}]
  如图 \eqref{fig:1},点 $A,B,C,D$ 顺次分布在圆周上.在复平面上考虑,设点
  $A,B,C,D$ 对应的复数分别为 $a,b,c,d$.由于 $\angle ABD=\angle DCA$,因
  此
$$
   \frac{a-b}{d-b}=\lambda_{1}\frac{a-c}{d-c},
$$
其中 $\lambda_{1}\in \mathbf{R}^{+}$.即
\begin{equation}
  \label{eq:1}
  (a-b)(d-c)=\lambda_{1}(a-c)(d-b).
\end{equation}
同理,
\begin{equation}\label{eq:2}
  (a-d)(b-c)=\lambda_{2}(a-c)(b-d).
\end{equation}
其中 $\lambda_2\in \mathbf{R}^{+}$.
为了证明 Ptolemy 定理,我们只用证明
$$
\lambda_1+\lambda_2=1.
$$
也就是证明
$$
(a-b)(d-c)-(a-d)(b-c)=(a-c)(d-b).
$$
这是简单的验证就可以证明的.
\begin{figure}[h]
\psset{xunit=1.0cm,yunit=1.0cm,algebraic=true,dotstyle=o,dotsize=3pt 0,linewidth=0.8pt,arrowsize=3pt 2,arrowinset=0.25}
\begin{pspicture*}(-1.3,-5.88)(23.02,6.3)
\pscircle(7.94,0.28){5.32}
\psline(3.99,3.84)(11.58,4.16)
\psline(11.58,4.16)(12.7,-2.08)
\psline(12.7,-2.08)(4.33,-3.63)
\psline(3.99,3.84)(4.33,-3.63)
\psline(11.58,4.16)(4.33,-3.63)
\psline(3.99,3.84)(12.7,-2.08)
\begin{scriptsize}
\psdots[dotstyle=*](3.99,3.84)
\rput[bl](3.76,3.96){$A$}
\psdots[dotstyle=*](11.58,4.16)
\rput[bl](11.68,4.38){$B$}
\psdots[dotstyle=*](12.7,-2.08)
\rput[bl](12.9,-2.16){$C$}
\psdots[dotstyle=*](4.33,-3.63)
\rput[bl](3.78,-4.02){$D$}
\end{scriptsize}
\end{pspicture*}
    \caption{}
    \label{fig:1}
  \end{figure}
\end{proof}
Ptolemy 定理的逆定理如下:
\begin{theorem}[Ptolemy定理的逆]
如果一个四边形对角线的长度乘积等于四边形的两对对边长度乘积的和,则四边形内
接于一个圆.  
\end{theorem}
\begin{proof}[\textbf{证明}]
  我们已经知道,
$$
\frac{(a-b)(d-c)}{(a-c)(d-b)}+\frac{(a-d)(b-c)}{(a-c)(b-d)}=1,
$$
而且现在有
$$
\left|\frac{(a-b)(d-c)}{(a-c)(d-b)}\right|+\left|\frac{(a-d)(b-c)}{(a-c)(b-d)}\right|=1,
$$
上面两条式子告诉我们,
$$
\frac{(a-b)(d-c)}{(a-c)(d-b)},\frac{(a-d)(b-c)}{(a-c)(b-d)}
$$
都是正实数.这样就证明了Ptolemy逆定理.
\end{proof}
\end{document}



















