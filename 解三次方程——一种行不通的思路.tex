\documentclass[a4paper]{article} 
\usepackage{amsmath,amsfonts,bm}
\usepackage{hyperref}
\usepackage{amsthm} 
\usepackage{geometry}
\usepackage{amssymb}
\usepackage{pstricks-add}
\usepackage{framed,mdframed}
\usepackage{graphicx,color} 
\usepackage{mathrsfs,xcolor} 
\usepackage[all]{xy}
\usepackage{fancybox} 
% \usepackage{CJKutf8}
\usepackage{xeCJK}
\newtheorem{theorem}{定理}
\newtheorem{lemma}{引理}[section]
\newtheorem{corollary}{推论}
\newtheorem*{exercise}{习题}
\newtheorem{example}{例}
\geometry{left=2.5cm,right=2.5cm,top=2.5cm,bottom=2.5cm}
\setCJKmainfont[BoldFont=Adobe Heiti Std R]{Adobe Song Std L}
\renewcommand{\today}{\number\year 年 \number\month 月 \number\day 日}
\newcommand{\D}{\displaystyle}
\newcommand{\ds}{\displaystyle} \renewcommand{\ni}{\noindent}
\newcommand{\pa}{\partial} \newcommand{\Om}{\Omega}
\newcommand{\om}{\omega} \newcommand{\sik}{\sum_{i=1}^k}
\newcommand{\vov}{\Vert\omega\Vert} \newcommand{\Umy}{U_{\mu_i,y^i}}
\newcommand{\lamns}{\lambda_n^{^{\scriptstyle\sigma}}}
\newcommand{\chiomn}{\chi_{_{\Omega_n}}}
\newcommand{\ullim}{\underline{\lim}} \newcommand{\bsy}{\boldsymbol}
\newcommand{\mvb}{\mathversion{bold}} \newcommand{\la}{\lambda}
\newcommand{\La}{\Lambda} \newcommand{\va}{\varepsilon}
\newcommand{\be}{\beta} \newcommand{\al}{\alpha}
\newcommand{\dis}{\displaystyle} \newcommand{\R}{{\mathbb R}}
\newcommand{\N}{{\mathbb N}} \newcommand{\cF}{{\mathcal F}}
\newcommand{\gB}{{\mathfrak B}} \newcommand{\eps}{\epsilon}
\renewcommand\refname{参考文献}
\begin{document}
\title{\huge{\bf{解三次方程——一种行不通的思路}}} \author{\small{叶卢
    庆\footnote{叶卢庆(1992---),男,杭州师范大学理学院数学与应用数学专业
      本科在读,E-mail:h5411167@gmail.com}}\\{\small{杭州师范大学理学院,浙
      江~杭州~310036}}}
\maketitle
根据分解定理,寻求三次方程 $x^3 + bx^2 + cx + d = 0$ 的根 $x,y,z$等价于求解下述
方程组:
$$
\begin{cases}
  x+y+z = -b\\
xy + yz + xz = c\\
xyz = -d.\\
\end{cases}
$$
其中 $x,y,z$ 是复数.为了求解方程组,我们先介绍一种思路.但是这是一种行
不通的思路.
\section{行不通的思路}
\label{sec:1}
复数 $x,y,z$ 在复平面上对应于点 $X,Y,Z$,这是三角形
的三个顶点(当然,如果 $X,Y,Z$ 有重合,则三角形为退化的三角形,因此我们假设 $X,Y,Z$ 两两不重合).由于
$x+y+z=-b$,因此三角形的重心对应于复数 $\frac{-b}{3}$.我们只要将三个顶
点 $X,Y,Z$ 平移成 $X',Y',Z'$,其中 $X',Y',Z'$ 对应的复数分别为
$x',y',z'$,且 $x'=x+\frac{b}{3},y'=y+\frac{b}{3},z'=z+\frac{b}{3}$.就
得到了重心位于原点的三角形 $X'Y'Z'$.可得
$$
\begin{cases}
  x'+y'+z'=0\\
x'y'+y'z'+x'z'=c'\\
x'y'z'=-d'
\end{cases}.
$$
现在,我们考虑从 $\mathbf{R}^2$ 到 $\mathbf{R}^2$ 的可逆线性映射 $T$,该
可逆线性映射要把三角形 $X'Y'Z'$ 变成顶点在单位圆上的三角形
$X''Y''Z''$,其中点 $X'$ 被 $T$ 映射成 $X''$,$Y'$ 被 $T$ 映射成 $Y''$,$Z'$
被 $T$ 映射成 $Z''$.且 $X''=(1,0)$,$Y''=(\cos \frac{2\pi}{3},\sin
\frac{2\pi}{3})$,$Z''=(\cos \frac{4\pi}{3},\sin \frac{4\pi}{3})$.然而
事实上这是不可能的,因为\textbf{$\mathbf{R}^2$ 到 $\mathbf{R}^2$ 的可逆
  线性映射不可能把平面上的非正三角形变成正三角形.}因此这种思路行不通.我
们换一种思路.
\end{document}








