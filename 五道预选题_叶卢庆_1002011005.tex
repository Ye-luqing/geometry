\documentclass[a4paper]{article}
\usepackage{amsmath,amsfonts,amsthm,amssymb}
\usepackage{bm}
\usepackage{hyperref}
\usepackage{geometry}
\usepackage{yhmath}
\usepackage{pstricks-add}
\usepackage{framed,mdframed}
\usepackage{graphicx,color} 
\usepackage{mathrsfs,xcolor} 
\usepackage[all]{xy}
\usepackage{fancybox} 
\usepackage{xeCJK}
\newtheorem{theo}{定理}
\newtheorem*{hypo}{题目}
\newmdtheoremenv{lemma}{引理}
\newmdtheoremenv{corollary}{推论}
\newmdtheoremenv{example}{例}
\newenvironment{theorem}
{\bigskip\begin{mdframed}\begin{theo}}
    {\end{theo}\end{mdframed}\bigskip}
\newenvironment{hypothethis}
{\bigskip\begin{mdframed}\begin{hypo}}
    {\end{hypo}\end{mdframed}\bigskip}
\geometry{left=2.5cm,right=2.5cm,top=2.5cm,bottom=2.5cm}
\setCJKmainfont[BoldFont=FZHei-B01S]{FZFangSong-Z02S}
\renewcommand{\today}{\number\year 年 \number\month 月 \number\day 日}
\newcommand{\D}{\displaystyle}\newcommand{\ri}{\Rightarrow}
\newcommand{\ds}{\displaystyle} \renewcommand{\ni}{\noindent}
\newcommand{\pa}{\partial} \newcommand{\Om}{\Omega}
\newcommand{\om}{\omega} \newcommand{\sik}{\sum_{i=1}^k}
\newcommand{\vov}{\Vert\omega\Vert} \newcommand{\Umy}{U_{\mu_i,y^i}}
\newcommand{\lamns}{\lambda_n^{^{\scriptstyle\sigma}}}
\newcommand{\chiomn}{\chi_{_{\Omega_n}}}
\newcommand{\ullim}{\underline{\lim}} \newcommand{\bsy}{\boldsymbol}
\newcommand{\mvb}{\mathversion{bold}} \newcommand{\la}{\lambda}
\newcommand{\La}{\Lambda} \newcommand{\va}{\varepsilon}
\newcommand{\be}{\beta} \newcommand{\al}{\alpha}
\newcommand{\dis}{\displaystyle} \newcommand{\R}{{\mathbb R}}
\newcommand{\N}{{\mathbb N}} \newcommand{\cF}{{\mathcal F}}
\newcommand{\gB}{{\mathfrak B}} \newcommand{\eps}{\epsilon}
\renewcommand\refname{参考文献}\renewcommand\figurename{图}
\usepackage[]{caption2} 
\renewcommand{\captionlabeldelim}{}
\begin{document}
\title{\huge{\bf{“数学解题研究”课程试题五道}}} \author{\small{叶卢
    庆\footnote{叶卢庆(1992---),男,杭州师范大学理学院数学与应用数学专业
      本科在读,E-mail:yeluqingmathematics@gmail.com}}\\{\small{杭州师范大学理学院}}}\date{}
\maketitle
\section{选择题}
\label{sec:1}
\begin{hypo}
(2013年辽宁高考理科数学第12题)设函数 $f(x)$ 满足
$x^2f'(x)+2xf(x)=\frac{e^x}{x}$,$f(2)=\frac{e^2}{8}$,则 $x>0$ 时,$f(x)$
\begin{enumerate}
\item[A] 有极大值,无极小值.
\item[B] 有极小值,无极大值.
\item[C] 既有极大值,又有极小值.
\item[D] 既无极大值也无极小值.
\end{enumerate}
\end{hypo}
\begin{proof}[\textbf{解}]
答案是 D.我们直接来解微分方程
\begin{equation}
  \label{eq:1}
 \frac{dy}{dx}x^2+2xy=\frac{e^x}{x}.
\end{equation}
将 \eqref{eq:1} 化为
\begin{equation}
  \label{eq:2}
  x^2dy+(2xy-\frac{e^x}{x})dx=0.
\end{equation}
我们发现 \eqref{eq:2} 是一个恰当微分方程.设二元函数 $\phi(x,y)$ 满足
\begin{equation}
  \label{eq:3}
    \frac{\pa\phi}{\pa y}=x^2
\end{equation}
以及
\begin{equation}
  \label{eq:4}
  \frac{\pa\phi}{\pa x}=2xy-\frac{e^x}{x}.
\end{equation}
由 \eqref{eq:3} 可得,
\begin{equation}
  \label{eq:5}
  \phi(x,y)=yx^2+g(x).
\end{equation}
其中 $g(x)$ 是关于 $x$ 的函数.将 \eqref{eq:5} 代入 \eqref{eq:4},可得
\begin{equation}
  \label{eq:6}
  2xy+g'(x)=2xy-\frac{e^x}{x}\ri g'(x)=-\frac{e^x}{x}.
\end{equation}
因此 $g(x)=-\int \frac{e^x}{x}dx+C$,其中 $C$ 是一个常数.因此我们可得通积
分为
$$
\phi(x,y)\equiv yx^2-\int \frac{e^x}{x}dx+C=0.
$$
令 $H(x)=\int \frac{e^x}{x}dx$,则由题目条件可知,
$$
\frac{e^2}{2}-H(2)+C=0.
$$
可见,
$$
y=\frac{H(x)+\frac{e^2}{2}-H(2)}{x^2}=\frac{\int_2^x \frac{e^x}{x}dx+\frac{e^2}{2}}{x^2}.
$$
因此
$$
y'=\frac{e^{x}-2(\int_2^{x} \frac{e^{x}}{x}dx+\frac{e^2}{2})}{x^3}.
$$
下面我们来看函数
$$
p(x)=e^x-2(\int_2^x \frac{e^x}{x}dx+\frac{e^2}{2}).
$$
易得
$$
p'(x)=e^x-\frac{2e^x}{x}.
$$
可见,当 $0<x<2$ 时,$p(x)$ 递减,当 $x\geq 2$ 时,$p(x)$ 递增,且
$p(2)=0$.可见,$y'$ 恒不小于0,且只有在 $x=2$ 处等于0.可见,$y$ 在 $x>0$
时没有极值点.于是选D.
\end{proof}

\section{判断题}
\label{sec:2}
\begin{hypo}
 空间直角坐标系中存在五个不同的点,使得五个点之间的距离(这里的距离,指的是通常的欧氏距
 离,也就是中学几何里的距离)两两相等.
\end{hypo}
\begin{proof}[\textbf{答案}]
  错误.三维空间中最多存在四个距离两两相等的点.
\end{proof}

\section{填空题}
\label{sec:3}
\begin{hypo}
  已知函数 $f:x\to f(x)$ 的定义域是 $[0,1]$,值域是 $\mathbf{R}$.则函数
  $g:x\to f(x+1)$ 的定
  义域是\underline{\hspace{1cm}}.
\end{hypo}
\begin{proof}[\textbf{答案}]
 $[-1,0]$
\end{proof}

\section{两道证明题}
\begin{hypo}
  设$\alpha,\beta$ 是满足等
  式$\sin^{2}\alpha+\sin^{2}\beta=\sin(\alpha+\beta)$的两个锐
  角,证明 $\alpha+\beta=\frac{\pi}{2}$.
\end{hypo}
\begin{proof}[\textbf{解}]
  根据柯西不等式,
\begin{align*}\sin^{2}\alpha+\sin^{2}\beta&=\sin(\alpha+\beta)\\&=\sin\alpha\cos\beta+\sin\beta\cos\alpha\\&\leq
  \sqrt{\sin^{2}\alpha+\sin^2\beta}\sqrt{\cos^{2}\alpha+\cos^{2}\beta},\end{align*}
于是 $\sin^{2}\alpha+\sin^{2}\beta\leq \cos^{2}\alpha+\cos^{2}\beta$,因
此$\cos2\alpha\geq \cos(\pi -2\beta)$,结合单调性
即$2\alpha\geq\pi-2\beta$,即$\alpha+\beta\geq \frac{\pi}{2}$.又因
为$\sin^{2}\alpha+\sin^{2}\beta=\sin(\alpha+\beta)\leq 1$,于是
$\sin\alpha\leq \sin(\frac{\pi}{2}-\beta)$,结合单调性,$\alpha\leq
\frac{\pi}{2}-\beta$,即$\alpha+\beta\leq \frac{\pi}{2}$.故只能是
$\alpha+\beta=\frac{\pi}{2}$.而且易得当 $\alpha+\beta=\frac{\pi}{2}$
时,总会有 $\sin^{2}\alpha+\sin^{2}\beta=\sin(\alpha+\beta)$.证明完毕.
\end{proof}
\begin{hypo}
  能表示成形如 $\frac{p}{q}$ 的实数叫有理数,其中 $p,q\in
  \mathbf{Z}$ 且 $q\neq 0$.不是有理数的实数叫无理数.请证明 $\tan
  1^{\circ}$ 是无理数.
\end{hypo}
\begin{proof}[\textbf{解}]
  反证法.假若 $\tan 1^{\circ}\in \mathbf{Q}$,则 $\tan
  2^{\circ}=\frac{2\tan 1^{\circ}}{1-\tan^{2} 1^{\circ}}\in
  \mathbf{Q}$,$\tan 3^{\circ}=\frac{\tan 1^{\circ}+\tan
    2^{\circ}}{1-\tan 1^{\circ}\tan 2^{\circ}}\in
  \mathbf{Q}$,$\cdots$,$\tan 60^{\circ}\in \mathbf{Q}$.然而$\tan
  60^{\circ}=\frac{\sqrt{2}}{2}$ 是无理数,矛盾.因此假设错误.于是$\tan
  1^{\circ}$ 是无理数.
\end{proof}
\end{document}
