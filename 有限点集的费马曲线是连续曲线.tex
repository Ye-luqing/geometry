\documentclass[a4paper]{article}
\usepackage{amsmath,amsfonts,amsthm,amssymb}
\usepackage{bm}
\usepackage{hyperref}
\usepackage{geometry}
\usepackage{yhmath}
\usepackage{pstricks-add}
\usepackage{framed,mdframed}
\usepackage{graphicx,color} 
\usepackage{mathrsfs,xcolor} 
\usepackage[all]{xy}
\usepackage{fancybox} 
\usepackage{xeCJK}
\newtheorem{theo}{定理}
\newtheorem*{hypo}{猜想}
\newmdtheoremenv{lemma}{引理}
\newmdtheoremenv{corollary}{推论}
\newmdtheoremenv{exercise}{习题}
\newmdtheoremenv{example}{例}
\newenvironment{theorem}
{\bigskip\begin{mdframed}\begin{theo}}
    {\end{theo}\end{mdframed}\bigskip}
\newenvironment{hypothethis}
{\bigskip\begin{mdframed}\begin{hypo}}
    {\end{hypo}\end{mdframed}\bigskip}
\geometry{left=2.5cm,right=2.5cm,top=2.5cm,bottom=2.5cm}
\setCJKmainfont[BoldFont=FZHei-B01S]{FZFangSong-Z02S}
\renewcommand{\today}{\number\year 年 \number\month 月 \number\day 日}
\newcommand{\D}{\displaystyle}\newcommand{\ri}{\Rightarrow}
\newcommand{\ds}{\displaystyle} \renewcommand{\ni}{\noindent}
\newcommand{\pa}{\partial} \newcommand{\Om}{\Omega}
\newcommand{\om}{\omega} \newcommand{\sik}{\sum_{i=1}^k}
\newcommand{\vov}{\Vert\omega\Vert} \newcommand{\Umy}{U_{\mu_i,y^i}}
\newcommand{\lamns}{\lambda_n^{^{\scriptstyle\sigma}}}
\newcommand{\chiomn}{\chi_{_{\Omega_n}}}
\newcommand{\ullim}{\underline{\lim}} \newcommand{\bsy}{\boldsymbol}
\newcommand{\mvb}{\mathversion{bold}} \newcommand{\la}{\lambda}
\newcommand{\La}{\Lambda} \newcommand{\va}{\varepsilon}
\newcommand{\be}{\beta} \newcommand{\al}{\alpha}
\newcommand{\dis}{\displaystyle} \newcommand{\R}{{\mathbb R}}
\newcommand{\N}{{\mathbb N}} \newcommand{\cF}{{\mathcal F}}
\newcommand{\gB}{{\mathfrak B}} \newcommand{\eps}{\epsilon}
\renewcommand\refname{参考文献}\renewcommand\figurename{图}
\usepackage[]{caption2} 
\renewcommand{\captionlabeldelim}{}
\begin{document}
\title{\huge{\bf{平面上有限点集的费马曲线是连续曲线}}}
\author{\small{叶卢庆\footnote{叶卢庆(1992---),男,杭州师范大学理学院数
      学与应用数学专业本科在
      读,E-mail:yeluqingmathematics@gmail.com}}\\{\small{杭州师范大学理
      学院}}}
\maketitle
平面 $\mathbf{R}^2$ 上的有限点
集 $\{P_1,P_2,\cdots,P_n\}$,其中$P_1,P_2,\cdots,P_n\in \mathbf{R}^2$.该
点集的费马曲线,指的是这样的轨迹,该轨迹上的所有点到
点 $P_1,P_2,\cdots,P_n$ 的距离和是一个定
值 $c>0$,且到$P_1,P_2,\cdots,P_n$ 的距离和是一个定值 $c$ 的点全在该轨迹
上.也就是说,该轨迹是这样的集合:
$$
\left\{ |PP_1|+|PP_2|+\cdots+|PP_n|=c:P,P_1,\cdots,P_n\in
  \mathbf{R}^2\right\}
$$
设 $P=(x,y)$ 位于轨迹上,则我们得到
\begin{equation}\label{eq:1}
\sqrt{(x-a_1)^2+(y-b_1)^2}+\sqrt{(x-a_2)^2+(y-b_2)^2}+\cdots+\sqrt{(x-a_n)^2+(y-b_n)^2}=c.
\end{equation}
设
$$
f(x,y)=\sqrt{(x-a_1)^2+(y-b_1)^2}+\sqrt{(x-a_2)^2+(y-b_2)^2}+\cdots+\sqrt{(x-a_n)^2+(y-b_n)^2}.
$$
则 \eqref{eq:1} 会成为$f(x,y)=c$.我们要证明 $f(x,y)=c$ 是一条连续的曲
线.当 $P\not\in \{P_1,\cdots,P_n\}$ 时,我们来看
\begin{equation}
  \label{eq:2}
  \begin{cases}\displaystyle
    \frac{\partial f}{\partial
      x}=\frac{x-a_1}{\sqrt{(x-a_1)^2+(y-b_1)^2}}+\frac{x-a_2}{\sqrt{(x-a_2)^2+(y-b_2)^2}}+\cdots+\frac{x-a_n}{\sqrt{(x-a_n)^2+(y-b_n)^2}}=0,\\
   \displaystyle \frac{\partial f}{\partial
     y}=\frac{y-b_1}{\sqrt{(x-a_1)^2+(y-b_1)^2}}+\frac{y-b_2}{\sqrt{(x-a_2)^2+(y-b_2)^2}}+\cdots+\frac{y-b_n}{\sqrt{(x-a_n)^2+(y-b_n)^2}}=0.
  \end{cases}
\end{equation}
令
$$
\mathbf{L_1}=(x-a_1,y-b_1),\mathbf{L_2}=(x-a_2,y-b_2),\cdots,\mathbf{L_n}=(x-a_n,y-b_n).
$$
则方程组 \eqref{eq:2} 等价于
\begin{equation}\label{eq:3}
\frac{\mathbf{L_1}}{|\mathbf{L_1}|}+\cdots+\frac{\mathbf{L_n}}{|\mathbf{L_n}|}=\mathbf{0}.
\end{equation}
易得 \eqref{eq:3} 式成立当且仅当向量
$$
\frac{\mathbf{L_1}}{|\mathbf{L_1}|},\frac{\mathbf{L_2}}{|\mathbf{L_2}|},\cdots,\frac{\mathbf{L_n}}{|\mathbf{L_n}|}
$$
两两之间的夹角为 $\frac{2\pi}{n}$.
\end{document}








