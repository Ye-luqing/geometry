\documentclass[a4paper, 12pt]{article} % Font size (can be 10pt, 11pt or 12pt) and paper size (remove a4paper for US letter paper)
\usepackage{amsmath,amsfonts,bm}
\usepackage{hyperref,verbatim}
\usepackage{amsthm,epigraph} 
\usepackage{amssymb}
\usepackage{framed,mdframed}
\usepackage{graphicx,color} 
\usepackage{mathrsfs,xcolor} 
\usepackage[all]{xy}
\usepackage{fancybox} 
\usepackage{xeCJK}
\usepackage{pstricks-add}
\pagestyle{empty}
\newrgbcolor{xdxdff}{0.49 0.49 1}
\newrgbcolor{zzttqq}{0.6 0.2 0}
\psset{xunit=0.5cm,yunit=0.5cm}
\newtheorem*{adtheorem}{定理}
\setCJKmainfont[BoldFont=FZYaoTi,ItalicFont=FZYaoTi]{FZYaoTi}
\definecolor{shadecolor}{rgb}{1.0,0.9,0.9} %背景色为浅红色
\newenvironment{theorem}
{\bigskip\begin{mdframed}[backgroundcolor=gray!40,rightline=false,leftline=false,topline=false,bottomline=false]\begin{adtheorem}}
    {\end{adtheorem}\end{mdframed}\bigskip}
\newtheorem*{bdtheorem}{定义}
\newenvironment{definition}
{\bigskip\begin{mdframed}[backgroundcolor=gray!40,rightline=false,leftline=false,topline=false,bottomline=false]\begin{bdtheorem}}
    {\end{bdtheorem}\end{mdframed}\bigskip}
\newtheorem*{cdtheorem}{习题}
\newenvironment{exercise}
{\bigskip\begin{mdframed}[backgroundcolor=gray!40,rightline=false,leftline=false,topline=false,bottomline=false]\begin{cdtheorem}}
    {\end{cdtheorem}\end{mdframed}\bigskip}
\newtheorem*{ddtheorem}{注}
\newenvironment{remark}
{\bigskip\begin{mdframed}[backgroundcolor=gray!40,rightline=false,leftline=false,topline=false,bottomline=false]\begin{ddtheorem}}
    {\end{ddtheorem}\end{mdframed}\bigskip}
\newtheorem*{edtheorem}{引理}
\newenvironment{lemma}
{\bigskip\begin{mdframed}[backgroundcolor=gray!40,rightline=false,leftline=false,topline=false,bottomline=false]\begin{edtheorem}}
    {\end{edtheorem}\end{mdframed}\bigskip}
\newtheorem*{pdtheorem}{例}
\newenvironment{example}
{\bigskip\begin{mdframed}[backgroundcolor=gray!40,rightline=false,leftline=false,topline=false,bottomline=false]\begin{pdtheorem}}
    {\end{pdtheorem}\end{mdframed}\bigskip}

\usepackage[protrusion=true,expansion=true]{microtype} % Better typography
\usepackage{wrapfig} % Allows in-line images
\usepackage{mathpazo} % Use the Palatino font
\usepackage[T1]{fontenc} % Required for accented characters
\linespread{1.05} % Change line spacing here, Palatino benefits from a slight increase by default

\makeatletter
\renewcommand\@biblabel[1]{\textbf{#1.}} % Change the square brackets for each bibliography item from '[1]' to '1.'
\renewcommand{\@listI}{\itemsep=0pt} % Reduce the space between items in the itemize and enumerate environments and the bibliography

\renewcommand{\maketitle}{ % Customize the title - do not edit title
  % and author name here, see the TITLE block
  % below
  \renewcommand\refname{参考文献}
  \newcommand{\D}{\displaystyle}\newcommand{\ri}{\Rightarrow}
  \newcommand{\ds}{\displaystyle} \renewcommand{\ni}{\noindent}
  \newcommand{\pa}{\partial} \newcommand{\Om}{\Omega}
  \newcommand{\om}{\omega} \newcommand{\sik}{\sum_{i=1}^k}
  \newcommand{\vov}{\Vert\omega\Vert} \newcommand{\Umy}{U_{\mu_i,y^i}}
  \newcommand{\lamns}{\lambda_n^{^{\scriptstyle\sigma}}}
  \newcommand{\chiomn}{\chi_{_{\Omega_n}}}
  \newcommand{\ullim}{\underline{\lim}} \newcommand{\bsy}{\boldsymbol}
  \newcommand{\mvb}{\mathversion{bold}} \newcommand{\la}{\lambda}
  \newcommand{\La}{\Lambda} \newcommand{\va}{\varepsilon}
  \newcommand{\be}{\beta} \newcommand{\al}{\alpha}
  \newcommand{\dis}{\displaystyle} \newcommand{\R}{{\mathbb R}}
  \newcommand{\N}{{\mathbb N}} \newcommand{\cF}{{\mathcal F}}
  \newcommand{\gB}{{\mathfrak B}} \newcommand{\eps}{\epsilon}
  \begin{flushright} % Right align
    {\LARGE\@title} % Increase the font size of the title
    
    \vspace{50pt} % Some vertical space between the title and author name
    
    {\large\@author} % Author name
    \\\@date % Date
    
    \vspace{40pt} % Some vertical space between the author block and abstract
  \end{flushright}
}

% ----------------------------------------------------------------------------------------
%	TITLE
% ----------------------------------------------------------------------------------------
\begin{document}
\title{\textbf{初等阵对单位正方形的作用}}
% \setlength\epigraphwidth{0.7\linewidth}
\author{\small{叶卢庆}\\{\small{杭州师范大学理学院,学
      号:1002011005}}\\{\small{Email:h5411167@gmail.com}}} % Institution
\renewcommand{\today}{\number\year. \number\month. \number\day}
\date{\today} % Date
  
% ----------------------------------------------------------------------------------------
  
  
\maketitle % Print the title section
  
% ----------------------------------------------------------------------------------------
% ABSTRACT AND KEYWORDS
% ----------------------------------------------------------------------------------------
  
% \renewcommand{\abstractname}{摘要} % Uncomment to change the name of the abstract to something else
  
% \begin{abstract}
  
% \end{abstract}
  
% \hspace*{3,6mm}\textit{关键词:} % Keywords
  
% \vspace{30pt} % Some vertical space between the abstract and first section
  
% ----------------------------------------------------------------------------------------
% ESSAY BODY
% ----------------------------------------------------------------------------------------
在这篇文章里,我们研究 $n\times n$ 的初等矩阵 $P$ 对 $n$ 维单位正方形的
作用.这里的初等阵 $P$ 代表的是矩阵的初等行变换,且不包括将某一行乘以一个
非零数这种操作.我们希望证明,$n$ 维单位正方形在 $P$ 所代表的
线性映射的作用下,体积不变.\\

为此我们先考虑 $n=2$ 的特殊情形.此时,不妨设单位正方形为 $[0,1]\times
[0,1]$.在此时,若 $P$ 代表的操作是行交换,则 $\mathbf{R}^2$ 中的
点$(x,y)$在 $P$ 的作用下会变成 $(y,x)$.因此单位正方形在 $P$ 的作用下连
位置都没发生变动,更不必说面积了.若 $P$ 代表的操作是某一行乘以一个非零
常数后加到另一行上,不妨设是第一行乘以 $\lambda $ 后加到第二行上,则 $P$
对应的矩阵为
$$
\begin{pmatrix}
  1&0\\
\lambda &1
\end{pmatrix}.
$$
点 $(x,y)$ 在 $P$ 的作用下变为
$$
\begin{pmatrix}
  1&0\\
\lambda&1
\end{pmatrix}\begin{pmatrix}
  x\\
y\\
\end{pmatrix}=\begin{pmatrix}
  x\\
\lambda x+y
\end{pmatrix}.
$$
这个时候, $[0,1]\times [0,1]$ 会如何变化呢?我们画图.作为
$\lambda=\frac{1}{2}$ 的特例,我们发现正方形 $ABCD$ 会成为平行四边形
$AEFD$.显然,此时面积并不发生变化.\\

\begin{pspicture*}(-5.63,-2.42)(8.9,4.06)
\psgrid[subgriddiv=0,gridlabels=0,gridcolor=lightgray](0,0)(-5.63,-2.42)(8.9,4.06)
\psset{xunit=1.0cm,yunit=1.0cm,algebraic=true,dotstyle=o,dotsize=3pt 0,linewidth=0.8pt,arrowsize=3pt 2,arrowinset=0.25}
\psaxes[labelFontSize=\scriptstyle,xAxis=true,yAxis=true,Dx=0.5,Dy=0.5,ticksize=-2pt 0,subticks=2]{->}(0,0)(-2.81,-1.21)(4.45,2.03)
\pspolygon[linecolor=zzttqq,fillcolor=zzttqq,fillstyle=solid,opacity=0.1](0,1)(1,1)(1,0)(0,0)
\pspolygon[linecolor=zzttqq,fillcolor=zzttqq,fillstyle=solid,opacity=0.1](1,1.5)(1,0.5)(0,0)(0,1)
\psline[linecolor=zzttqq](0,1)(1,1)
\psline[linecolor=zzttqq](1,1)(1,0)
\psline[linecolor=zzttqq](1,0)(0,0)
\psline[linecolor=zzttqq](0,0)(0,1)
\psline[linecolor=zzttqq](1,1.5)(1,0.5)
\psline[linecolor=zzttqq](1,0.5)(0,0)
\psline[linecolor=zzttqq](0,0)(0,1)
\psline[linecolor=zzttqq](0,1)(1,1.5)
\begin{scriptsize}
\psdots[dotstyle=*,linecolor=xdxdff](0,1)
\rput[bl](0.02,1.03){\xdxdff{$A$}}
\psdots[dotstyle=*,linecolor=blue](1,1)
\rput[bl](1.02,1.03){\blue{$B$}}
\psdots[dotstyle=*,linecolor=blue](1,0)
\rput[bl](1.02,0.03){\blue{$C$}}
\psdots[dotstyle=*,linecolor=darkgray](0,0)
\rput[bl](0.02,0.03){\darkgray{$D$}}
\psdots[dotstyle=*,linecolor=blue](1,1.5)
\rput[bl](1.02,1.53){\blue{$E$}}
\psdots[dotstyle=*,linecolor=xdxdff](1,0.5)
\rput[bl](1.02,0.53){\xdxdff{$F$}}
\end{scriptsize}
\end{pspicture*}

现在我们探索一般情形.在一般情形,当 $n\times n$ 矩阵 $P$ 对应的操作是第
$i$ 行和第 $j$ 行互换之时,显然 $P$ 会把一个单位正方形变成同一个单位正方形,此时显然面积
不变.当 $P$ 对应的操作是第 $i$ 行乘以一个非零常数 $\lambda$ 加到第 $j$
行上时,$\mathbf{R}^n$ 中的点 $(x_1,x_2,\cdots,x_i,\cdots,x_j,\cdots,x_n)$
在 $P$ 的作用下变成 $(x_1,x_2,\cdots,x_i,\cdots,\lambda
x_i+x_j,\cdots,x_n)$.此时,我们发现,单位正方形在 $P$ 的作用下形成的几何
体在任何一个维度上的横截面的测度相比于对应的单位正方形的面的测度,并没发生变化,因此其体积仍然等于单位正方形体积.
% ----------------------------------------------------------------------------------------
% BIBLIOGRAPHY
% ----------------------------------------------------------------------------------------
  
\bibliographystyle{unsrt}
  
\bibliography{sample}
  
% ----------------------------------------------------------------------------------------

\end{document}