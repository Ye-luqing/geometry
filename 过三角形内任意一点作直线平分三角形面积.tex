\documentclass[a4paper]{article}

\usepackage{amsmath,amsfonts,amsthm,amssymb}

\usepackage{bm}
\usepackage{hyperref}
\usepackage{geometry}
\usepackage{yhmath}
\usepackage{pstricks-add}
\usepackage{framed,mdframed}
\usepackage{graphicx,color} 
\usepackage{mathrsfs,xcolor} 
\usepackage[all]{xy}
\usepackage{fancybox} 
\usepackage{xeCJK}
\newtheorem{theo}{定理}
\newtheorem*{hypo}{猜想}
\newmdtheoremenv{lemma}{引理}
\newmdtheoremenv{corollary}{推论}
\newmdtheoremenv{exercise}{习题}
\newtheorem{rem}{注}
\newenvironment{theorem}
{\bigskip\begin{mdframed}\begin{theo}}
    {\end{theo}\end{mdframed}\bigskip}
\newenvironment{remark}
{\bigskip\begin{mdframed}\begin{rem}}
    {\end{rem}\end{mdframed}\bigskip}
\newenvironment{hypothethis}
{\bigskip\begin{mdframed}\begin{hypo}}
    {\end{hypo}\end{mdframed}\bigskip}
\geometry{left=2.5cm,right=2.5cm,top=2.5cm,bottom=2.5cm}
\setCJKmainfont[BoldFont=FZHei-B01S]{FZFangSong-Z02S}
\renewcommand{\today}{\number\year 年 \number\month 月 \number\day 日}
\newcommand{\D}{\displaystyle}\newcommand{\ri}{\Rightarrow}
\newcommand{\ds}{\displaystyle} \renewcommand{\ni}{\noindent}
\newcommand{\pa}{\partial} \newcommand{\Om}{\Omega}
\newcommand{\om}{\omega} \newcommand{\sik}{\sum_{i=1}^k}
\newcommand{\vov}{\Vert\omega\Vert} \newcommand{\Umy}{U_{\mu_i,y^i}}
\newcommand{\lamns}{\lambda_n^{^{\scriptstyle\sigma}}}
\newcommand{\chiomn}{\chi_{_{\Omega_n}}}
\newcommand{\ullim}{\underline{\lim}} \newcommand{\bsy}{\boldsymbol}
\newcommand{\mvb}{\mathversion{bold}} \newcommand{\la}{\lambda}
\newcommand{\La}{\Lambda} \newcommand{\va}{\varepsilon}
\newcommand{\be}{\beta} \newcommand{\al}{\alpha}
\newcommand{\dis}{\displaystyle} \newcommand{\R}{{\mathbb R}}
\newcommand{\N}{{\mathbb N}} \newcommand{\cF}{{\mathcal F}}
\newcommand{\gB}{{\mathfrak B}} \newcommand{\eps}{\epsilon}
\renewcommand\refname{参考文献}\renewcommand\figurename{图}
\usepackage[]{caption2} 
\renewcommand{\captionlabeldelim}{}
\begin{document}
\title{\huge{\bf{过三角形内任意一点作直线平分三角形面积}}} \author{\small{叶卢
    庆\footnote{叶卢庆(1992---),男,杭州师范大学理学院数学与应用数学专业
      本科在读,E-mail:yeluqingmathematics@gmail.com}}\\{\small{杭州师范大学理学院}}}
\maketitle
如图\eqref{fig:1},非退化三角形 $ABC$.$D$ 是三角形内任意一点.下面我们考察如何作一条通过点
$D$ 的直线,该直线平分三角形 $ABC$ 的面积.
\begin{figure}[h]
\psset{xunit=1.0cm,yunit=1.0cm,algebraic=true,dotstyle=o,dotsize=3pt 0,linewidth=0.8pt,arrowsize=3pt 2,arrowinset=0.25}
\begin{pspicture*}(-0.3,-5.88)(23.02,6.3)
\psline(6.14,4)(2.78,-0.88)
\psline(2.78,-0.88)(11.42,-0.9)
\psline(11.42,-0.9)(6.14,4)
\psplot{-4.3}{23.02}{(-8.28--1.98*x)/0.6}
\begin{scriptsize}
\psdots[dotstyle=*](6.14,4)
\rput[bl](6.06,4.26){{$A$}}
\psdots[dotstyle=*](2.78,-0.88)
\rput[bl](2.28,-1.08){{$B$}}
\psdots[dotstyle=*](11.42,-0.9)
\rput[bl](11.86,-1.06){{$C$}}
\psdots[dotstyle=*](4.18,0)
\rput[bl](4.26,0.12){{$D$}}
\psdots[dotstyle=*](3.58,-1.98)
\rput[bl](3.68,-2){{$N$}}
\psdots[dotstyle=*](5.73,5.13)
\rput[bl](5.75,5.23){{$M$}}
\end{scriptsize}
\end{pspicture*}
  \caption{}
  \label{fig:1}
\end{figure}



首先我们来证明,
\begin{theorem}
  对于三角形内的任意一点$D$,总存在一条通过 $D$ 的直线,该直线平分三角形
  $ABC$ 的面积.
\end{theorem}
\begin{proof}[\textbf{证明}]
如图\eqref{fig:1} 所示.作任意一条过 $D$ 的直线$l$,该直线把三角形分成两
部分,一部分的面积为$S_1$,另一部分的面积为 $S_2$.如果 $S_1=S_2$,则 $l$ 即为所求直线.否则
不妨设 $S_1<S_2$.\\

假想你站在线段 $DN$ 的中点,点 $N$ 在你右边,点 $M$ 在你左边,不妨设你面对的区域的
面积为 $S_2$,背对的区域的面积为 $S_1$.将直线绕着点 $D$ 逆
时针连续旋转 $180^{\circ}$,此时你面对的区域的面积将从 $S_2$ 连续地变成
$S_1$,而你背对的区域的面积将从 $S_1$ 连续地变成 $S_2$.根据连续函数的介
值定理,当直线逆时针旋转某个角度时,你背对的区域和面对的区域的面积会相
同.命题得证.
\end{proof}

\begin{remark}
  通过完全类似的证明,我们有如下命题:对于平面上任意给定的一个点和平面上
  任何一个具有一定面积的几何图形来说,总存在通过该点的直线使得直线把几
  何图形的面积二等分.
\end{remark}
下面我们具体刻画出把三角形面积二等分的直线的性质.首先,直线肯定把三角形 $ABC$
分成一个三角形和另外一个图形.如图\eqref{fig:2}.设直线 $PQ$ 把三角形
$ABC$ 分成两个面积相等的部分.则由正弦定理易得
$$
2|AP||AQ|=|AB||AC|,
$$
即
$$
|AP||AQ|=|AB||AE|,
$$
其中 $E$ 是线段 $AC$ 的中点.也即
$$
\frac{|AP|}{|AB|}=\frac{|AE|}{|AQ|}.
$$
这等价于向量 $\overrightarrow{PE}$ 与向量 $\overrightarrow{BQ}$ 共线.注意,我们根本不怕点 $Q$ 不在线段
$AC$ 上,因为当 $Q$ 不在线段 $AC$ 上时,必定在线段 $AC$ 的延长线上.
\begin{figure}[h]
\psset{xunit=1.0cm,yunit=1.0cm,algebraic=true,dotstyle=o,dotsize=3pt 0,linewidth=0.8pt,arrowsize=3pt 2,arrowinset=0.25}
\begin{pspicture*}(-1.3,-5.88)(23.02,6.3)
\psline(5.96,4.18)(2.64,-2.34)
\psline(5.96,4.18)(11.86,-2.54)
\psline(2.64,-2.34)(11.86,-2.54)
\psline(2.64,-2.34)(8.91,0.82)
\psplot{-4.3}{23.02}{(--6.23-1.04*x)/3.26}
\begin{scriptsize}
\psdots[dotstyle=*](5.96,4.18)
\rput[bl](5.82,4.5){{$A$}}
\psdots[dotstyle=*](2.64,-2.34)
\rput[bl](2.16,-2.54){{$B$}}
\psdots[dotstyle=*](11.86,-2.54)
\rput[bl](12.02,-2.74){{$C$}}
\psdots[dotstyle=*](8.91,0.82)
\rput[bl](8.98,0.94){{$E$}}
\psdots[dotstyle=*](4.74,0.4)
\rput[bl](4.82,0.52){{$D$}}
\psdots[dotstyle=*](4.13,0.59)
\rput[bl](3.86,0.82){{$P$}}
\psdots[dotstyle=*](11.04,-1.61)
\rput[bl](11.12,-1.5){{$Q$}}
\end{scriptsize}
\end{pspicture*}
  \caption{}
  \label{fig:2}
\end{figure}

\end{document}








