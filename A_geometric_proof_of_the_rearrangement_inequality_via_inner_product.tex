\documentclass{amsart}

\usepackage{amsmath}
\usepackage{amsthm}
\usepackage{amsfonts}
\usepackage{amssymb}

\parindent = 5 pt
\parskip = 12 pt

\theoremstyle{plain}
\newtheorem{theorem}{Theorem}
\newtheorem{conjecture}[theorem]{Conjecture}
\newtheorem{problem}[theorem]{Problem}
\newtheorem{exercise}[theorem]{Exercise}
\newtheorem{assumption}[theorem]{Assumption}
\newtheorem{heuristic}[theorem]{Heuristic}
\newtheorem{proposition}[theorem]{Proposition}
\newtheorem{fact}[theorem]{Fact}
\newtheorem{lemma}[theorem]{Lemma}
\newtheorem{corollary}[theorem]{Corollary}
\newtheorem{claim}[theorem]{Claim}
\newtheorem{question}[theorem]{Question}

\theoremstyle{definition}
\newtheorem{definition}[theorem]{Definition}
\newtheorem{example}[theorem]{Example}
\newtheorem{remark}[theorem]{Remark}

\include{psfig}

\begin{document}

\title{A geometrical proof of the rearrangement inequality via inner product}

\author{Luqing Ye}
\address{An undergrad at College of Science, Hangzhou Normal University,Hangzhou City,Zhejiang Province,China}
\email{yeluqingmathematics@gmail.com}
\maketitle
For every choice of real numbers $x_1\leq \cdots\leq x_n$ ,$y_1\leq
\cdots\leq y_n$ and every permutation
$x_{\sigma_{1}},\cdots,x_{\sigma_n}$ of $x_1,\cdots,x_n$,the rearrangement inequality states that
\begin{equation*}
  x_ny_1 + \cdots + x_1y_n
\le x_{\sigma_{1}}y_1 + \cdots + x_{\sigma_{n}}y_{n},
\end{equation*}
the equality holds if and only if
$\sigma_1=n,\cdots,\sigma_n=1$.And
\begin{equation*}
  x_{\sigma_{1}}y_1 + \cdots + x_{\sigma_{n}}y_{n}\leq x_1y_1+\cdots+x_ny_n,
\end{equation*}
the equality holds if and only if $\sigma_1=1,\cdots,\sigma_n$.

\setcounter{tocdepth}{2}
% \tableofcontents




\end{document} 



