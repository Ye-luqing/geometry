\documentclass[a4paper]{article}
\usepackage{amsmath,amsfonts,amsthm,amssymb}
\usepackage{bm}
\usepackage{hyperref}
\usepackage{geometry}
\usepackage{yhmath}
\usepackage{pstricks-add}
\usepackage{framed,mdframed}
\usepackage{graphicx,color} 
\usepackage{mathrsfs,xcolor} 
\usepackage[all]{xy}
\usepackage{fancybox} 
\usepackage{xeCJK}
\newtheorem{theo}{定理}
\newtheorem*{hypo}{猜想}
\newmdtheoremenv{lemma}{引理}
\newmdtheoremenv{corollary}{推论}
\newmdtheoremenv{exercise}{习题}
\newmdtheoremenv{example}{例}
\newenvironment{theorem}
{\bigskip\begin{mdframed}\begin{theo}}
    {\end{theo}\end{mdframed}\bigskip}
\newenvironment{hypothethis}
{\bigskip\begin{mdframed}\begin{hypo}}
    {\end{hypo}\end{mdframed}\bigskip}
\geometry{left=2.5cm,right=2.5cm,top=2.5cm,bottom=2.5cm}
\setCJKmainfont[BoldFont=FZHei-B01S]{FZFangSong-Z02S}
\renewcommand{\today}{\number\year 年 \number\month 月 \number\day 日}
\newcommand{\D}{\displaystyle}\newcommand{\ri}{\Rightarrow}
\newcommand{\ds}{\displaystyle} \renewcommand{\ni}{\noindent}
\newcommand{\pa}{\partial} \newcommand{\Om}{\Omega}
\newcommand{\om}{\omega} \newcommand{\sik}{\sum_{i=1}^k}
\newcommand{\vov}{\Vert\omega\Vert} \newcommand{\Umy}{U_{\mu_i,y^i}}
\newcommand{\lamns}{\lambda_n^{^{\scriptstyle\sigma}}}
\newcommand{\chiomn}{\chi_{_{\Omega_n}}}
\newcommand{\ullim}{\underline{\lim}} \newcommand{\bsy}{\boldsymbol}
\newcommand{\mvb}{\mathversion{bold}} \newcommand{\la}{\lambda}
\newcommand{\La}{\Lambda} \newcommand{\va}{\varepsilon}
\newcommand{\be}{\beta} \newcommand{\al}{\alpha}
\newcommand{\dis}{\displaystyle} \newcommand{\R}{{\mathbb R}}
\newcommand{\N}{{\mathbb N}} \newcommand{\cF}{{\mathcal F}}
\newcommand{\gB}{{\mathfrak B}} \newcommand{\eps}{\epsilon}
\renewcommand\refname{参考文献}\renewcommand\figurename{图}
\usepackage[]{caption2} 
\renewcommand{\captionlabeldelim}{}
\begin{document}
\title{\huge{\bf{平面上的点到平面上三定点距离和的最小值}}} \author{\small{叶卢
    庆\footnote{叶卢庆(1992---),男,杭州师范大学理学院数学与应用数学专业
      本科在读,E-mail:h5411167@gmail.com}}}
\maketitle
设平面上有三点 $(a_1,b_1),(a_2,b_2),(a_3,b_3)$.我们来考察平面上的点到这三点的距
离和的最小值是否存在,以及如果存在,会是多少.设 $(x,y)$ 是平面上的任意一
点,则点 $(x,y)$ 到三个定点的距离和为
\begin{equation}
  \label{eq:1}
f(x,y)= \sqrt{(x-a_{1})^2+(y-b_1)^2}+\sqrt{(x-a_2)^2+(y-b_2)^2}+\sqrt{(x-a_3)^2+(y-b_3)^2}.
\end{equation}
我们来证明 $f(x,y)$ 的最小值存在,并且将其求出.易得
\begin{equation}
  \label{eq:2}
  \frac{\pa f}{\pa x}=\frac{x-a_1}{\sqrt{(x-a_1)^2+(y-b_1)^2}}+\frac{x-a_2}{\sqrt{(x-a_2)^2+(y-b_2)^2}}+\frac{x-a_3}{\sqrt{(x-a_3)^2+(y-b_3)^2}},
\end{equation}
\begin{equation}
  \label{eq:3}
    \frac{\pa f}{\pa y}=\frac{y-b_1}{\sqrt{(x-a_1)^2+(y-b_1)^2}}+\frac{y-b_2}{\sqrt{(x-a_2)^2+(y-b_2)^2}}+\frac{y-b_3}{\sqrt{(x-a_3)^2+(y-b_3)^2}}.
\end{equation}
在平面的某点 $(x_0,y_0)$ 处,有三种情况.
\begin{enumerate}
\item $\frac{\pa f}{\pa y}(x_0,y_{0})\neq 0$.
\item $\frac{\pa f}{\pa y}(x_0,y_0)=0$,$\frac{\pa f}{\pa x}(x_0,y_{0})\neq 0$.
\item $\frac{\pa f}{\pa x}(x_{0},y_{0}),\frac{\pa f}{\pa y}(x_0,y_0)$ 都为 $0$.
\end{enumerate}
当 情况 1 成立时,根据隐函数定理,存在 $(x_0,y_0)$ 的某邻域 $U_{0}$,在该邻域内,$y$
是关于 $x$ 的连续可微的函数 $y=g(x)$.则在该邻域内,
$$
z=f(x,y)=f(x,g(x)),
$$
于是,
$$
\frac{\pa z}{\pa x}=\frac{\pa f}{\pa x}+\frac{\pa f}{\pa y}\frac{\pa
  y}{\pa x}=0.
$$


\end{document}
















