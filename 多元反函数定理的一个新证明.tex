\documentclass[twoside,11pt]{article} 
\usepackage{amsmath,amsfonts,bm}
\usepackage{hyperref}
\usepackage{amsthm} 
\usepackage{amssymb}
\usepackage{framed,mdframed}
\usepackage{graphicx,color} 
\usepackage{mathrsfs,xcolor} 
\usepackage[all]{xy}
\usepackage{fancybox} 
%\usepackage{CJKutf8}
\usepackage{xeCJK}
\newtheorem{theorem}{定理}
\newtheorem{lemma}{引理}
\newtheorem{corollary}{推论}[lemma]
\setCJKmainfont[BoldFont=Adobe Heiti Std R]{Adobe Song Std L}
% \usepackage{latexdef}
\def\ZZ{\mathbb{Z}} \topmargin -0.40in \oddsidemargin 0.08in
\evensidemargin 0.08in \marginparwidth 0.00in \marginparsep 0.00in
\textwidth 16cm \textheight 24cm \newcommand{\D}{\displaystyle}
\newcommand{\ds}{\displaystyle} \renewcommand{\ni}{\noindent}
\newcommand{\pa}{\partial} \newcommand{\Om}{\Omega}
\newcommand{\om}{\omega} \newcommand{\sik}{\sum_{i=1}^k}
\newcommand{\vov}{\Vert\omega\Vert} \newcommand{\Umy}{U_{\mu_i,y^i}}
\newcommand{\lamns}{\lambda_n^{^{\scriptstyle\sigma}}}
\newcommand{\chiomn}{\chi_{_{\Omega_n}}}
\newcommand{\ullim}{\underline{\lim}} \newcommand{\bsy}{\boldsymbol}
\newcommand{\mvb}{\mathversion{bold}} \newcommand{\la}{\lambda}
\newcommand{\La}{\Lambda} \newcommand{\va}{\varepsilon}
\newcommand{\be}{\beta} \newcommand{\al}{\alpha}
\newcommand{\dis}{\displaystyle} \newcommand{\R}{{\mathbb R}}
\newcommand{\N}{{\mathbb N}} \newcommand{\cF}{{\mathcal F}}
\newcommand{\gB}{{\mathfrak B}} \newcommand{\eps}{\epsilon}
\renewcommand\refname{参考文献} \def \qed {\hfill \vrule height6pt
  width 6pt depth 0pt} \topmargin -0.40in \oddsidemargin 0.08in
\evensidemargin 0.08in \marginparwidth0.00in \marginparsep 0.00in
\textwidth 15.5cm \textheight 24cm \pagestyle{myheadings}
\markboth{\rm \centerline{}} {\rm \centerline{}}
\begin{document}
\title{\huge{\textbf{多元反函数定理的一个证明}}} \author{\small{叶卢
    庆\footnote{叶卢庆(1992---),男,杭州师范大学理学院数学与应用数学专业
      本科在读,E-mail:h5411167@gmail.com}}\\{\small{杭州师范大学理学院,浙
      江~杭州~310036}}} \date{}
\maketitle
  
% ----------------------------------------------------------------------------------------
% ABSTRACT AND KEYWORDS
% ----------------------------------------------------------------------------------------


\textbf{\small{摘要}:}仅利用微分中值定理,以及微扰不改变一个可逆矩阵的
可逆性,给出了多元反函数定理的一个证明. \smallskip

\textbf{\small{关键词}:}多元反函数定理;微分中值定理;可逆矩阵\smallskip

\textbf{\small{中图分类号}:}O172.1
  
\vspace{30pt} % Some vertical space between the abstract and first section
  
% ----------------------------------------------------------------------------------------
%	ESSAY BODY
% ----------------------------------------------------------------------------------------
多元反函数定理是多元微分学中的核心定理之一.利用它能直接推出隐函数定理.其
叙述如下:
\begin{theorem}[多元反函数定理]
  设 $E$ 是 $\mathbf{R}^n$ 的开集合,并设 $T:E\to
  \mathbf{R}^n$ 是在 $E$上连续可微的函数.假设 $\mathbf{x_0}\in E$ 使得
  线性变换$f'(\mathbf{x_0}):\mathbf{R}^n\to \mathbf{R}^n$ 是可逆的,那么
  存在含有$\mathbf{x_0}$ 的开集 $U\subset E$ 以及含
  有 $f(\mathbf{x_0})$ 的开集$V\subset \mathbf{R}^n$,使得 $f$ 是从 $U$
  到 $V$ 的双射,而且逆映射$f^{-1}:V\to U$ 在点 $f(\mathbf{x_0})$ 处可
  微,而且
$$
(f^{-1})'(f(\mathbf{x_0}))=(f'(\mathbf{x_0}))^{-1}.
$$
\end{theorem}
\bigskip\bigskip
现在,笔者来阐述自己发现的证明,这种证明只用到了微分中值定理以及简单的矩阵知
识.为此,我们先来看一个引理:
\begin{lemma}[微扰不改变可逆矩阵的可逆性]
  设 $A_{n,n}$ 是一个 $n$ 行 $n$ 列的可逆矩阵,其第 $i$ 行,第 $j$ 列的项
  记为 $a_{ij}$.则存在 $\varepsilon>0$,使得 $\forall 0\leq
  \delta_{ij}<\varepsilon$,矩阵
$$
B=\begin{pmatrix}
  a_{11}+\delta_{11}&a_{12}+\delta_{12}&\cdots&a_{1n}+\delta_{1n}\\
  a_{21}+\delta_{21}&a_{22}+\delta_{22}&\cdots&a_{2n}+\delta_{2n}\\
  \vdots&\vdots&\vdots&\vdots\\
  a_{n1}+\delta_{n1}&a_{n2}+\delta_{n2}&\cdots&a_{nn}+\delta_{nn}\\
\end{pmatrix}
$$
可逆.
\end{lemma}
\begin{proof}[引理证明]
 我们来看 $n^2$ 元函数 $\det A_{n,n}$,该函数的 $n^2$ 个自变量分别是矩阵
 $A_{n,n}$ 中的各个项,易得该 $n^2$ 元函数关于各个自变量连续.当矩阵 $A$
 可逆时,$\det A_{n,n}\neq 0$.此时对于每个变量 $a_{ij}$ 来说,存在
 $\va_{ij}>0$,使得 $\forall 0\leq\delta_{ij}<\va_{ij}$,当 $a_{ij}$ 被
 $a_{ij}+\delta_{ij}$ 替代时,$\det A_{n,n}$ 依然非零,而且正负符号和原
 来的 $\det A_{n,n}$ 相比没有变号.

令 $\va=\min
\{\va_{11},\va_{12},\cdots,\va_{1n},\va_{21},\va_{22},\cdots,\va_{2n},\cdots,\va_{n1},\va_{n2},\cdots,\va_{nn}\}$,
且矩阵可逆当且仅当行列式非零,即可得引理.
\end{proof}
\bigskip
引理1有如下推论:

\begin{corollary}
存在含有 $\mathbf{x_0}$ 的开集 $U'$,使得 $f$ 在 $U'$ 上的每一点处的导数可逆.
\end{corollary}

下面我们来证明 $f(U')$ 也是 $\mathbf{R}^n$ 中的一个开集.为此,我们只用
证明如下结论:

\begin{lemma}
对于 $f(U')$ 中的任意一个点 $\mathbf{x'}$,当该点的邻域足够小时,整个邻域都会含于 $f(U')$.  
\end{lemma}
\begin{proof}[引理证明]
 
\end{proof}

% BIBLIOGRAPHY
% ----------------------------------------------------------------------------------------
%
%  \begin{thebibliography}{}
%    \small{
%    \bibitem[1]{tao}Terence Tao.陶哲轩实分析[M].王昆扬,译.北京:人民邮电
%      出版社,2008:376
%    \bibitem[2]{rudin}Walter Rudin.数学分析原理[M].赵慈庚,蒋铎,译.原书
%      第3版.北京:机械工业出版社,2004:202 

%  \bibitem[3]{apostol}Tom M.Apostol .数学分析[M].邢富冲,邢辰,李松洁,贾
%    婉丽,译.原书第2版.北京:机械工业出版社,2006:302-303
%}
%  \end{thebibliography}
  % ----------------------------------------------------------------------------------------
\end{document}










