\documentclass[a4paper]{article}
\usepackage{amsmath,amsfonts,amsthm,amssymb}
\usepackage{bm}
\usepackage{draftwatermark,euler}
\SetWatermarkText{http://blog.sciencenet.cn/u/Yaleking}%设置水印文字
\SetWatermarkLightness{0.8}%设置水印亮度
\SetWatermarkScale{0.35}%设置水印大小
\usepackage{hyperref}
\usepackage{geometry}
\usepackage{yhmath}
\usepackage{pstricks-add}
\usepackage{framed,mdframed}
\usepackage{graphicx,color} 
\usepackage{mathrsfs,xcolor} 
\usepackage[all]{xy}
\usepackage{fancybox} 
\usepackage{xeCJK}
\newtheorem*{theo}{定理}
\newtheorem*{exe}{题目}
\newtheorem*{rem}{评论}
\newmdtheoremenv{lemma}{引理}
\newmdtheoremenv{corollary}{推论}
\newmdtheoremenv{example}{例}
\newenvironment{theorem}
{\bigskip\begin{mdframed}\begin{theo}}
    {\end{theo}\end{mdframed}\bigskip}
\newenvironment{exercise}
{\bigskip\begin{mdframed}\begin{exe}}
    {\end{exe}\end{mdframed}\bigskip}
\geometry{left=2.5cm,right=2.5cm,top=2.5cm,bottom=2.5cm}
\setCJKmainfont[BoldFont=SimHei]{SimSun}
\renewcommand{\today}{\number\year 年 \number\month 月 \number\day 日}
\newcommand{\D}{\displaystyle}\newcommand{\ri}{\Rightarrow}
\newcommand{\ds}{\displaystyle} \renewcommand{\ni}{\noindent}
\newcommand{\ov}{\overrightarrow}
\newcommand{\pa}{\partial} \newcommand{\Om}{\Omega}
\newcommand{\om}{\omega} \newcommand{\sik}{\sum_{i=1}^k}
\newcommand{\vov}{\Vert\omega\Vert} \newcommand{\Umy}{U_{\mu_i,y^i}}
\newcommand{\lamns}{\lambda_n^{^{\scriptstyle\sigma}}}
\newcommand{\chiomn}{\chi_{_{\Omega_n}}}
\newcommand{\ullim}{\underline{\lim}} \newcommand{\bsy}{\boldsymbol}
\newcommand{\mvb}{\mathversion{bold}} \newcommand{\la}{\lambda}
\newcommand{\La}{\Lambda} \newcommand{\va}{\varepsilon}
\newcommand{\be}{\beta} \newcommand{\al}{\alpha}
\newcommand{\dis}{\displaystyle} \newcommand{\R}{{\mathbb R}}
\newcommand{\N}{{\mathbb N}} \newcommand{\cF}{{\mathcal F}}
\newcommand{\gB}{{\mathfrak B}} \newcommand{\eps}{\epsilon}
\renewcommand\refname{参考文献}\renewcommand\figurename{图}
\usepackage[]{caption2} 
\renewcommand{\captionlabeldelim}{}
\begin{document}
\title{\huge{\bf{吕林根,许子道《解析几何》习题4.7.5}}} \author{\small{叶卢庆\footnote{叶卢庆(1992---),男,杭州师范大学理学院数学与应用数学专业本科在读,E-mail:yeluqingmathematics@gmail.com}}}
\maketitle
\begin{exercise}
  求与两直线$\frac{x-6}{3}=\frac{y}{2}=\frac{z-1}{1}$与
  $\frac{x}{3}=\frac{y-8}{2}=\frac{z+4}{-2}$相交,而且与平面$2x+3y-5=0$
  平行的直线的轨迹.
\end{exercise}
\begin{proof}[\textbf{解}]
设直线为
\begin{equation}\label{eq:1}
\frac{x-p_1}{\lambda_1}=\frac{y-p_2}{\lambda_2}=\frac{z-p_3}{\lambda_3}.
\end{equation}
由直线\eqref{eq:1}与平面$2x+3y-5$平行可得
\begin{equation}
  \label{eq:2}
  2\lambda_1+3\lambda_2=0.
\end{equation}
直线\eqref{eq:1}与直线 $\frac{x-6}{3}=\frac{y}{2}=\frac{z-1}{1}$相交,
意味着
$$
\begin{vmatrix}
  \lambda_1&\lambda_2&\lambda_3\\
3&2&1\\
p_1-6&p_2&p_3-1
\end{vmatrix}=0,
$$
也即
\begin{equation}
  \label{eq:3}
  (2p_3-p_2-2)\lambda_1+(p_1-3p_3-3)\lambda_2+(3p_2-2p_1+12)\lambda_3=0.
\end{equation}
式\eqref{eq:3}结合式\eqref{eq:2},可得
\begin{equation}
  \label{eq:3.1}
 \lambda_2(2p_1+3p_2-12p_3)+\lambda_3(6p_2-4p_1+24)=0.
\end{equation}
直线\eqref{eq:1}与直线$\frac{x}{3}=\frac{y-8}{2}=\frac{z+4}{-2}$相交,
意味着
$$
\begin{vmatrix}
  \lambda_1&\lambda_2&\lambda_3\\
3&2&-2\\
p_1&p_2-8&p_3+4
\end{vmatrix}=0,
$$
也即
\begin{equation}
  \label{eq:4}
  (2p_3+2p_2-8)\lambda_1-(2p_1+3p_3+12)\lambda_2+(3p_2-2p_1-24)\lambda_3=0.
\end{equation}
式\eqref{eq:4}结合式\eqref{eq:2}可得
\begin{equation}
  \label{eq:5}
  \lambda_2(-6p_3-3p_2-2p_1)+(3p_2-2p_1-24)\lambda_3=0.
\end{equation}
方程\eqref{eq:3.1}和\eqref{eq:5}联立表明
$$
\begin{cases}
  (3p_2-2p_1)\lambda_3=6p_3\lambda_2,\\
-24\lambda_3=(2p_1+3p_2)\lambda_2.
\end{cases}
$$
显然$\lambda_2\neq 0$,因此
\begin{equation}
  \label{eq:6}
  9p_2^2-4p_1^2+144p_3=0.
\end{equation}
于是,直线的轨迹就是
$$
9y^2-4x^2+144z=0.
$$
\end{proof}
\end{document}
