\documentclass[a4paper]{article}
\usepackage{amsmath,amsfonts,amsthm,amssymb}
\usepackage{bm}
\usepackage{hyperref}
\usepackage{geometry}
\usepackage{yhmath}
\usepackage{pstricks-add}
\usepackage{framed,mdframed}
\usepackage{graphicx,color} 
\usepackage{mathrsfs,xcolor} 
\usepackage[all]{xy}
\usepackage{fancybox} 
\usepackage{xeCJK}
\newtheorem*{theo}{定理}
\newtheorem*{exe}{题目}
\newtheorem*{rem}{评论}
\newtheorem*{lemma}{引理}
\newtheorem*{coro}{推论}
\newtheorem*{exa}{例}
\newenvironment{corollary}
{\bigskip\begin{mdframed}\begin{coro}}
    {\end{coro}\end{mdframed}\bigskip}
\newenvironment{theorem}
{\bigskip\begin{mdframed}\begin{theo}}
    {\end{theo}\end{mdframed}\bigskip}
\newenvironment{exercise}
{\bigskip\begin{mdframed}\begin{exe}}
    {\end{exe}\end{mdframed}\bigskip}
\newenvironment{example}
{\bigskip\begin{mdframed}\begin{exa}}
    {\end{exa}\end{mdframed}\bigskip}
\newenvironment{remark}
{\bigskip\begin{mdframed}\begin{rem}}
    {\end{rem}\end{mdframed}\bigskip}
\geometry{left=2.5cm,right=2.5cm,top=2.5cm,bottom=2.5cm}
\setCJKmainfont[BoldFont=SimHei]{SimSun}
\renewcommand{\today}{\number\year 年 \number\month 月 \number\day 日}
\newcommand{\D}{\displaystyle}\newcommand{\ri}{\Rightarrow}
\newcommand{\ds}{\displaystyle} \renewcommand{\ni}{\noindent}
\newcommand{\ov}{\overrightarrow}
\newcommand{\pa}{\partial} \newcommand{\Om}{\Omega}
\newcommand{\om}{\omega} \newcommand{\sik}{\sum_{i=1}^k}
\newcommand{\vov}{\Vert\omega\Vert} \newcommand{\Umy}{U_{\mu_i,y^i}}
\newcommand{\lamns}{\lambda_n^{^{\scriptstyle\sigma}}}
\newcommand{\chiomn}{\chi_{_{\Omega_n}}}
\newcommand{\ullim}{\underline{\lim}} \newcommand{\bsy}{\boldsymbol}
\newcommand{\mvb}{\mathversion{bold}} \newcommand{\la}{\lambda}
\newcommand{\La}{\Lambda} \newcommand{\va}{\varepsilon}
\newcommand{\be}{\beta} \newcommand{\al}{\alpha}
\newcommand{\dis}{\displaystyle} \newcommand{\R}{{\mathbb R}}
\newcommand{\N}{{\mathbb N}} \newcommand{\cF}{{\mathcal F}}
\newcommand{\gB}{{\mathfrak B}} \newcommand{\eps}{\epsilon}
\renewcommand\refname{参考文献}\renewcommand\figurename{图}
\usepackage[]{caption2} 
\renewcommand{\captionlabeldelim}{}
\setlength\parindent{0pt}
\begin{document}
\title{\huge{\bf{习题4.4.6}}} \author{\small{叶卢庆\footnote{叶卢庆(1992---),男,杭州师范大学理学院数学与应用数学专业本科在读,E-mail:yeluqingmathematics@gmail.com}}}
\maketitle
\begin{exercise}
  已知椭球面
  $\frac{x^2}{a^2}+\frac{y^2}{b^2}+\frac{z^2}{c^2}=1(c<a<b)$,试求过$x$
  轴并与曲面的交线是圆的平面.
\end{exercise}
\begin{proof}[\textbf{解}]
设平面的方程为
$$
z=py.
$$
则要使得平面与椭球面的交线是圆,必须使$p\neq 0$.将平面方程代入椭球面
方程,可得曲线的方程为
$$
\begin{cases}
  \frac{x^2}{a^2}+\frac{y^2}{b^2}+\frac{z^2}{c^2}=1,\\
z=py
\end{cases}.
$$
即
\begin{equation}\label{eq:1}
\begin{cases}
  \frac{x^2}{a^2}+(\frac{1}{b^2}+\frac{p^2}{c^2})y^2=1,\\
z=py.
\end{cases}
\end{equation}
在$XOY$平面上,
$$
\frac{x^2}{a^2}+(\frac{1}{b^2}+\frac{p^2}{c^2})y^2=1
$$
的两条半轴长度分别为
$$
a,\frac{1}{\sqrt{\frac{1}{b^2}+\frac{p^2}{c^2}}}.
$$
两条半轴依次投影到$z=py$平面上,得到的线段的长度的平方依次为
$$
a^{2},(p\frac{1}{\sqrt{\frac{1}{b^2}+\frac{p^2}{c^2}}})^{2}+(\frac{1}{\sqrt{\frac{1}{b^2}+\frac{p^2}{c^2}}})^{2}.
$$
令
$$
a^{2}=(p\frac{1}{\sqrt{\frac{1}{b^2}+\frac{p^2}{c^2}}})^{2}+(\frac{1}{\sqrt{\frac{1}{b^2}+\frac{p^2}{c^2}}})^{2}.
$$
解得
$$
p\pm \frac{c}{b} \sqrt{\frac{b^2-a^2}{a^2-c^2}}.
$$
这样我们就确定了平面,使得曲线\eqref{eq:1}是一个圆.
\end{proof}
\end{document}
