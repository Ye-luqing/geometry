\documentclass[a4paper]{article}
\usepackage{amsmath,amsfonts,amsthm,amssymb}
\usepackage{bm}
\usepackage{draftwatermark,euler}
\SetWatermarkText{http://blog.sciencenet.cn/u/Yaleking}%设置水印文字
\SetWatermarkLightness{0.8}%设置水印亮度
\SetWatermarkScale{0.35}%设置水印大小
\usepackage{hyperref}
\usepackage{geometry}
\usepackage{yhmath}
\usepackage{pstricks-add}
\usepackage{framed,mdframed}
\usepackage{graphicx,color} 
\usepackage{mathrsfs,xcolor} 
\usepackage[all]{xy}
\usepackage{fancybox} 
\usepackage{xeCJK}
\newtheorem*{theo}{定理}
\newtheorem*{exe}{题目}
\newtheorem*{rem}{评论}
\newmdtheoremenv{lemma}{引理}
\newmdtheoremenv{corollary}{推论}
\newmdtheoremenv{example}{例}
\newenvironment{theorem}
{\bigskip\begin{mdframed}\begin{theo}}
    {\end{theo}\end{mdframed}\bigskip}
\newenvironment{exercise}
{\bigskip\begin{mdframed}\begin{exe}}
    {\end{exe}\end{mdframed}\bigskip}
\geometry{left=2.5cm,right=2.5cm,top=2.5cm,bottom=2.5cm}
\setCJKmainfont[BoldFont=SimHei]{SimSun}
\renewcommand{\today}{\number\year 年 \number\month 月 \number\day 日}
\newcommand{\D}{\displaystyle}\newcommand{\ri}{\Rightarrow}
\newcommand{\ds}{\displaystyle} \renewcommand{\ni}{\noindent}
\newcommand{\ov}{\overrightarrow}
\newcommand{\pa}{\partial} \newcommand{\Om}{\Omega}
\newcommand{\om}{\omega} \newcommand{\sik}{\sum_{i=1}^k}
\newcommand{\vov}{\Vert\omega\Vert} \newcommand{\Umy}{U_{\mu_i,y^i}}
\newcommand{\lamns}{\lambda_n^{^{\scriptstyle\sigma}}}
\newcommand{\chiomn}{\chi_{_{\Omega_n}}}
\newcommand{\ullim}{\underline{\lim}} \newcommand{\bsy}{\boldsymbol}
\newcommand{\mvb}{\mathversion{bold}} \newcommand{\la}{\lambda}
\newcommand{\La}{\Lambda} \newcommand{\va}{\varepsilon}
\newcommand{\be}{\beta} \newcommand{\al}{\alpha}
\newcommand{\dis}{\displaystyle} \newcommand{\R}{{\mathbb R}}
\newcommand{\N}{{\mathbb N}} \newcommand{\cF}{{\mathcal F}}
\newcommand{\gB}{{\mathfrak B}} \newcommand{\eps}{\epsilon}
\renewcommand\refname{参考文献}\renewcommand\figurename{图}
\usepackage[]{caption2} 
\renewcommand{\captionlabeldelim}{}
\begin{document}
\title{\huge{\bf{吕林根,许子道《解析几何》习题4.7.10}}} \author{\small{叶卢庆\footnote{叶卢庆(1992---),男,杭州师范大学理学院数学与应用数学专业本科在读,E-mail:yeluqingmathematics@gmail.com}}}
\maketitle
\begin{exercise}
  已知空间两异面直线间的距离为$2a$,夹角为$2\theta$,过这两直线分别作平
  面,并使这两平面互相垂直.求这样的两平面交线的轨迹.
\end{exercise}
\begin{proof}[\textbf{解}]
设其中一条直线的方程为$\frac{x-0}{1}=\frac{y-0}{0}=\frac{z-0}{0}$.另一
条直线的方程为$\frac{x-0}{\cos2\theta}=\frac{y-0}{\sin
  2\theta}=\frac{z-2a}{0}$.其中$0\leq\theta\leq \frac{\pi}{2}$.过直线
$\frac{x-0}{1}=\frac{y-0}{0}=\frac{z-0}{0}$的平面设为
\begin{equation}
  \label{eq:1}
  my+nz=0.
\end{equation}
过直线$\frac{x-0}{\cos2\theta}=\frac{y-0}{\sin
  2\theta}=\frac{z-2a}{0}$的平面设为
\begin{equation}
  \label{eq:2}
  -\sin 2\theta x+\cos 2\theta y+Cz-2aC=0.
\end{equation}
平面\eqref{eq:1}和平面\eqref{eq:2}垂直,当且仅当
\begin{equation}
  \label{eq:3}
  m\cos2\theta+Cn=0.
\end{equation}
于是平面\eqref{eq:1}的方程可以化为
\begin{equation}
  \label{eq:4}
  -Cy+\cos2\theta z=0.
\end{equation}
结合式\eqref{eq:4}和式\eqref{eq:1},可得平面的交线运动形成的轨迹为
$$
z^2\cos2\theta-2az\cos2\theta+y^2\cos2\theta-xy\sin 2\theta=0.
$$
当$\theta=\frac{\pi}{4}$时,可得轨迹为$xy=0$,这是两个坐标平面.当
$\theta\neq \frac{\pi}{4}$时,可得轨迹为
\begin{equation}\label{eq:5}
(z-a)^2+y^2-xy\tan2\theta=a^2.
\end{equation}
为了判断\eqref{eq:5}的类型,我们考虑以$z$坐标轴为转轴旋转$xOy$平面.令
$$
\begin{cases}
  z=z',\\
x=x'\cos\phi-y'\sin\phi,\\
y=x'\sin\phi+y'\cos\phi.
\end{cases}
$$
代入式\eqref{eq:5},要消去交叉项$x'y'$,让$\phi=\theta$,可得方程
\eqref{eq:5}会变成
$$
(z'-a)^2+x'^2(\sin^2\theta-\frac{1}{2}\sin2\theta\tan2\theta)+y'^2(\cos^2\theta+\frac{1}{2}\sin
2\theta\tan2\theta)=a^2.
$$
当$\sin^2\theta-\frac{1}{2}\sin2\theta\tan2\theta=0$,即$\theta=0$或
$\frac{\pi}{2}$或$\arctan \frac{\sqrt{5}-1}{2}$时,曲面是柱面.\\
当$\cos^2\theta-\frac{1}{2}\sin2\theta\tan2\theta=0$,即
$\theta=\frac{\pi}{2}$时,曲面是柱面.\\
当$\sin^2\theta-\frac{1}{2}\sin2\theta\tan2\theta>0$且
$\cos^2\theta-\frac{1}{2}\sin2\theta\tan2\theta>0$时,曲面是椭球面.\\
当$\sin^2\theta-\frac{1}{2}\sin2\theta\tan2\theta<0$且
$\cos^2\theta-\frac{1}{2}\sin2\theta\tan2\theta<0$时,曲面是双叶双曲
面.\\
当
$(\sin^2\theta-\frac{1}{2}\sin2\theta\tan2\theta)(\cos^2\theta-\frac{1}{2}\sin2\theta\tan2\theta)<0$
时,曲面是单叶双曲面.
\end{proof}
\end{document}
