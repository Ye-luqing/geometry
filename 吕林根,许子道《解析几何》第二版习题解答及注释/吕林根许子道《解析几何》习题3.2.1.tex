\documentclass[a4paper]{article}
\usepackage{amsmath,amsfonts,amsthm,amssymb} \usepackage{bm}
\usepackage{euler} \usepackage{hyperref} \usepackage{geometry}
\usepackage{yhmath} \usepackage{pstricks-add}
\usepackage{framed,mdframed} \usepackage{graphicx,color}
\usepackage{mathrsfs,xcolor} \usepackage[all]{xy}
\usepackage{fancybox} \usepackage{xeCJK} \newtheorem{theo}{定理}
\newtheorem*{exe}{题目} \newtheorem*{rem}{评论}
\newmdtheoremenv{lemma}{引理} \newmdtheoremenv{corollary}{推论}
\newmdtheoremenv{example}{例} \newenvironment{theorem}
{\bigskip\begin{mdframed}\begin{theo}}
    {\end{theo}\end{mdframed}\bigskip} \newenvironment{exercise}
{\bigskip\begin{mdframed}\begin{exe}}
    {\end{exe}\end{mdframed}\bigskip}
\geometry{left=2.5cm,right=2.5cm,top=2.5cm,bottom=2.5cm}
\setCJKmainfont[BoldFont=SimHei]{SimSun}
\renewcommand{\today}{\number\year 年 \number\month 月 \number\day 日}
\newcommand{\D}{\displaystyle}\newcommand{\ri}{\Rightarrow}
\newcommand{\ds}{\displaystyle} \renewcommand{\ni}{\noindent}
\newcommand{\pa}{\partial} \newcommand{\Om}{\Omega}
\newcommand{\om}{\omega} \newcommand{\sik}{\sum_{i=1}^k}
\newcommand{\vov}{\Vert\omega\Vert} \newcommand{\Umy}{U_{\mu_i,y^i}}
\newcommand{\lamns}{\lambda_n^{^{\scriptstyle\sigma}}}
\newcommand{\chiomn}{\chi_{_{\Omega_n}}}
\newcommand{\ullim}{\underline{\lim}} \newcommand{\bsy}{\boldsymbol}
\newcommand{\mvb}{\mathversion{bold}} \newcommand{\la}{\lambda}
\newcommand{\La}{\Lambda} \newcommand{\va}{\varepsilon}
\newcommand{\be}{\beta} \newcommand{\al}{\alpha}
\newcommand{\dis}{\displaystyle} \newcommand{\R}{{\mathbb R}}
\newcommand{\N}{{\mathbb N}} \newcommand{\cF}{{\mathcal F}}
\newcommand{\gB}{{\mathfrak B}} \newcommand{\eps}{\epsilon}
\renewcommand\refname{参考文献}\renewcommand\figurename{图}
\usepackage[]{caption2} \renewcommand{\captionlabeldelim}{}
\begin{document}
\title{\huge{\bf{吕林根,许子道《解析几何》习题3.2.1}}} \author{\small{叶卢
    庆\footnote{叶卢庆(1992---),男,杭州师范大学理学院数学与应用数学专业
      本科在读,E-mail:yeluqingmathematics@gmail.com}}}
\maketitle
\begin{exercise}
计算下列点和平面间的离差和距离.
\begin{itemize}
\item $M(-2,4,3)$,$\pi:2x-y+2z+3=0$.
\item $M(1,2-3),\pi:5x-3y+z+4=0$.
\end{itemize}
\end{exercise}
\begin{itemize}
\item 首先将平面$\pi$化为法式方程
$$
-\frac{2}{3}x+\frac{1}{3}y-\frac{2}{3}z-1=0
$$
可见,点$M$和平面之间的离差为
$$
(\frac{-2}{3},\frac{1}{3},\frac{-2}{3})\cdot (-2,4,3)-1=\frac{-1}{3}.
$$
可见点$M$到平面的距离为$\frac{1}{3}$.
\item 首先将平面$\pi$化为法式方程
$$
-\frac{5}{\sqrt{35}}x+\frac{3}{\sqrt{35}}y-\frac{1}{\sqrt{35}}z-\frac{4}{\sqrt{35}}=0.
$$
可见,点$M$和平面之间的离差为
$$
(\frac{-5}{\sqrt{35}},\frac{3}{\sqrt{35}},\frac{-1}{\sqrt{35}})\cdot (1,2,-3)-\frac{4}{\sqrt{35}}=0.
$$
可见,点$M$和平面之间的距离为$0$.
\end{itemize}
\end{document}
