\documentclass[a4paper]{article}
\usepackage{amsmath,amsfonts,amsthm,amssymb}
\usepackage{bm}
\usepackage{draftwatermark,euler}
\SetWatermarkText{http://blog.sciencenet.cn/u/Yaleking}%设置水印文字
\SetWatermarkLightness{0.8}%设置水印亮度
\SetWatermarkScale{0.35}%设置水印大小
\usepackage{hyperref}
\usepackage{geometry}
\usepackage{yhmath}
\usepackage{pstricks-add}
\usepackage{framed,mdframed}
\usepackage{graphicx,color} 
\usepackage{mathrsfs,xcolor} 
\usepackage[all]{xy}
\usepackage{fancybox} 
\usepackage{xeCJK}
\newtheorem*{theo}{定理}
\newtheorem*{exe}{题目}
\newtheorem*{rem}{评论}
\newmdtheoremenv{lemma}{引理}
\newmdtheoremenv{corollary}{推论}
\newmdtheoremenv{example}{例}
\newenvironment{theorem}
{\bigskip\begin{mdframed}\begin{theo}}
    {\end{theo}\end{mdframed}\bigskip}
\newenvironment{exercise}
{\bigskip\begin{mdframed}\begin{exe}}
    {\end{exe}\end{mdframed}\bigskip}
\geometry{left=2.5cm,right=2.5cm,top=2.5cm,bottom=2.5cm}
\setCJKmainfont[BoldFont=SimHei]{SimSun}
\renewcommand{\today}{\number\year 年 \number\month 月 \number\day 日}
\newcommand{\D}{\displaystyle}\newcommand{\ri}{\Rightarrow}
\newcommand{\ds}{\displaystyle} \renewcommand{\ni}{\noindent}
\newcommand{\ov}{\overrightarrow}
\newcommand{\pa}{\partial} \newcommand{\Om}{\Omega}
\newcommand{\om}{\omega} \newcommand{\sik}{\sum_{i=1}^k}
\newcommand{\vov}{\Vert\omega\Vert} \newcommand{\Umy}{U_{\mu_i,y^i}}
\newcommand{\lamns}{\lambda_n^{^{\scriptstyle\sigma}}}
\newcommand{\chiomn}{\chi_{_{\Omega_n}}}
\newcommand{\ullim}{\underline{\lim}} \newcommand{\bsy}{\boldsymbol}
\newcommand{\mvb}{\mathversion{bold}} \newcommand{\la}{\lambda}
\newcommand{\La}{\Lambda} \newcommand{\va}{\varepsilon}
\newcommand{\be}{\beta} \newcommand{\al}{\alpha}
\newcommand{\dis}{\displaystyle} \newcommand{\R}{{\mathbb R}}
\newcommand{\N}{{\mathbb N}} \newcommand{\cF}{{\mathcal F}}
\newcommand{\gB}{{\mathfrak B}} \newcommand{\eps}{\epsilon}
\renewcommand\refname{参考文献}\renewcommand\figurename{图}
\usepackage[]{caption2} 
\renewcommand{\captionlabeldelim}{}
\begin{document}
\title{\huge{\bf{吕林根,许子道《解析几何》习题4.7.9}}} \author{\small{叶卢庆\footnote{叶卢庆(1992---),男,杭州师范大学理学院数学与应用数学专业本科在读,E-mail:yeluqingmathematics@gmail.com}}}
\maketitle
\begin{exercise}
  试证明双曲抛物面$\frac{x^2}{a^2}-\frac{y^2}{b^2}=2z(a\neq b)$上的两
  直母线直交时,其交点必在一双曲线上.
\end{exercise}
\begin{proof}[\textbf{解}]
双曲抛物面的$u$族直母线为
$$
\begin{cases}
  \frac{x}{a}+\frac{y}{b}=uz,\\
u(\frac{x}{a}-\frac{y}{b})=2
\end{cases}
$$
$v$族直母线为
$$
\begin{cases}
v(\frac{x}{a}+\frac{y}{b})=2,\\
\frac{x}{a}-\frac{y}{b}=vz.
\end{cases}
$$
其中$u,v$为参数.由于双曲抛物面的同族直母线必异面,异族直母线必相交,因此只可能是异族直母
线直交.于是,存在$u_0$和$v_0$,使得
$$
\left[\left(\frac{1}{a},\frac{1}{b},-u_0\right)\times
  \left(\frac{u_0}{a},\frac{-u_0}{b},0\right)\right]\cdot \left[\left(\frac{v_0}{a},\frac{v_{0}}{b},0\right)\times
  \left(\frac{1}{a},\frac{-1}{b},-v_0\right)\right]=0.
$$
根据Lagrange恒等式,即
$$
\begin{vmatrix}
  \frac{v_0}{a^2}+\frac{v_0}{b^2}&\frac{1}{a^{2}}-\frac{1}{b^2}+u_0v_0\\
\frac{u_0v_0}{a^2}-\frac{u_0v_0}{b^2}&\frac{u_0}{a^2}+\frac{u_0}{b^2}
\end{vmatrix}=0,
$$
可得
$$
\left(\frac{1}{a^2}+\frac{1}{b^2}\right)^2-\left(\frac{1}{a^2}-\frac{1}{b^2}\right)^2=\left(\frac{1}{a^2}-\frac{1}{b^2}\right)u_0v_0.
$$
异族直线的交点$(x_0,y_0,z_0)$满足
$$
\begin{cases}
  \frac{x_{0}}{a}+\frac{y_{0}}{b}=u_{0}z_0,\\
u_{0}(\frac{x_{0}}{a}-\frac{y_{0}}{b})=2,\\
v_{0}(\frac{x_{0}}{a}+\frac{y_{0}}{b})=2,\\
\frac{x_{0}}{a}-\frac{y_{0}}{b}=v_{0}z_{0}.
\end{cases}
$$
可得$z_0=\frac{2}{u_0v_0}$是一个非零常数,且
$$
\frac{x_0^2}{a^2}-\frac{y_0^2}{b^2}=2z_0
$$
可见,交点在双曲线上.
\end{proof}
\end{document}
