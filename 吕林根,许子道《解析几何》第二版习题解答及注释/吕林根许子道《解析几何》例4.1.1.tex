\documentclass[a4paper]{article}
\usepackage{amsmath,amsfonts,amsthm,amssymb} \usepackage{bm}
\usepackage{draftwatermark,euler}
\SetWatermarkText{http://blog.sciencenet.cn/u/Yaleking}%设置水印文字
\SetWatermarkLightness{0.8}%设置水印亮度
\SetWatermarkScale{0.35}%设置水印大小
\usepackage{hyperref} \usepackage{geometry} \usepackage{yhmath}
\usepackage{pstricks-add} \usepackage{framed,mdframed}
\usepackage{graphicx,color} \usepackage{mathrsfs,xcolor}
\usepackage[all]{xy} \usepackage{fancybox} \usepackage{xeCJK}
\newtheorem*{theo}{定理} 
\newtheorem*{exe}{题目}
\newenvironment{theorem}
{\bigskip\begin{mdframed}\begin{theo}}
    {\end{theo}\end{mdframed}\bigskip} 
\newenvironment{exercise}
{\bigskip\begin{mdframed}\begin{exe}}
    {\end{exe}\end{mdframed}\bigskip}
\geometry{left=2.5cm,right=2.5cm,top=2.5cm,bottom=2.5cm}
\setCJKmainfont[BoldFont=SimHei]{SimSun}
\numberwithin{equation}{section}
\setlength\parindent{0pt}
\newcommand{\D}{\displaystyle}\newcommand{\ri}{\Rightarrow}
\newcommand{\ds}{\displaystyle} \renewcommand{\ni}{\noindent}
\newcommand{\pa}{\partial} \newcommand{\Om}{\Omega}
\newcommand{\om}{\omega} \newcommand{\sik}{\sum_{i=1}^k}
\newcommand{\vov}{\Vert\omega\Vert} \newcommand{\Umy}{U_{\mu_i,y^i}}
\newcommand{\lamns}{\lambda_n^{^{\scriptstyle\sigma}}}
\newcommand{\chiomn}{\chi_{_{\Omega_n}}}
\newcommand{\ullim}{\underline{\lim}}
\newcommand{\mvb}{\mathversion{bold}} \newcommand{\la}{\lambda}
\newcommand{\La}{\Lambda} \newcommand{\va}{\varepsilon}
\newcommand{\be}{\beta}
\newcommand{\dis}{\displaystyle} \newcommand{\R}{{\mathbb R}}
\renewcommand{\today}{\number\year 年 \number\month 月 \number\day 日}
\newcommand{\N}{{\mathbb N}} \newcommand{\cF}{{\mathcal F}}
\newcommand{\gB}{{\mathfrak B}} \newcommand{\eps}{\epsilon}
\renewcommand\refname{参考文献}\renewcommand\figurename{图}
\usepackage[]{caption2} \renewcommand{\captionlabeldelim}{}
\begin{document}
\title{{\bf{吕林根,许子道《解析几何》例4.1.1\footnote{本解答作为交给解
        析几何赵老师的第四份作业.}}}} \author{\small{叶卢庆
    \footnote{叶卢庆(1992-),男,杭州师范大学理学院数学与应用数学专业大
      四.学号:1002011005.E-mail:yeluqingmathematics@gmail.com}}}
\maketitle
\begin{exercise}[例4.1.1]
柱面的准线方程是
$$
\begin{cases}
  x^2+y^2+z^2=1,\\
2x^2+2y^2+z^2=2,
\end{cases}
$$
而母线的方向数为$-1,0,1$,求这柱面的方程.
\end{exercise}
\begin{proof}[\textbf{解}]
设$(x_0,y_0,z_0)$是柱面上的任意一点.则通过这一点的母线的方程为
$$
\frac{x-x_0}{-1}=\frac{y-y_0}{0}=\frac{z-z_0}{1}.
$$
该母线通过准线,这意味着存在$(x_1,y_1,z_1)$,满足
$$
\begin{cases}
  x_{1}^2+y_{1}^2+z_{1}^2=1,\\
2x_{1}^2+2y_{1}^2+z_{1}^2=2,\\
\frac{x_{1}-x_0}{-1}=\frac{y_{1}-y_0}{0}=\frac{z_{1}-z_0}{1}
\end{cases}.
$$
也即,
$$
\begin{cases}
  x_{1}^2+y_{0}^2+z_{1}^2=1,\\
2x_{1}^2+2y_{0}^2+z_{1}^2=2,\\
x_0-x_1=z_1-z_0,\\
\end{cases}.
$$
于是,
$$
(x_0+z_{0})^2+y_{0}^2=1,
$$
于是柱面方程为
$$
(x+z)^2+y^2=1.
$$
\end{proof}
\end{document}