\documentclass[a4paper]{article}
\usepackage{amsmath,amsfonts,amsthm,amssymb}
\usepackage{bm}
\usepackage{hyperref}
\usepackage{geometry}
\usepackage{yhmath}
\usepackage{pstricks-add}
\usepackage{framed,mdframed}
\usepackage{graphicx,color} 
\usepackage{mathrsfs,xcolor} 
\usepackage[all]{xy}
\usepackage{fancybox} 
\usepackage{xeCJK}
\newtheorem*{theo}{定理}
\newtheorem*{exe}{题目}
\newtheorem*{rem}{评论}
\newtheorem*{lemma}{引理}
\newtheorem*{coro}{推论}
\newtheorem*{exa}{例}
\newenvironment{corollary}
{\bigskip\begin{mdframed}\begin{coro}}
    {\end{coro}\end{mdframed}\bigskip}
\newenvironment{theorem}
{\bigskip\begin{mdframed}\begin{theo}}
    {\end{theo}\end{mdframed}\bigskip}
\newenvironment{exercise}
{\bigskip\begin{mdframed}\begin{exe}}
    {\end{exe}\end{mdframed}\bigskip}
\newenvironment{example}
{\bigskip\begin{mdframed}\begin{exa}}
    {\end{exa}\end{mdframed}\bigskip}
\newenvironment{remark}
{\bigskip\begin{mdframed}\begin{rem}}
    {\end{rem}\end{mdframed}\bigskip}
\geometry{left=2.5cm,right=2.5cm,top=2.5cm,bottom=2.5cm}
\setCJKmainfont[BoldFont=SimHei]{SimSun}
\renewcommand{\today}{\number\year 年 \number\month 月 \number\day 日}
\newcommand{\D}{\displaystyle}\newcommand{\ri}{\Rightarrow}
\newcommand{\ds}{\displaystyle} \renewcommand{\ni}{\noindent}
\newcommand{\ov}{\overrightarrow}
\newcommand{\pa}{\partial} \newcommand{\Om}{\Omega}
\newcommand{\om}{\omega} \newcommand{\sik}{\sum_{i=1}^k}
\newcommand{\vov}{\Vert\omega\Vert} \newcommand{\Umy}{U_{\mu_i,y^i}}
\newcommand{\lamns}{\lambda_n^{^{\scriptstyle\sigma}}}
\newcommand{\chiomn}{\chi_{_{\Omega_n}}}
\newcommand{\ullim}{\underline{\lim}} \newcommand{\bsy}{\boldsymbol}
\newcommand{\mvb}{\mathversion{bold}} \newcommand{\la}{\lambda}
\newcommand{\La}{\Lambda} \newcommand{\va}{\varepsilon}
\newcommand{\be}{\beta} \newcommand{\al}{\alpha}
\newcommand{\dis}{\displaystyle} \newcommand{\R}{{\mathbb R}}
\newcommand{\N}{{\mathbb N}} \newcommand{\cF}{{\mathcal F}}
\newcommand{\gB}{{\mathfrak B}} \newcommand{\eps}{\epsilon}
\renewcommand\refname{参考文献}\renewcommand\figurename{图}
\usepackage[]{caption2} 
\renewcommand{\captionlabeldelim}{}
\setlength\parindent{0pt}
\begin{document}
\title{\huge{\bf{习题1.10.6}}} \author{\small{叶卢庆\footnote{叶卢庆(1992---),男,杭州师范大学理学院数学与应用数学专业本科在读,E-mail:yeluqingmathematics@gmail.com}}}
\maketitle
\begin{exercise}
  对于任意$\mathbf{a},\mathbf{b},\mathbf{c},\mathbf{d}$,证明
$$
(\mathbf{bcd})\mathbf{a}-(\mathbf{cda})\mathbf{b}+(\mathbf{dab})\mathbf{c}-(\mathbf{abc})\mathbf{d}=\mathbf{0}.
$$
\end{exercise}
\begin{proof}[\textbf{证明}]
我们知道空间中任意四个向量
$\mathbf{a},\mathbf{b},\mathbf{c},\mathbf{d}$都是线性相关的,这个题目给
出了系数的具体构造.设
$$
\lambda_1\mathbf{a}+\lambda_2\mathbf{b}+\lambda_3\mathbf{c}+\lambda_4\mathbf{d}=\mathbf{0}.
$$
则
$$
\lambda_1\mathbf{a}\cdot (\mathbf{a}\times
\mathbf{b})+\lambda_2\mathbf{b}\cdot (\mathbf{a}\times
\mathbf{b})+\lambda_3\mathbf{c}\cdot (\mathbf{a}\times
\mathbf{b})+\lambda_4\mathbf{d}\cdot (\mathbf{a}\times \mathbf{b})=\mathbf{0}.
$$
即
$$
\lambda_3\mathbf{c}\cdot (\mathbf{a}\times
\mathbf{b})+\lambda_4\mathbf{d}\cdot (\mathbf{a}\times \mathbf{b})=\mathbf{0}.
$$
这样,我们就得到
$$
\lambda_3:\lambda_4=\mathbf{abd}:-\mathbf{abc}.
$$
类似地我们可以得到$\lambda_1$和$\lambda_2$的比值,以及$\lambda_2$和
$\lambda_3$的比值.这样我们就得到了题目中的等式.
\end{proof}
\end{document}
