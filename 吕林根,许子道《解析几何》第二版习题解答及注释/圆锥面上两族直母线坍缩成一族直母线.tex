\documentclass[a4paper]{article}
\usepackage{amsmath,amsfonts,amsthm,amssymb}
\usepackage{bm}
\usepackage{draftwatermark,euler}
\SetWatermarkText{http://blog.sciencenet.cn/u/Yaleking}%设置水印文字
\SetWatermarkLightness{0.8}%设置水印亮度
\SetWatermarkScale{0.35}%设置水印大小
\usepackage{hyperref}
\usepackage{geometry}
\usepackage{yhmath}
\usepackage{pstricks-add}
\usepackage{framed,mdframed}
\usepackage{graphicx,color} 
\usepackage{mathrsfs,xcolor} 
\usepackage[all]{xy}
\usepackage{fancybox} 
\usepackage{xeCJK}
\newtheorem*{theo}{定理}
\newtheorem*{exe}{题目}
\newtheorem*{rem}{评论}
\newmdtheoremenv{lemma}{引理}
\newmdtheoremenv{corollary}{推论}
\newmdtheoremenv{example}{例}
\newenvironment{theorem}
{\bigskip\begin{mdframed}\begin{theo}}
    {\end{theo}\end{mdframed}\bigskip}
\newenvironment{exercise}
{\bigskip\begin{mdframed}\begin{exe}}
    {\end{exe}\end{mdframed}\bigskip}
\geometry{left=2.5cm,right=2.5cm,top=2.5cm,bottom=2.5cm}
\setCJKmainfont[BoldFont=SimHei]{SimSun}
\renewcommand{\today}{\number\year 年 \number\month 月 \number\day 日}
\newcommand{\D}{\displaystyle}\newcommand{\ri}{\Rightarrow}
\newcommand{\ds}{\displaystyle} \renewcommand{\ni}{\noindent}
\newcommand{\ov}{\overrightarrow}
\newcommand{\pa}{\partial} \newcommand{\Om}{\Omega}
\newcommand{\om}{\omega} \newcommand{\sik}{\sum_{i=1}^k}
\newcommand{\vov}{\Vert\omega\Vert} \newcommand{\Umy}{U_{\mu_i,y^i}}
\newcommand{\lamns}{\lambda_n^{^{\scriptstyle\sigma}}}
\newcommand{\chiomn}{\chi_{_{\Omega_n}}}
\newcommand{\ullim}{\underline{\lim}} \newcommand{\bsy}{\boldsymbol}
\newcommand{\mvb}{\mathversion{bold}} \newcommand{\la}{\lambda}
\newcommand{\La}{\Lambda} \newcommand{\va}{\varepsilon}
\newcommand{\be}{\beta} \newcommand{\al}{\alpha}
\newcommand{\dis}{\displaystyle} \newcommand{\R}{{\mathbb R}}
\newcommand{\N}{{\mathbb N}} \newcommand{\cF}{{\mathcal F}}
\newcommand{\gB}{{\mathfrak B}} \newcommand{\eps}{\epsilon}
\renewcommand\refname{参考文献}\renewcommand\figurename{图}
\usepackage[]{caption2} 
\renewcommand{\captionlabeldelim}{}
\setlength\parindent{0pt}
\begin{document}
\title{\huge{\bf{圆锥面上两族直母线坍缩成一族直母线}}} \author{\small{叶卢庆\footnote{叶卢庆(1992---),男,杭州师范大学理学院数学与应用数学专业本科在读,E-mail:yeluqingmathematics@gmail.com}}}
\maketitle
我们来看锥面
\begin{equation}
  \label{eq:1}
  \frac{x^2}{a^2}+\frac{y^2}{b^2}-\frac{z^2}{c^2}=0.
\end{equation}
该锥面可以看成单叶双曲面
\begin{equation}
  \label{eq:2}
  \frac{x^2}{a^2}+\frac{y^2}{b^2}-\frac{z^2}{c^2}=\frac{1}{\lambda^2}
\end{equation}
当$\lambda\to\infty$的极限情形.我们知道,单叶双曲面\eqref{eq:2}有两族直
母线
$$
\begin{cases}
  \gamma(\frac{x}{a}+\frac{z}{c})= \mu(\frac{1}{\lambda}+\frac{y}{b}),\\
\mu\frac{x}{a}-\frac{z}{c}=\gamma (\frac{1}{\lambda}-\frac{y}{b})
\end{cases},
\begin{cases}
  \alpha(\frac{x}{a}+\frac{z}{c})=\gamma(\frac{1}{\lambda}-\frac{y}{b}),\\
\gamma(\frac{x}{a}-\frac{z}{c})=\alpha (\frac{1}{\lambda}+\frac{y}{b})  
\end{cases}
$$
当$\frac{1}{\lambda}$没消失的时候,两族直母线还是两族直母线.而当
$\frac{1}{\lambda}$消失的时候,也就是单叶双曲面\eqref{eq:2}退化为锥面
\eqref{eq:1}的时候,两族直母线就坍缩为一族了.这是有点意思的现象.
\end{document}
