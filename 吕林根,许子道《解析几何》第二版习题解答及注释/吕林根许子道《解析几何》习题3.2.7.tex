\documentclass[a4paper]{article}
\usepackage{amsmath,amsfonts,amsthm,amssymb}
\usepackage{bm}
\usepackage{draftwatermark,euler}
\SetWatermarkText{http://blog.sciencenet.cn/u/Yaleking}%设置水印文字
\SetWatermarkLightness{0.8}%设置水印亮度
\SetWatermarkScale{0.35}%设置水印大小
\usepackage{hyperref}
\usepackage{geometry}
\usepackage{yhmath}
\usepackage{pstricks-add}
\usepackage{framed,mdframed}
\usepackage{graphicx,color} 
\usepackage{mathrsfs,xcolor} 
\usepackage[all]{xy}
\usepackage{fancybox} 
\usepackage{xeCJK}
\newtheorem{theo}{定理}
\newtheorem*{exe}{题目}
\newtheorem*{rem}{评论}
\newmdtheoremenv{lemma}{引理}
\newmdtheoremenv{corollary}{推论}
\newmdtheoremenv{example}{例}
\newenvironment{theorem}
{\bigskip\begin{mdframed}\begin{theo}}
    {\end{theo}\end{mdframed}\bigskip}
\newenvironment{exercise}
{\bigskip\begin{mdframed}\begin{exe}}
    {\end{exe}\end{mdframed}\bigskip}
\geometry{left=2.5cm,right=2.5cm,top=2.5cm,bottom=2.5cm}
\setCJKmainfont[BoldFont=SimHei]{SimSun}
\renewcommand{\today}{\number\year 年 \number\month 月 \number\day 日}
\newcommand{\D}{\displaystyle}\newcommand{\ri}{\Rightarrow}
\newcommand{\ds}{\displaystyle} \renewcommand{\ni}{\noindent}
\newcommand{\pa}{\partial} \newcommand{\Om}{\Omega}
\newcommand{\ov}{\overrightarrow} \newcommand{\sik}{\sum_{i=1}^k}
\newcommand{\vov}{\Vert\omega\Vert} \newcommand{\Umy}{U_{\mu_i,y^i}}
\newcommand{\lamns}{\lambda_n^{^{\scriptstyle\sigma}}}
\newcommand{\chiomn}{\chi_{_{\Omega_n}}}
\newcommand{\ullim}{\underline{\lim}} \newcommand{\bsy}{\boldsymbol}
\newcommand{\mvb}{\mathversion{bold}} \newcommand{\la}{\lambda}
\newcommand{\La}{\Lambda} \newcommand{\va}{\varepsilon}
\newcommand{\be}{\beta} \newcommand{\al}{\alpha}
\newcommand{\dis}{\displaystyle} \newcommand{\R}{{\mathbb R}}
\newcommand{\N}{{\mathbb N}} \newcommand{\cF}{{\mathcal F}}
\newcommand{\gB}{{\mathfrak B}} \newcommand{\eps}{\epsilon}
\renewcommand\refname{参考文献}\renewcommand\figurename{图}
\usepackage[]{caption2} 
\renewcommand{\captionlabeldelim}{}
\begin{document}
\title{\huge{\bf{吕林根,许子道《解析几何》习题3.2.7}}} \author{\small{叶卢
    庆\footnote{叶卢庆(1992---),男,杭州师范大学理学院数学与应用数学专业
      本科在读,E-mail:yeluqingmathematics@gmail.com}}}
\maketitle
\begin{exercise}
  设平面$\pi$为$Ax+By+Cz+D=0$,它与联结两点$M_1(x_1,y_1,z_1)$和
  $M_2(x_2,y_2,z_2)$的直线相交于点$M$,且$\ov{M_1M}=\lambda \ov{MM_2}$,
  求证
$$
\lambda=-\frac{Ax_1+By_1+Cz_1+D}{Ax_2+By_2+Cz_2+D}.
$$
\end{exercise}
\begin{proof}[\textbf{证明}]
$\lambda$之值,即为$M_1$到平面的距离与$M_2$到平面的距离之比,即位$M_1$与
平面的离差与$M_2$与平面的离差之比的相反数.我们先把平面的一般方程化为法
式方程,为
$$
\pm\frac{A}{\sqrt{A^2+B^2+C^2}}x\pm\frac{B}{\sqrt{A^2+B^2+C^2}}\pm\frac{C}{\sqrt{A^2+B^2+C^2}}\pm\frac{D}{\sqrt{A^2+B^2+C^2}}=0.
$$
于是点$M_1$和平面的离差为
$$
\pm\frac{A}{\sqrt{A^2+B^2+C^2}}x_1\pm
\frac{B}{\sqrt{A^2+B^2+C^2}}y_{1}\pm \frac{C}{\sqrt{A^2+B^2+C^2}}z_{1}\pm \frac{D}{\sqrt{A^2+B^2+C^2}},
$$
同理点$M_2$和平面的离差为
$$
\pm\frac{A}{\sqrt{A^2+B^2+C^2}}x_2\pm
\frac{B}{\sqrt{A^2+B^2+C^2}}y_2\pm \frac{C}{\sqrt{A^2+B^2+C^2}}z_2\pm \frac{D}{\sqrt{A^2+B^2+C^2}},
$$
两个离差相除,并取相反数,可得
$$
\lambda=-\frac{Ax_1+By_1+Cz_1+D}{Ax_2+By_2+Cz_2+D}.
$$
\end{proof}
\end{document}
