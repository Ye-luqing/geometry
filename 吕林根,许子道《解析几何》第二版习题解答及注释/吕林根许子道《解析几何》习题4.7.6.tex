\documentclass[a4paper]{article}
\usepackage{amsmath,amsfonts,amsthm,amssymb}
\usepackage{bm}
\usepackage{draftwatermark,euler}
\SetWatermarkText{http://blog.sciencenet.cn/u/Yaleking}%设置水印文字
\SetWatermarkLightness{0.8}%设置水印亮度
\SetWatermarkScale{0.35}%设置水印大小
\usepackage{hyperref}
\usepackage{geometry}
\usepackage{yhmath}
\usepackage{pstricks-add}
\usepackage{framed,mdframed}
\usepackage{graphicx,color} 
\usepackage{mathrsfs,xcolor} 
\usepackage[all]{xy}
\usepackage{fancybox} 
\usepackage{xeCJK}
\newtheorem*{theo}{定理}
\newtheorem*{exe}{题目}
\newtheorem*{rem}{评论}
\newmdtheoremenv{lemma}{引理}
\newmdtheoremenv{corollary}{推论}
\newmdtheoremenv{example}{例}
\newenvironment{theorem}
{\bigskip\begin{mdframed}\begin{theo}}
    {\end{theo}\end{mdframed}\bigskip}
\newenvironment{exercise}
{\bigskip\begin{mdframed}\begin{exe}}
    {\end{exe}\end{mdframed}\bigskip}
\geometry{left=2.5cm,right=2.5cm,top=2.5cm,bottom=2.5cm}
\setCJKmainfont[BoldFont=SimHei]{SimSun}
\renewcommand{\today}{\number\year 年 \number\month 月 \number\day 日}
\newcommand{\D}{\displaystyle}\newcommand{\ri}{\Rightarrow}
\newcommand{\ds}{\displaystyle} \renewcommand{\ni}{\noindent}
\newcommand{\ov}{\overrightarrow}
\newcommand{\pa}{\partial} \newcommand{\Om}{\Omega}
\newcommand{\om}{\omega} \newcommand{\sik}{\sum_{i=1}^k}
\newcommand{\vov}{\Vert\omega\Vert} \newcommand{\Umy}{U_{\mu_i,y^i}}
\newcommand{\lamns}{\lambda_n^{^{\scriptstyle\sigma}}}
\newcommand{\chiomn}{\chi_{_{\Omega_n}}}
\newcommand{\ullim}{\underline{\lim}} \newcommand{\bsy}{\boldsymbol}
\newcommand{\mvb}{\mathversion{bold}} \newcommand{\la}{\lambda}
\newcommand{\La}{\Lambda} \newcommand{\va}{\varepsilon}
\newcommand{\be}{\beta} \newcommand{\al}{\alpha}
\newcommand{\dis}{\displaystyle} \newcommand{\R}{{\mathbb R}}
\newcommand{\N}{{\mathbb N}} \newcommand{\cF}{{\mathcal F}}
\newcommand{\gB}{{\mathfrak B}} \newcommand{\eps}{\epsilon}
\renewcommand\refname{参考文献}\renewcommand\figurename{图}
\usepackage[]{caption2} 
\renewcommand{\captionlabeldelim}{}
\begin{document}
\title{\huge{\bf{吕林根,许子道《解析几何》习题4.7.6}}} \author{\small{叶卢庆\footnote{叶卢庆(1992---),男,杭州师范大学理学院数学与应用数学专业本科在读,E-mail:yeluqingmathematics@gmail.com}}}
\maketitle
\begin{exercise}
  求与下列三条直线
$$
\begin{cases}
  x=1,\\
y=z,
\end{cases}
\begin{cases}
  x=-1,\\
y=-z
\end{cases}
$$
与
$$
\frac{x-2}{-3}=\frac{y+1}{4}=\frac{z+2}{5}.
$$
都共面的直线所构成的曲面.
\end{exercise}
\begin{proof}[\textbf{解}]
 设直线的方程为
 \begin{equation}
   \label{eq:1}
   \frac{x-p_1}{\lambda_1}=\frac{y-p_2}{\lambda_2}=\frac{z-p_3}{\lambda_3}.
 \end{equation}
直线\eqref{eq:1}与直线$
\begin{cases}
  x=1,\\
y=z
\end{cases}
$共面,表明
\begin{equation}\label{eq:2}
\begin{vmatrix}
  \lambda_1&\lambda_2&\lambda_3\\
0&1&1\\
p_1-1&p_2&p_3
\end{vmatrix}=0,
\end{equation}
直线\eqref{eq:1}与直线$
\begin{cases}
  x=-1,\\
y=-z
\end{cases}
$共面,表明
\begin{equation}
  \label{eq:3}
  \begin{vmatrix}
    \lambda_1&\lambda_2&\lambda_3\\
0&1&-1\\
p_1+1&p_2&p_3
  \end{vmatrix}=0,
\end{equation}
直线\eqref{eq:1}与直线$\frac{x-2}{-3}=\frac{y+1}{4}=\frac{z+2}{5}$共面,表
明
\begin{equation}
  \label{eq:4}
  \begin{vmatrix}
    \lambda_1&\lambda_2&\lambda_3\\
-3&4&5\\
p_1-2&p_2+1&p_3+2
  \end{vmatrix}=0.
\end{equation}
方程\eqref{eq:2},\eqref{eq:3},\eqref{eq:4}联立,可知
$$
\begin{cases}
  (p_3-p_2)\lambda_1+(p_1-1)\lambda_2-(p_1-1)\lambda_3=0,\\
(p_3+p_2)\lambda_1-(p_1+1)\lambda_2-(p_1+1)\lambda_3=0,\\
(4p_3-5p_2+3)\lambda_1+(5p_1+3p_3-4)\lambda_2+(-3p_2-4p_1+5)\lambda_3=0
\end{cases}
$$
该关于$\lambda_1,\lambda_2,\lambda_3$的方程组显然不止有一个解
$(0,0,0)$,因此行列式
$$
\begin{vmatrix}
p_3-p_2&p_1-1&-(p_1-1)\\
p_3+p_2&-(p_1+1)&-(p_1+1)\\
4p_3-5p_2+3&5p_1+3p_3-4&-3p_2-4p_1+5
\end{vmatrix}=0.
$$
将第二行加到第一行上,可得
$$
\begin{vmatrix}
2p_3&-2&-2p_1\\
p_3+p_2&-(p_1+1)&-(p_1+1)\\
4p_3-5p_2+3&5p_1+3p_3-4&-3p_2-4p_1+5  
\end{vmatrix}=0.
$$
将第一行乘以$-2$加到第三行上,可得
$$
\begin{vmatrix}
2p_3&-2&-2p_1\\
p_3+p_2&-(p_1+1)&-(p_1+1)\\
-5p_2+3&5p_1+3p_3&-3p_2+5  
\end{vmatrix}=0.
$$
即
$$
\begin{vmatrix}
p_3&-1&-p_1\\
p_3+p_2&-(p_1+1)&-(p_1+1)\\
-5p_2+3&5p_1+3p_3&-3p_2+5  
\end{vmatrix}=0.
$$
再将第一行乘以$-1$加到第二行上,可得
$$
\begin{vmatrix}
p_3&-1&-p_1\\
p_2&-p_1&-1\\
-5p_2+3&5p_1+3p_3&-3p_2+5  
\end{vmatrix}=0.
$$
再将第二行乘以$5$加到第三行上,可得
$$
\begin{vmatrix}
p_3&-1&-p_1\\
p_2&-p_1&-1\\
3&3p_3&-3p_2
\end{vmatrix}=0.
$$
即
$$
p_1^2+p_2^2-p_3^2=1.
$$
可见,曲面为
$$
x^2+y^2-z^2=1.
$$
\end{proof}
\end{document}
