\documentclass[a4paper]{article}
\usepackage{amsmath,amsfonts,amsthm,amssymb}
\usepackage{bm}
\usepackage{euler}
\usepackage{hyperref}
\usepackage{geometry}
\usepackage{yhmath}
\usepackage{pstricks-add}
\usepackage{framed,mdframed}
\usepackage{graphicx,color} 
\usepackage{mathrsfs,xcolor} 
\usepackage[all]{xy}
\usepackage{fancybox} 
\usepackage{xeCJK}
\newtheorem*{theo}{定理}
\newtheorem*{exe}{题目}
\newtheorem*{rem}{评论}
\newmdtheoremenv{lemma}{引理}
\newmdtheoremenv{corollary}{推论}
\newmdtheoremenv{example}{例}
\newenvironment{theorem}
{\bigskip\begin{mdframed}\begin{theo}}
    {\end{theo}\end{mdframed}\bigskip}
\newenvironment{exercise}
{\bigskip\begin{mdframed}\begin{exe}}
    {\end{exe}\end{mdframed}\bigskip}
\newenvironment{remark}
{\bigskip\begin{mdframed}\begin{rem}}
    {\end{rem}\end{mdframed}\bigskip}
\geometry{left=2.5cm,right=2.5cm,top=2.5cm,bottom=2.5cm}
\setCJKmainfont[BoldFont=FZQingKeBenYueSongS-R-GB]{FZSongKeBenXiuKaiS-R-GB}
\renewcommand{\today}{\number\year 年 \number\month 月 \number\day 日}
\newcommand{\D}{\displaystyle}\newcommand{\ri}{\Rightarrow}
\newcommand{\ds}{\displaystyle} \renewcommand{\ni}{\noindent}
\newcommand{\ov}{\overrightarrow}
\newcommand{\pa}{\partial} \newcommand{\Om}{\Omega}
\newcommand{\om}{\omega} \newcommand{\sik}{\sum_{i=1}^k}
\newcommand{\vov}{\Vert\omega\Vert} \newcommand{\Umy}{U_{\mu_i,y^i}}
\newcommand{\lamns}{\lambda_n^{^{\scriptstyle\sigma}}}
\newcommand{\chiomn}{\chi_{_{\Omega_n}}}
\newcommand{\ullim}{\underline{\lim}} \newcommand{\bsy}{\boldsymbol}
\newcommand{\mvb}{\mathversion{bold}} \newcommand{\la}{\lambda}
\newcommand{\La}{\Lambda} \newcommand{\va}{\varepsilon}
\newcommand{\be}{\beta} \newcommand{\al}{\alpha}
\newcommand{\dis}{\displaystyle} \newcommand{\R}{{\mathbb R}}
\newcommand{\N}{{\mathbb N}} \newcommand{\cF}{{\mathcal F}}
\newcommand{\gB}{{\mathfrak B}} \newcommand{\eps}{\epsilon}
\renewcommand\refname{参考文献}\renewcommand\figurename{图}
\usepackage[]{caption2} 
\renewcommand{\captionlabeldelim}{}
\begin{document}
\title{\huge{\bf{吕林根,许子道《解析几何》习题4.4.5}}} \author{\small{叶卢庆\footnote{叶卢庆(1992---),男,杭州师范大学理学院数学与应用数学专业本科在读,E-mail:yeluqingmathematics@gmail.com}}}
\maketitle
\begin{exercise}
一直线分别交坐标面$yOz$,$zOx$,$xOy$于三点$A,B,C$.当直线变动时,直线上的
三定点$A,B,C$也分别在三个坐标面上变动,另外直线上有第四个点$P$,它与
$A,B,C$三点的距离分别为$a,b,c$,当直线按照这样的规定变动时,试求$P$点的
轨迹.  
\end{exercise}
\begin{proof}[\textbf{解}]
设直线的方程为
$$
\frac{x-q_1}{t_1}=\frac{y-q_2}{t_2}=\frac{z-q_3}{t_3},
$$
其中 $q_1,q_2,q_3,t_1,t_2,t_3$都是参数,且$t_1,t_2,t_3$为直线的一个方向
余弦.令$x=0$,可得
$$
A:(0,q_2-\frac{q_1}{t_1}t_2,q_3-\frac{q_1}{t_1}t_3).
$$
令$y=0$,可得
$$
B:(q_1-\frac{q_2}{t_2}t_1,0,q_3-\frac{q_2}{t_2}t_3).
$$
令$z=0$,可得
$$
C:(q_1-\frac{q_3}{t_3}t_1,q_2-\frac{q_3}{t_3}t_2,0).
$$
%$A,B,C$是直线上的定点,当且仅当存在常数$\lambda$,使得
%\begin{equation}\label{eq:1}
%\ov{AB}=\lambda \ov{BC},
%\end{equation}
%且 $|\ov{AB}|^{2}$是一个常数$l$.由方程\eqref{eq:1}可得
%\begin{equation}
%  \label{eq:2}
%  \lambda=\frac{q_1-\frac{q_2}{t_2}t_1}{\frac{q_2}{t_2}t_1-\frac{q_3}{t_3}t_1}=\frac{\frac{q_1}{t_1}-\frac{q_2}{t_2}}{\frac{q_2}%{t_2}-\frac{q_3}{t_3}}.
%\end{equation}
%由$|\ov{AB}|^{2}$是一个常数$l$,最终可得,
%\begin{equation}
%  \label{eq:3}
%  (t_{1}^{2}+t_{2}^{2}+t_{3}^{2})(\frac{q_1}{t_{1}}-\frac{q_2}{t_2})^2=l^2.\Rightarrow
%  (\frac{q_1}{t_{1}}-\frac{q_2}{t_2})^{2}=l^{2}.
%\end{equation}
现在,设$P$的坐标为$(x_{p},y_{p},z_p)$.则由题目条件可得
$$
\begin{cases}
  (\frac{x_p}{t_1},\frac{y_p}{t_2},\frac{z_p}{t_3})=(0,\frac{q_2}{t_{2}}-\frac{q_1}{t_1},\frac{q_3}{t_{3}}-\frac{q_1}{t_1})\pm
  a(1,1,1),\\
(\frac{x_p}{t_1},\frac{y_p}{t_2},\frac{z_p}{t_3})=(\frac{q_1}{t_{1}}-\frac{q_2}{t_2},0,\frac{q_3}{t_{3}}-\frac{q_2}{t_2})\pm
b(1,1,1),\\
(\frac{x_p}{t_1},\frac{y_p}{t_2},\frac{z_p}{t_3})=(\frac{q_1}{t_{1}}-\frac{q_3}{t_3},\frac{q_2}{t_{2}}-\frac{q_3}{t_3},0)\pm c(1,1,1).
\end{cases}
$$
可见,
$$
\frac{x_p}{t_1}=\pm a,\frac{y_p}{t_2}=\pm b,\frac{z_p}{t_3}=\pm c,
$$
可见,
$$
\frac{x_p^2}{a^2}+\frac{y_p^2}{b^2}+\frac{z_p^2}{c^2}=t_1^{2}+t_{2}^{2}+t_{3}^{2}=1.
$$
可见,点$P$的轨迹为$\frac{x^2}{a^2}+\frac{y^2}{b^2}+\frac{z^2}{c^2}=1$.
\end{proof}
\begin{remark}
此题中,“定点”这个条件是不必要的.  
\end{remark}
\end{document}
