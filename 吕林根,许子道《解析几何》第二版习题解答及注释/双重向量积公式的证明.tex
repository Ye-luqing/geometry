\documentclass[a4paper]{article}
\usepackage{amsmath,amsfonts,amsthm,amssymb}
\usepackage{bm}
\usepackage{hyperref}
\usepackage{geometry}
\usepackage{yhmath}
\usepackage{pstricks-add}
\usepackage{framed,mdframed}
\usepackage{graphicx,color} 
\usepackage{mathrsfs,xcolor} 
\usepackage[all]{xy}
\usepackage{fancybox} 
\usepackage{xeCJK}
\newtheorem{theo}{定理}
\newtheorem*{exe}{题目}
\newtheorem*{rem}{评论}
\setlength\parindent{0pt}
\newmdtheoremenv{lemma}{引理}
\newmdtheoremenv{corollary}{推论}
\newmdtheoremenv{example}{例}
\newenvironment{theorem}
{\bigskip\begin{mdframed}\begin{theo}}
    {\end{theo}\end{mdframed}\bigskip}
\newenvironment{exercise}
{\bigskip\begin{mdframed}\begin{exe}}
    {\end{exe}\end{mdframed}\bigskip}
\geometry{left=2.5cm,right=2.5cm,top=2.5cm,bottom=2.5cm}
\setCJKmainfont[BoldFont=SimHei]{SimSun}
\renewcommand{\today}{\number\year 年 \number\month 月 \number\day 日}
\newcommand{\D}{\displaystyle}\newcommand{\ri}{\Rightarrow}
\newcommand{\ds}{\displaystyle} \renewcommand{\ni}{\noindent}
\newcommand{\pa}{\partial} \newcommand{\Om}{\Omega}
\newcommand{\om}{\omega} \newcommand{\sik}{\sum_{i=1}^k}
\newcommand{\vov}{\Vert\omega\Vert} \newcommand{\Umy}{U_{\mu_i,y^i}}
\newcommand{\lamns}{\lambda_n^{^{\scriptstyle\sigma}}}
\newcommand{\chiomn}{\chi_{_{\Omega_n}}}
\newcommand{\ullim}{\underline{\lim}} \newcommand{\bsy}{\boldsymbol}
\newcommand{\mvb}{\mathversion{bold}} \newcommand{\la}{\lambda}
\newcommand{\La}{\Lambda} \newcommand{\va}{\varepsilon}
\newcommand{\be}{\beta} \newcommand{\al}{\alpha}
\newcommand{\dis}{\displaystyle} \newcommand{\R}{{\mathbb R}}
\newcommand{\N}{{\mathbb N}} \newcommand{\cF}{{\mathcal F}}
\newcommand{\gB}{{\mathfrak B}} \newcommand{\eps}{\epsilon}
\renewcommand\refname{参考文献}\renewcommand\figurename{图}
\usepackage[]{caption2} 
\renewcommand{\captionlabeldelim}{}
\begin{document}
\title{\huge{\bf{双重向量积公式}}} \author{\small{叶卢
    庆\footnote{叶卢庆(1992---),男,杭州师范大学理学院数学与应用数学专业
      本科在读,E-mail:yeluqingmathematics@gmail.com}}}
\maketitle
设$\mathbf{a,b,c}$是三维欧式空间$\mathbf{R}^3$中的三个向量,且
$\mathbf{a,b}$线性无关.我们来计算
\begin{equation}
  \label{eq:1}
  (\mathbf{a}\times \mathbf{b})\times \mathbf{c}.
\end{equation}
设向量$\mathbf{c}=\lambda \mathbf{a}+\mu \mathbf{b}+k\mathbf{d}$,其中向量
$\mathbf{d}$和向量$\mathbf{a},\mathbf{b}$都正交.于是式\eqref{eq:1}成为
\begin{equation}
  \label{eq:2}
(\mathbf{a}\times \mathbf{b})\times \mathbf{c}=(\mathbf{a}\times \mathbf{b})\times(\lambda
  \mathbf{a}+\mu \mathbf{b}+k\mathbf{d})=\lambda (\mathbf{a}\times
  \mathbf{b})\times \mathbf{a}+\mu (\mathbf{a}\times \mathbf{b})\times \mathbf{b}.
\end{equation}
无论是$(\mathbf{a}\times \mathbf{b})\times \mathbf{a}$,还是
$(\mathbf{a}\times \mathbf{b})\times \mathbf{b}$,都是在向量
$\mathbf{a},\mathbf{b}$张成的二维平面上.令
$$
(\mathbf{a}\times \mathbf{b})\times \mathbf{a}=\gamma
(\mathbf{b}-t\mathbf{a}),
$$
其中 $\gamma>0$,且$(\mathbf{b}-t\mathbf{a})\cdot \mathbf{a}=0$.于是
$t=\frac{\mathbf{b}\cdot \mathbf{a}}{\mathbf{a}\cdot \mathbf{a}}$.因此,
$$
(\mathbf{a}\times \mathbf{b})\times \mathbf{a}=\gamma(\mathbf{b}-
\frac{\mathbf{b}\cdot \mathbf{a}}{\mathbf{a}\cdot \mathbf{a}}\mathbf{a}).
$$
由于
$$
|(\mathbf{a}\times \mathbf{b})\times
\mathbf{a}|=(\mathbf{a}\cdot \mathbf{a})|\mathbf{b}|\sin\langle \mathbf{a},\mathbf{b}\rangle,
$$
因此,$\gamma=\mathbf{a}\cdot \mathbf{a}$.可见,
\begin{equation}\label{eq:3}
(\mathbf{a}\times \mathbf{b})\times \mathbf{a}=(\mathbf{a}\cdot
\mathbf{a})\mathbf{b}-(\mathbf{b}\cdot \mathbf{a})\mathbf{a}.
\end{equation}
且
\begin{equation}\label{eq:4}
(\mathbf{a}\times \mathbf{b})\times \mathbf{b}=-(\mathbf{b}\times
\mathbf{a})\times \mathbf{b}=(\mathbf{a}\cdot
\mathbf{b})\mathbf{b}-(\mathbf{b}\cdot \mathbf{b})\mathbf{a}.
\end{equation}
下面我们来确定$\lambda,\mu$.因为
$$
\begin{cases}
  (\mathbf{c}-\lambda \mathbf{a}-\mu \mathbf{b})\cdot \mathbf{a}=0,\\
  (\mathbf{c}-\lambda \mathbf{a}-\mu \mathbf{b})\cdot \mathbf{b}=0.
\end{cases}
$$
所以$\lambda$和$\mu$就能解出,
\begin{equation}\label{eq:5}
\mu=\frac{\mathbf{(a\cdot a)(\mathbf{c}\cdot
    \mathbf{b})-(\mathbf{a}\cdot \mathbf{b})(\mathbf{c}\cdot
    \mathbf{a})}}{\mathbf{(\mathbf{a}\cdot \mathbf{a})(\mathbf{b}\cdot
  \mathbf{b})-(\mathbf{a}\cdot \mathbf{b})(\mathbf{b}\cdot \mathbf{a})}},\lambda=\frac{\mathbf{(b\cdot b)(\mathbf{c}\cdot
    \mathbf{a})-(\mathbf{b}\cdot \mathbf{a})(\mathbf{c}\cdot
    \mathbf{b})}}{\mathbf{(\mathbf{b}\cdot \mathbf{b})(\mathbf{a}\cdot
  \mathbf{a})-(\mathbf{b}\cdot \mathbf{a})(\mathbf{a}\cdot \mathbf{b})}}.
\end{equation}
将式\eqref{eq:3},\eqref{eq:4},\eqref{eq:5}代入式\eqref{eq:2},化简可得
$$
(\mathbf{a}\times \mathbf{b})\times \mathbf{c}=(\mathbf{a}\cdot
\mathbf{c})\mathbf{b}-(\mathbf{b}\cdot \mathbf{c})\mathbf{a}.
$$
\end{document}
