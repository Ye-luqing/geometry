\documentclass[a4paper]{article}
\usepackage{amsmath,amsfonts,amsthm,amssymb}
\usepackage{bm}
\usepackage{draftwatermark,euler}
\SetWatermarkText{http://blog.sciencenet.cn/u/Yaleking}%设置水印文字
\SetWatermarkLightness{0.8}%设置水印亮度
\SetWatermarkScale{0.35}%设置水印大小
\usepackage{hyperref}
\usepackage{geometry}
\usepackage{yhmath}
\usepackage{pstricks-add}
\usepackage{framed,mdframed}
\usepackage{graphicx,color} 
\usepackage{mathrsfs,xcolor} 
\usepackage[all]{xy}
\usepackage{fancybox} 
\usepackage{xeCJK}
\newtheorem*{theo}{定理}
\newtheorem*{exe}{题目}
\newtheorem*{rem}{评论}
\newmdtheoremenv{lemma}{引理}
\newmdtheoremenv{corollary}{推论}
\newmdtheoremenv{example}{例}
\newenvironment{theorem}
{\bigskip\begin{mdframed}\begin{theo}}
    {\end{theo}\end{mdframed}\bigskip}
\newenvironment{exercise}
{\bigskip\begin{mdframed}\begin{exe}}
    {\end{exe}\end{mdframed}\bigskip}
\geometry{left=2.5cm,right=2.5cm,top=2.5cm,bottom=2.5cm}
\setCJKmainfont[BoldFont=SimHei]{SimSun}
\renewcommand{\today}{\number\year 年 \number\month 月 \number\day 日}
\newcommand{\D}{\displaystyle}\newcommand{\ri}{\Rightarrow}
\newcommand{\ds}{\displaystyle} \renewcommand{\ni}{\noindent}
\newcommand{\pa}{\partial} \newcommand{\Om}{\Omega}
\newcommand{\om}{\omega} \newcommand{\sik}{\sum_{i=1}^k}
\newcommand{\vov}{\Vert\omega\Vert} \newcommand{\Umy}{U_{\mu_i,y^i}}
\newcommand{\lamns}{\lambda_n^{^{\scriptstyle\sigma}}}
\newcommand{\chiomn}{\chi_{_{\Omega_n}}}
\newcommand{\ullim}{\underline{\lim}} \newcommand{\bsy}{\boldsymbol}
\newcommand{\mvb}{\mathversion{bold}} \newcommand{\la}{\lambda}
\newcommand{\La}{\Lambda} \newcommand{\va}{\varepsilon}
\newcommand{\be}{\beta} \newcommand{\al}{\alpha}
\newcommand{\dis}{\displaystyle} \newcommand{\R}{{\mathbb R}}
\newcommand{\N}{{\mathbb N}} \newcommand{\cF}{{\mathcal F}}
\newcommand{\gB}{{\mathfrak B}} \newcommand{\eps}{\epsilon}
\renewcommand\refname{参考文献}\renewcommand\figurename{图}
\usepackage[]{caption2} 
\renewcommand{\captionlabeldelim}{}
\begin{document}
\title{\huge{\bf{两直线共面的充要条件}}} \author{\small{叶卢
    庆\footnote{叶卢庆(1992---),男,杭州师范大学理学院数学与应用数学专业
      本科在读,E-mail:yeluqingmathematics@gmail.com}}}
\maketitle\ni
下面这道题目来自吕林根,许子道编《解析几何》例3.8.3.
\begin{theorem}
  试证两直线
$$
l_1:
\begin{cases}
  A_1x+B_1y+C_1z+D_1=0,\\
  A_2x+B_2y+C_2z+D_2=0.
\end{cases}
$$
与
$$l_2:
\begin{cases}
  A_3x+B_3y+C_3z+D_3=0,\\
  A_4x+B_4y+C_4z+D_4=0
\end{cases}
$$
在同一平面上的充要条件是
$$
\begin{vmatrix}
  A_1&B_1&C_1&D_1\\
  A_2&B_2&C_2&D_2\\
  A_3&B_3&C_3&D_3\\
  A_4&B_4&C_4&D_4\\
\end{vmatrix}=0.
$$
\end{theorem}
\begin{proof}[\textbf{证明}]
  行列式为$0$时,向
  量$(A_1,B_1,C_1,D_1),\cdots,(A_4,B_4,C_4,D_4)$线性相关.不失一般性
  地,不妨
  设$(A_1,B_1,C_1,D_1)$能被
  $(A_2,B_2,C_2,D_2),(A_3,B_3,C_3,D_3),(A_4,B_4,C_4,D_4)$线性表出为
$$
(A_1,B_1,C_1,D_1)=a(A_2,B_2,C_2,D_2)+b(A_3,B_3,C_3,D_3)+c(A_4,B_4,C_4,D_4).
$$
于是,
\begin{equation}
  \label{eq:1}
  (A_1,B_1,C_1,D_1)-a(A_2,B_2,C_2,D_2)=b(A_3,B_3,C_3,D_3)+c(A_4,B_4,C_4,D_4).
\end{equation}
由于平面$A_3x+B_3y+C_3z+D_3=0$和$A_4x+B_4y+C_4z+D_4=0$相交于直
线$l_2$,因此平面$b(A_3,B_3,C_3,D_3)+c(A_4,B_4,C_4,D_4)$也经过直
线$l_2$,于是,平面$(A_1,B_1,C_1,D_1)-a(A_2,B_2,C_2,D_2)$经过直线
$l_2$.但是由于平面$(A_1,B_1,C_1,D_1)-a(A_2,B_2,C_2,D_2)$还经过$l_1$,因
此$l_1$和$l_2$共面.\\\\

\ni 而当$l_1,l_2$共面时,必定有一个平面$s$同时经过$l_1,l_2$.且$s$能表示
成
$$
 p(A_1x+B_1y+C_1z+D_1)+q(A_2x+B_2y+C_2z+D_2),
$$
而且$s$能表示成
$$
m(A_3x+B_3y+C_3z+D_3)+n(A_4x+B_4y+C_4z+D_4),
$$
于是,
$$
 p(A_1x+B_1y+C_1z+D_1)+q(A_2x+B_2y+C_2z+D_2)=m(A_3x+B_3y+C_3z+D_3)+n(A_4x+B_4y+C_4z+D_4).
$$
这就表明了向量$(A_1,B_1,C_1,D_1),\cdots,(A_4,B_4,C_4,D_4)$线性相关,于
是行列式为$0$.
\end{proof}
\end{document}




















为了证明这个定理,我们只用证明,直线$l_1$和$l_2$共面的充要条件是,四维空
间$\mathbf{R}^4$中经过原点的四个三维超平面
\begin{equation}
  \label{eq:1}
  A_1x+B_1y+C_1z+D_1w=0,
\end{equation}
\begin{equation}
  \label{eq:2}
  A_2x+B_2y+C_2z+D_2w=0,
\end{equation}
\begin{equation}
  \label{eq:3}
  A_3x+B_3y+C_3z+D_3w=0,
\end{equation}
\begin{equation}
  \label{eq:4}
  A_4x+B_4y+C_4z+D_4w=0.
\end{equation}
的公共交点不唯一.