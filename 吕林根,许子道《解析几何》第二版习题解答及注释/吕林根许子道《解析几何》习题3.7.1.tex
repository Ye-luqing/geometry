\documentclass[a4paper]{article}
\usepackage{amsmath,amsfonts,amsthm,amssymb}
\usepackage{bm}
\usepackage{draftwatermark,euler}
\SetWatermarkText{http://blog.sciencenet.cn/u/Yaleking}%设置水印文字
\SetWatermarkLightness{0.8}%设置水印亮度
\SetWatermarkScale{0.35}%设置水印大小
\usepackage{hyperref}
\usepackage{geometry}
\usepackage{yhmath}
\usepackage{pstricks-add}
\usepackage{framed,mdframed}
\usepackage{graphicx,color} 
\usepackage{mathrsfs,xcolor} 
\usepackage[all]{xy}
\usepackage{fancybox} 
\usepackage{xeCJK}
\newtheorem{theo}{定理}
\newtheorem*{exe}{题目}
\newtheorem*{rem}{评论}
\newmdtheoremenv{lemma}{引理}
\newmdtheoremenv{corollary}{推论}
\newmdtheoremenv{example}{例}
\newenvironment{theorem}
{\bigskip\begin{mdframed}\begin{theo}}
    {\end{theo}\end{mdframed}\bigskip}
\newenvironment{exercise}
{\bigskip\begin{mdframed}\begin{exe}}
    {\end{exe}\end{mdframed}\bigskip}
\geometry{left=2.5cm,right=2.5cm,top=2.5cm,bottom=2.5cm}
\setCJKmainfont[BoldFont=SimHei]{SimSun}
\renewcommand{\today}{\number\year 年 \number\month 月 \number\day 日}
\newcommand{\D}{\displaystyle}\newcommand{\ri}{\Rightarrow}
\newcommand{\ds}{\displaystyle} \renewcommand{\ni}{\noindent}
\newcommand{\pa}{\partial} \newcommand{\Om}{\Omega}
\newcommand{\om}{\omega} \newcommand{\sik}{\sum_{i=1}^k}
\newcommand{\vov}{\Vert\omega\Vert} \newcommand{\Umy}{U_{\mu_i,y^i}}
\newcommand{\lamns}{\lambda_n^{^{\scriptstyle\sigma}}}
\newcommand{\chiomn}{\chi_{_{\Omega_n}}}
\newcommand{\ullim}{\underline{\lim}} \newcommand{\bsy}{\boldsymbol}
\newcommand{\mvb}{\mathversion{bold}} \newcommand{\la}{\lambda}
\newcommand{\La}{\Lambda} \newcommand{\va}{\varepsilon}
\newcommand{\be}{\beta} \newcommand{\al}{\alpha}
\newcommand{\dis}{\displaystyle} \newcommand{\R}{{\mathbb R}}
\newcommand{\N}{{\mathbb N}} \newcommand{\cF}{{\mathcal F}}
\newcommand{\gB}{{\mathfrak B}} \newcommand{\eps}{\epsilon}
\renewcommand\refname{参考文献}\renewcommand\figurename{图}
\usepackage[]{caption2} 
\renewcommand{\captionlabeldelim}{}
\begin{document}
\title{\huge{\bf{吕林根,许子道《解析几何》习题3.7.1}}} \author{\small{叶卢
    庆\footnote{叶卢庆(1992---),男,杭州师范大学理学院数学与应用数学专业
      本科在读,E-mail:yeluqingmathematics@gmail.com}}}
\maketitle
\begin{exercise}
直线方程
$$
\begin{cases}
  A_1x+B_1y+C_1z+D_1=0,\\
A_2x+B_2y+C_2z+D_2=0
\end{cases}
$$
的系数满足什么条件才能让
\begin{itemize}
\item 直线与$x$轴相交.
\item 直线与$x$轴平行.
\item 直线与$x$轴重合.
\end{itemize}
\end{exercise}
\begin{itemize}
\item 直线与$x$轴相交,说明当$y=z=0$时,$x$有唯一解.即
$$
\begin{cases}
  A_1x+D_1=0,\\
A_2x+D_2=0
\end{cases}
$$
有唯一解.于是,$A_1,A_2$必须不全为零.且当$A_i=0$时,$D_i=0$,且当
$A_1,A_2$全不为零时,有
$$
\frac{D_1}{A_1}=\frac{D_2}{A_2}.
$$
\item 直线的方向向量经过计算可得为
$$
(A_1,B_1,C_1)\times (A_2,B_2,C_2)=(B_1C_2-B_2C_1,A_2C_1-A_1C_2,A_1B_2-A_2B_1).
$$
必须使得$A_2C_1-A_1C_2=A_1B_2-A_2B_1=0$,且$B_1C_2-B_2C_1\neq 0$.而
且,$D_1,D_2$不全为零.
\item $A_2C_1-A_1C_2=A_1B_2-A_2B_1=0$,且$B_1C_2-B_2C_1\neq 0$.而
且,$D_1,D_2$全为零
\end{itemize}
\end{document}
