\documentclass[a4paper]{article}
\usepackage{amsmath,amsfonts,amsthm,amssymb}
\usepackage{bm}
\usepackage{hyperref}
\usepackage{geometry}
\usepackage{yhmath}
\usepackage{pstricks-add}
\usepackage{framed,mdframed}
\usepackage{graphicx,color} 
\usepackage{mathrsfs,xcolor} 
\usepackage[all]{xy}
\usepackage{fancybox} 
\usepackage{xeCJK}
\newtheorem*{theo}{定理}
\newtheorem*{exe}{题目}
\newtheorem*{rem}{评论}
\newtheorem*{lemma}{引理}
\newtheorem*{coro}{推论}
\newtheorem*{exa}{例}
\newenvironment{corollary}
{\bigskip\begin{mdframed}\begin{coro}}
    {\end{coro}\end{mdframed}\bigskip}
\newenvironment{theorem}
{\bigskip\begin{mdframed}\begin{theo}}
    {\end{theo}\end{mdframed}\bigskip}
\newenvironment{exercise}
{\bigskip\begin{mdframed}\begin{exe}}
    {\end{exe}\end{mdframed}\bigskip}
\newenvironment{example}
{\bigskip\begin{mdframed}\begin{exa}}
    {\end{exa}\end{mdframed}\bigskip}
\newenvironment{remark}
{\bigskip\begin{mdframed}\begin{rem}}
    {\end{rem}\end{mdframed}\bigskip}
\geometry{left=2.5cm,right=2.5cm,top=2.5cm,bottom=2.5cm}
\setCJKmainfont[BoldFont=SimHei]{SimSun}
\renewcommand{\today}{\number\year 年 \number\month 月 \number\day 日}
\newcommand{\D}{\displaystyle}\newcommand{\ri}{\Rightarrow}
\newcommand{\ds}{\displaystyle} \renewcommand{\ni}{\noindent}
\newcommand{\ov}{\overrightarrow}
\newcommand{\pa}{\partial} \newcommand{\Om}{\Omega}
\newcommand{\om}{\omega} \newcommand{\sik}{\sum_{i=1}^k}
\newcommand{\vov}{\Vert\omega\Vert} \newcommand{\Umy}{U_{\mu_i,y^i}}
\newcommand{\lamns}{\lambda_n^{^{\scriptstyle\sigma}}}
\newcommand{\chiomn}{\chi_{_{\Omega_n}}}
\newcommand{\ullim}{\underline{\lim}} \newcommand{\bsy}{\boldsymbol}
\newcommand{\mvb}{\mathversion{bold}} \newcommand{\la}{\lambda}
\newcommand{\La}{\Lambda} \newcommand{\va}{\varepsilon}
\newcommand{\be}{\beta} \newcommand{\al}{\alpha}
\newcommand{\dis}{\displaystyle} \newcommand{\R}{{\mathbb R}}
\newcommand{\N}{{\mathbb N}} \newcommand{\cF}{{\mathcal F}}
\newcommand{\gB}{{\mathfrak B}} \newcommand{\eps}{\epsilon}
\renewcommand\refname{参考文献}\renewcommand\figurename{图}
\usepackage[]{caption2} 
\renewcommand{\captionlabeldelim}{}
\setlength\parindent{0pt}
\begin{document}
\title{\huge{\bf{单叶双曲面异族直母线共面}}} \author{\small{叶卢庆\footnote{叶卢庆(1992---),男,杭州师范大学理学院数学与应用数学专业本科在读,E-mail:yeluqingmathematics@gmail.com}}}
\maketitle
\begin{theorem}
单叶双曲面异族直母线共面.
\end{theorem}
\begin{proof}[\textbf{证明}]
对于单叶双曲面
$$
\frac{x^2}{a^2}+\frac{y^2}{b^2}-\frac{z^2}{c^2}=1
$$
来说,它的方程可以化为
$$
\left(\frac{x}{a}+\frac{z}{c}\right)\left(\frac{x}{a}-\frac{z}{c}\right)=\left(1+\frac{y}{b}\right)\left(1-\frac{y}{b}\right).
$$
因此该单叶双曲面的$u$族直母线为
$$
\begin{cases}
  \frac{x}{a}+\frac{z}{c}=u(1+\frac{y}{b}),\\
u(\frac{x}{a}-\frac{z}{c})=1-\frac{y}{b}.
\end{cases},
$$
$v$族直母线为
$$
\begin{cases}
  \frac{x}{a}+\frac{z}{c}=v(1-\frac{y}{b}),\\
v(\frac{x}{a}-\frac{z}{c})=1+\frac{y}{b}.
\end{cases}
$$
下面我们来证明$u$族直母线和$v$族直母线共面.为此,只用证明
$$
\begin{vmatrix}
  \frac{1}{a}&\frac{-u}{b}&\frac{1}{c}&-u\\
\frac{u}{a}&\frac{1}{b}&\frac{-u}{c}&-1\\
\frac{1}{a}&\frac{v}{b}&\frac{1}{c}&-v\\
\frac{v}{a}&\frac{-1}{b}&\frac{-v}{c}&-1
\end{vmatrix}=0,
$$
也就是证明
\begin{align*}
\frac{1}{a}\left(-\frac{2}{bc}-\frac{2uv}{bc}\right)+\frac{u}{b}\left(\frac{-2u}{ac}+\frac{2v}{ac}\right)+\frac{1}{c}\left(\frac{-2uv}{ab}+\frac{2}{ab}\right)+u\left(\frac{2v}{abc}+\frac{2u}{abc}\right)=0
\end{align*}
这是容易验证的.所以单叶双曲面异族直母线共面.
\end{proof}
\end{document}
