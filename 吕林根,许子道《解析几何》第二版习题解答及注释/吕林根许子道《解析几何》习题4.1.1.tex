\documentclass[a4paper]{article}
\usepackage{amsmath,amsfonts,amsthm,amssymb}
\usepackage{bm}
\usepackage{draftwatermark,euler}
\SetWatermarkText{http://blog.sciencenet.cn/u/Yaleking}%设置水印文字
\SetWatermarkLightness{0.8}%设置水印亮度
\SetWatermarkScale{0.35}%设置水印大小
\usepackage{hyperref}
\usepackage{geometry}
\usepackage{yhmath}
\usepackage{pstricks-add}
\usepackage{framed,mdframed}
\usepackage{graphicx,color} 
\usepackage{mathrsfs,xcolor} 
\usepackage[all]{xy}
\usepackage{fancybox} 
\usepackage{xeCJK}
\newtheorem*{theo}{定理}
\newtheorem*{exe}{题目}
\newtheorem*{rem}{评论}
\newmdtheoremenv{lemma}{引理}
\newmdtheoremenv{corollary}{推论}
\newmdtheoremenv{example}{例}
\newenvironment{theorem}
{\bigskip\begin{mdframed}\begin{theo}}
    {\end{theo}\end{mdframed}\bigskip}
\newenvironment{exercise}
{\bigskip\begin{mdframed}\begin{exe}}
    {\end{exe}\end{mdframed}\bigskip}
\geometry{left=2.5cm,right=2.5cm,top=2.5cm,bottom=2.5cm}
\setCJKmainfont[BoldFont=SimHei]{SimSun}
\renewcommand{\today}{\number\year 年 \number\month 月 \number\day 日}
\newcommand{\D}{\displaystyle}\newcommand{\ri}{\Rightarrow}
\newcommand{\ds}{\displaystyle} \renewcommand{\ni}{\noindent}
\newcommand{\ov}{\overrightarrow}
\newcommand{\pa}{\partial} \newcommand{\Om}{\Omega}
\newcommand{\om}{\omega} \newcommand{\sik}{\sum_{i=1}^k}
\newcommand{\vov}{\Vert\omega\Vert} \newcommand{\Umy}{U_{\mu_i,y^i}}
\newcommand{\lamns}{\lambda_n^{^{\scriptstyle\sigma}}}
\newcommand{\chiomn}{\chi_{_{\Omega_n}}}
\newcommand{\ullim}{\underline{\lim}} \newcommand{\bsy}{\boldsymbol}
\newcommand{\mvb}{\mathversion{bold}} \newcommand{\la}{\lambda}
\newcommand{\La}{\Lambda} \newcommand{\va}{\varepsilon}
\newcommand{\be}{\beta} \newcommand{\al}{\alpha}
\newcommand{\dis}{\displaystyle} \newcommand{\R}{{\mathbb R}}
\newcommand{\N}{{\mathbb N}} \newcommand{\cF}{{\mathcal F}}
\newcommand{\gB}{{\mathfrak B}} \newcommand{\eps}{\epsilon}
\renewcommand\refname{参考文献}\renewcommand\figurename{图}
\usepackage[]{caption2} 
\renewcommand{\captionlabeldelim}{}
\begin{document}
\title{\huge{\bf{吕林根,许子道《解析几何》习题4.1.1}}} \author{\small{叶卢庆\footnote{叶卢庆(1992---),男,杭州师范大学理学院数学与应用数学专业本科在读,E-mail:yeluqingmathematics@gmail.com}}}
\maketitle
\begin{exercise}
已知柱面的准线为
$$
\begin{cases}
  (x-1)^2+(y+3)^2+(z-2)^2=25,\\
x+y-z+2=0,
\end{cases}
$$
求下列柱面的方程:
\begin{itemize}
\item 母线平行于$x$轴.
\item 母线平行于直线$x=y,z=c$.
\end{itemize}
\end{exercise}
\begin{itemize}
\item 设柱面上任意一点为$(x,y,z)$,则总存在$t\in \mathbf{R}$,使得
  $(x+t,y,z)$满足准线方程.也即,
$$
\begin{cases}
  (x+t-1)^2+(y+3)^2+(z-2)^2=25,\\
x+t+y-z+2=0.
\end{cases}
$$
消去$t$,解得
$$
(y-z+3)^2+(y+3)^2+(z-2)^2=25,
$$
即
$$
2y^2+2z^2-2yz+12y+2z=3.
$$
这就是柱面方程.
\item 设柱面上任意一点$(x,y,z)$,总存在实数$t$,使得$(x+t,y+t,z)$满足柱
  面的准线方程,也即,
$$
\begin{cases}
  (x+t-1)^2+(y+t+3)^2+(z-2)^2=25,\\
x+t+y+t-z+2=0.
\end{cases}
$$
消去$t$,可得
$$
(x+z-4-y)^2+(y+z-x+4)^2+(2z-4)^2=100.
$$
此即为柱面的方程.
\end{itemize}
\end{document}
