\documentclass[a4paper]{article}
\usepackage{amsmath,amsfonts,amsthm,amssymb}
\usepackage{bm}
\usepackage{draftwatermark,euler}
\SetWatermarkText{http://blog.sciencenet.cn/u/Yaleking}%设置水印文字
\SetWatermarkLightness{0.8}%设置水印亮度
\SetWatermarkScale{0.35}%设置水印大小
\usepackage{hyperref}
\usepackage{geometry}
\usepackage{yhmath}
\usepackage{pstricks-add}
\usepackage{framed,mdframed}
\usepackage{graphicx,color} 
\usepackage{mathrsfs,xcolor} 
\usepackage[all]{xy}
\usepackage{fancybox} 
\usepackage{xeCJK}
\newtheorem*{theo}{定理}
\newtheorem*{exe}{题目}
\newtheorem*{rem}{评论}
\newmdtheoremenv{lemma}{引理}
\newmdtheoremenv{corollary}{推论}
\newmdtheoremenv{example}{例}
\newenvironment{theorem}
{\bigskip\begin{mdframed}\begin{theo}}
    {\end{theo}\end{mdframed}\bigskip}
\newenvironment{exercise}
{\bigskip\begin{mdframed}\begin{exe}}
    {\end{exe}\end{mdframed}\bigskip}
\geometry{left=2.5cm,right=2.5cm,top=2.5cm,bottom=2.5cm}
\setCJKmainfont[BoldFont=SimHei]{SimSun}
\renewcommand{\today}{\number\year 年 \number\month 月 \number\day 日}
\newcommand{\D}{\displaystyle}\newcommand{\ri}{\Rightarrow}
\newcommand{\ds}{\displaystyle} \renewcommand{\ni}{\noindent}
\newcommand{\pa}{\partial} \newcommand{\Om}{\Omega}
\newcommand{\om}{\omega} \newcommand{\sik}{\sum_{i=1}^k}
\newcommand{\vov}{\Vert\omega\Vert} \newcommand{\Umy}{U_{\mu_i,y^i}}
\newcommand{\lamns}{\lambda_n^{^{\scriptstyle\sigma}}}
\newcommand{\chiomn}{\chi_{_{\Omega_n}}}
\newcommand{\ullim}{\underline{\lim}} \newcommand{\bsy}{\boldsymbol}
\newcommand{\mvb}{\mathversion{bold}} \newcommand{\la}{\lambda}
\newcommand{\La}{\Lambda} \newcommand{\va}{\varepsilon}
\newcommand{\be}{\beta} \newcommand{\al}{\alpha}
\newcommand{\dis}{\displaystyle} \newcommand{\R}{{\mathbb R}}
\newcommand{\N}{{\mathbb N}} \newcommand{\cF}{{\mathcal F}}
\newcommand{\gB}{{\mathfrak B}} \newcommand{\eps}{\epsilon}
\renewcommand\refname{参考文献}\renewcommand\figurename{图}
\usepackage[]{caption2} 
\renewcommand{\captionlabeldelim}{}
\begin{document}
\title{\huge{\bf{有轴平面束的方程}}} \author{\small{叶卢
    庆\footnote{叶卢庆(1992---),男,杭州师范大学理学院数学与应用数学专业
      本科在读,E-mail:yeluqingmathematics@gmail.com}}}
\maketitle\ni
我们来证明如下的定理,来自吕林根,许子道编的《解析几何》第四版定理3.8.2.
\begin{theorem}
如果两个平面
\begin{equation}
  \label{eq:1}
  \pi_1:A_1x+B_1y+C_1z+D_1=0,
\end{equation}
\begin{equation}
  \label{eq:2}
  \pi_2:A_2x+B_2y+C_2z+D_2=0.
\end{equation}
交于一条直线$L$,那么以直线$L$为轴的有轴平面束的方程是
\begin{equation}
  \label{eq:3}
  l(A_1x+B_1y+C_1z+D_1)+m(A_2x+B_2y+C_2z+D_2)=0,
\end{equation}
其中$l,m$是不全为零的任意实数.
\end{theorem}
\begin{proof}[\textbf{证明}]
首先,容易证明方程\eqref{eq:3}代表的平面都经过直线$L$,关键是证明,任意经
过直线$L$的平面都可以写成方程\eqref{eq:3}的形式.这是很简单的事实,因为
平面\eqref{eq:3}的法向量是
$$
l(A_1,B_1,C_1)+m(A_2,B_2,C_2),
$$
这是平面\eqref{eq:1}的法向量和平面\eqref{eq:2}的法向量的线性组合,当
$l,m$遍历实数时,向量$l(A_1,B_1,C_1)+m(A_2,B_2,C_2)$能取遍跟直线$L$垂直
的所有方向,这就表明了所有经过直线$L$的平面都能写成方程\eqref{eq:3}的形
式.
\end{proof}
\end{document}
