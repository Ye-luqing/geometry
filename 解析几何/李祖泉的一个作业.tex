\documentclass[a4paper]{article}
\usepackage{amsmath,amsfonts,amsthm,amssymb}
\usepackage{bm,euler}
\usepackage{hyperref}
\usepackage{geometry}
\usepackage{yhmath}
\usepackage{pstricks-add}
\usepackage{framed,mdframed}
\usepackage{graphicx,color} 
\usepackage{mathrsfs,xcolor} 
\usepackage[all]{xy}
\usepackage{fancybox} 
\usepackage{xeCJK}
\newtheorem*{theo}{定理}
\newtheorem*{exe}{题目}
\newtheorem*{rem}{评论}
\newtheorem*{lemma}{引理}
\newtheorem*{coro}{推论}
\newtheorem*{exa}{例}
\newenvironment{corollary}
{\bigskip\begin{mdframed}\begin{coro}}
    {\end{coro}\end{mdframed}\bigskip}
\newenvironment{theorem}
{\bigskip\begin{mdframed}\begin{theo}}
    {\end{theo}\end{mdframed}\bigskip}
\newenvironment{exercise}
{\bigskip\begin{mdframed}\begin{exe}}
    {\end{exe}\end{mdframed}\bigskip}
\newenvironment{example}
{\bigskip\begin{mdframed}\begin{exa}}
    {\end{exa}\end{mdframed}\bigskip}
\newenvironment{remark}
{\bigskip\begin{mdframed}\begin{rem}}
    {\end{rem}\end{mdframed}\bigskip}
\geometry{left=2.5cm,right=2.5cm,top=2.5cm,bottom=2.5cm}
\setCJKmainfont[BoldFont=SimHei]{SimSun}
\renewcommand{\today}{\number\year 年 \number\month 月 \number\day 日}
\newcommand{\D}{\displaystyle}\newcommand{\ri}{\Rightarrow}
\newcommand{\ds}{\displaystyle} \renewcommand{\ni}{\noindent}
\newcommand{\ov}{\overrightarrow}
\newcommand{\pa}{\partial} \newcommand{\Om}{\Omega}
\newcommand{\om}{\omega} \newcommand{\sik}{\sum_{i=1}^k}
\newcommand{\vov}{\Vert\omega\Vert} \newcommand{\Umy}{U_{\mu_i,y^i}}
\newcommand{\lamns}{\lambda_n^{^{\scriptstyle\sigma}}}
\newcommand{\chiomn}{\chi_{_{\Omega_n}}}
\newcommand{\ullim}{\underline{\lim}} \newcommand{\bsy}{\boldsymbol}
\newcommand{\mvb}{\mathversion{bold}} \newcommand{\la}{\lambda}
\newcommand{\La}{\Lambda} \newcommand{\va}{\varepsilon}
\newcommand{\be}{\beta} \newcommand{\al}{\alpha}
\newcommand{\dis}{\displaystyle} \newcommand{\R}{{\mathbb R}}
\newcommand{\N}{{\mathbb N}} \newcommand{\cF}{{\mathcal F}}
\newcommand{\gB}{{\mathfrak B}} \newcommand{\eps}{\epsilon}
\renewcommand\refname{参考文献}\renewcommand\figurename{图}
\usepackage[]{caption2} 
\renewcommand{\captionlabeldelim}{}
\setlength\parindent{0pt}
\begin{document}
\title{\huge{\bf{李祖泉的一个题}}} \author{\small{叶卢庆\footnote{叶卢庆(1992---),男,杭州师范大学理学院数学与应用数学专业本科在读,E-mail:yeluqingmathematics@gmail.com}}}
\maketitle
\begin{exercise}
在空间中,设$xOy$平面上有圆心在$x$正半轴且过$O$的圆.一条动直线与此圆交
于$A$点,与$z$轴交于点$B$.且满足
$$
|OA|=k|OB|,
$$
$k$为一个正常数.求此动直线所产生的曲面方程.
\end{exercise}
\begin{proof}[\textbf{解}]
设圆的圆心为$(a,0,0)$,其中$a$是个正常数.则圆的方程为
\begin{equation}\label{eq:1}
\begin{cases}
z=0\\
(x-a)^2+y^2=a^2.
\end{cases}
\end{equation}
设点$A$的坐标为$(x_A,y_A,0)$,设点$B$的坐标为$(0,0,b)$.则点$A$满足方程
\eqref{eq:1},也即,
\begin{equation}\label{eq:1.5}
  x_A^2-2ax_A+y_{A}^2=0.
\end{equation}
且由题意,
\begin{equation}\label{eq:2}
x_A^2+y_A^2=k^2b^2.
\end{equation}
结合方程\eqref{eq:1.5}和\eqref{eq:2},可得
\begin{equation}
  \label{eq:2.5}
  2ax_A=k^2b^2.
\end{equation}
经过点$A,B$的直线方程为
\begin{equation}\label{eq:3}
\frac{x}{x_A}=\frac{y}{y_A}=\frac{z-b}{-b}=t.
\end{equation}
将式\eqref{eq:3}代入式\eqref{eq:1.5},可得
$$
  (\frac{x}{t})^2-2a \frac{x}{t}+(\frac{y}{t})^2=0,
$$
即
\begin{equation}
  \label{eq:4}
  x^2-2axt+y^2=0.
\end{equation}
将方程\eqref{eq:3}代入方程\eqref{eq:2.5},可得
\begin{equation}
  \label{eq:5}
  2a \frac{x}{t}=k^2b.
\end{equation}
由方程\eqref{eq:3}可得
\begin{equation}
  \label{eq:6}
  
\end{equation}

\end{proof}
\end{document}
