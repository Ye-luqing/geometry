\documentclass[a4paper]{article}
\usepackage{amsmath,amsfonts,amsthm,amssymb}
\usepackage{bm}
\usepackage{hyperref}
\usepackage{geometry}
\usepackage{yhmath}
\usepackage{pstricks-add}
\usepackage{framed,mdframed}
\usepackage{graphicx,color} 
\usepackage{mathrsfs,xcolor} 
\usepackage[all]{xy}
\usepackage{fancybox} 
\usepackage{xeCJK}
\newtheorem*{theo}{定理}
\newtheorem*{exe}{题目}
\newtheorem*{rem}{评论}
\newtheorem*{lemma}{引理}
\newtheorem*{coro}{推论}
\newtheorem*{exa}{例}
\newenvironment{corollary}
{\bigskip\begin{mdframed}\begin{coro}}
    {\end{coro}\end{mdframed}\bigskip}
\newenvironment{theorem}
{\bigskip\begin{mdframed}\begin{theo}}
    {\end{theo}\end{mdframed}\bigskip}
\newenvironment{exercise}
{\bigskip\begin{mdframed}\begin{exe}}
    {\end{exe}\end{mdframed}\bigskip}
\newenvironment{example}
{\bigskip\begin{mdframed}\begin{exa}}
    {\end{exa}\end{mdframed}\bigskip}
\newenvironment{remark}
{\bigskip\begin{mdframed}\begin{rem}}
    {\end{rem}\end{mdframed}\bigskip}
\geometry{left=2.5cm,right=2.5cm,top=2.5cm,bottom=2.5cm}
\setCJKmainfont[BoldFont=SimHei]{SimSun}
\renewcommand{\today}{\number\year 年 \number\month 月 \number\day 日}
\newcommand{\D}{\displaystyle}\newcommand{\ri}{\Rightarrow}
\newcommand{\ds}{\displaystyle} \renewcommand{\ni}{\noindent}
\newcommand{\ov}{\overrightarrow}
\newcommand{\pa}{\partial} \newcommand{\Om}{\Omega}
\newcommand{\om}{\omega} \newcommand{\sik}{\sum_{i=1}^k}
\newcommand{\vov}{\Vert\omega\Vert} \newcommand{\Umy}{U_{\mu_i,y^i}}
\newcommand{\lamns}{\lambda_n^{^{\scriptstyle\sigma}}}
\newcommand{\chiomn}{\chi_{_{\Omega_n}}}
\newcommand{\ullim}{\underline{\lim}} \newcommand{\bsy}{\boldsymbol}
\newcommand{\mvb}{\mathversion{bold}} \newcommand{\la}{\lambda}
\newcommand{\La}{\Lambda} \newcommand{\va}{\varepsilon}
\newcommand{\be}{\beta} \newcommand{\al}{\alpha}
\newcommand{\dis}{\displaystyle} \newcommand{\R}{{\mathbb R}}
\newcommand{\N}{{\mathbb N}} \newcommand{\cF}{{\mathcal F}}
\newcommand{\gB}{{\mathfrak B}} \newcommand{\eps}{\epsilon}
\renewcommand\refname{参考文献}\renewcommand\figurename{图}
\usepackage[]{caption2} 
\renewcommand{\captionlabeldelim}{}
\setlength\parindent{0pt}
\begin{document}
\title{\huge{\bf{三角形垂心的存在性}}} \author{\small{叶卢庆\footnote{叶卢庆(1992---),男,杭州师范大学理学院数学与应用数学专业本科在读,E-mail:yeluqingmathematics@gmail.com}}}
\maketitle
如图\eqref{fig:1},设$H$是三角形$\triangle ABC$的垂心.
\begin{figure}[h]
\newrgbcolor{zzttqq}{0.6 0.2 0.}
\newrgbcolor{xdxdff}{0.490196078431 0.490196078431 1.}
\psset{xunit=1.0cm,yunit=1.0cm,algebraic=true,dimen=middle,dotstyle=o,dotsize=3pt 0,linewidth=0.8pt,arrowsize=3pt 2,arrowinset=0.25}
\begin{pspicture*}(-1.3,-5.82)(22.8,6.3)
\pspolygon[linecolor=zzttqq,fillcolor=zzttqq,fillstyle=solid,opacity=0.1](6.46,4.42)(2.5,-2.32)(13.88,-2.74)
\psline[linecolor=zzttqq](6.46,4.42)(2.5,-2.32)
\psline[linecolor=zzttqq](2.5,-2.32)(13.88,-2.74)
\psline[linecolor=zzttqq](13.88,-2.74)(6.46,4.42)
\psline(6.46,4.42)(6.16018136841,-2.45508577985)
\psline(2.5,-2.32)(7.51000739372,3.40678531819)
\psline(13.88,-2.74)(5.44992662722,2.700834714)
\begin{scriptsize}
\psdots[dotsize=1pt 0,dotstyle=*,linecolor=blue](6.46,4.42)
\rput[bl](6.42,4.62){\blue{$A$}}
\psdots[dotsize=1pt 0,dotstyle=*,linecolor=blue](2.5,-2.32)
\rput[bl](2.24,-2.56){\blue{$B$}}
\psdots[dotsize=1pt 0,dotstyle=*,linecolor=blue](13.88,-2.74)
\rput[bl](14.,-2.94){\blue{$C$}}
\psdots[dotsize=1pt 0,dotstyle=*,linecolor=xdxdff](6.16018136841,-2.45508577985)
\psdots[dotsize=1pt 0,dotstyle=*,linecolor=xdxdff](7.51000739372,3.40678531819)
\psdots[dotsize=1pt 0,dotstyle=*,linecolor=xdxdff](5.44992662722,2.700834714)
\psdots[dotstyle=*,linecolor=darkgray](6.35840868421,2.09042826245)
\rput[bl](6.44,1.72){\darkgray{$H$}}
\end{scriptsize}
\end{pspicture*}
  \caption{}
  \label{fig:1}
\end{figure}
则
$$
\begin{cases}
  \ov{AH}\cdot \left(\ov{AC}-\ov{AB}\right)=0,\\
\left(\ov{AH}-\ov{AB}\right)\cdot \ov{AC}=0,\\
\left(\ov{AH}-\ov{AC}\right)\cdot \ov{AB}=0
\end{cases}.
$$
易得从第一条式子和第二条式子出发可以推出第三条式子,因此三角形的三条高
线始终是交于一点的,这也论证了垂心的存在性.
\end{document}
