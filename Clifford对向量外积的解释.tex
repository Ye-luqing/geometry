\documentclass[a4paper]{article}
\usepackage{amsmath,amsfonts,amsthm,amssymb} \usepackage{bm}
\usepackage{draftwatermark,euler}
\SetWatermarkText{http://blog.sciencenet.cn/u/Yaleking}%设置水印文字
\SetWatermarkLightness{0.8}%设置水印亮度
\SetWatermarkScale{0.35}%设置水印大小
\usepackage{hyperref} \usepackage{geometry} \usepackage{yhmath}
\usepackage{pstricks-add} \usepackage{framed,mdframed}
\usepackage{graphicx,color} \usepackage{mathrsfs,xcolor}
\usepackage[all]{xy} \usepackage{fancybox} \usepackage{xeCJK,csquotes}
\usepackage{asymptote}
\newtheorem*{theo}{定理} 
\newtheorem*{exa}{}
\newenvironment{theorem}
{\bigskip\begin{mdframed}\begin{theo}}
    {\end{theo}\end{mdframed}\bigskip} 
\newenvironment{example}
{\bigskip\begin{mdframed}\begin{exa}}
    {\end{exa}\end{mdframed}\bigskip}
\geometry{left=2.5cm,right=2.5cm,top=2.5cm,bottom=2.5cm}
\setCJKmainfont[BoldFont=SimHei]{SimSun}
\setlength\parindent{0pt}
\newcommand{\ov}{\overrightarrow}\newcommand{\ri}{\Rightarrow}
\newcommand{\ds}{\displaystyle} \renewcommand{\ni}{\noindent}
\newcommand{\pa}{\partial} \newcommand{\Om}{\Omega}
\newcommand{\om}{\omega} \newcommand{\sik}{\sum_{i=1}^k}
\newcommand{\vov}{\Vert\omega\Vert} \newcommand{\Umy}{U_{\mu_i,y^i}}
\newcommand{\lamns}{\lambda_n^{^{\scriptstyle\sigma}}}
\newcommand{\chiomn}{\chi_{_{\Omega_n}}}
\newcommand{\ullim}{\underline{\lim}} \newcommand{\bsy}{\boldsymbol}
\newcommand{\mvb}{\mathversion{bold}} \newcommand{\la}{\lambda}
\newcommand{\La}{\Lambda} \newcommand{\va}{\varepsilon}
\newcommand{\be}{\beta} \newcommand{\al}{\alpha}
\newcommand{\dis}{\displaystyle} \newcommand{\R}{{\mathbb R}}
\renewcommand{\today}{\number\year 年 \number\month 月 \number\day 日}
\newcommand{\N}{{\mathbb N}} \newcommand{\cF}{{\mathcal F}}
\newcommand{\gB}{{\mathfrak B}} \newcommand{\eps}{\epsilon}
\renewcommand\refname{参考文献}\renewcommand\figurename{图}
\usepackage[]{caption2} \renewcommand{\captionlabeldelim}{}
\begin{document}
\title{\huge{\bf{Clifford对向量外积的解释}}} \author{\small{叶卢
    庆\footnote{叶卢庆(1992-),男,杭州师范大学理学院数学112.学
      号:1002011005.E-mail:yeluqingmathematics@gmail.com}}}
\maketitle
向量的外积是数学家在长期的数学实践中逐渐提炼出来的概念.正如很多数学概念
一样,了解向量外积的历史发展过程,对于我们理解这个概念是有益的.在这篇短
文里,我们来阐述 William Kingdon Clifford 对向量外积的解释.我
只阐述Clifford的精神,而不打算使用他的符号以及词汇.\\

\begin{figure}[h]
  \centering \newrgbcolor{zzttqq}{0.6 0.2 0.}
  \psset{xunit=1.0cm,yunit=1.0cm,algebraic=true,dimen=middle,dotstyle=o,dotsize=3pt
    0,linewidth=0.8pt,arrowsize=3pt 2,arrowinset=0.25}
  \begin{pspicture*}(-1.86,-4.42)(8,6)
    \pspolygon[linecolor=zzttqq,fillcolor=zzttqq,fillstyle=solid,opacity=0.1](0.,0.)(6.,2.)(7.,5.)(1.,3.)
    \psline[linecolor=zzttqq](0.,0.)(6.,2.)
    \psline[linecolor=zzttqq](6.,2.)(7.,5.)
    \psline[linecolor=zzttqq](7.,5.)(1.,3.)
    \psline[linecolor=zzttqq](1.,3.)(0.,0.)  \psline{->}(0.,0.)(6.,2.)
    \psline{->}(0.,0.)(1.,3.)
    \begin{scriptsize}
      % \psdots[dotstyle=*,linecolor=darkgray](0.,0.)
      \rput[bl](-0.28,-0.16){\blue{$A$}}
      % \psdots[dotstyle=*,linecolor=blue](6.,2.)
      \rput[bl](6.2,1.74){\blue{$B$}}
      % \psdots[dotstyle=*,linecolor=blue](7.,5.)
      \rput[bl](7.08,5.12){\blue{$C$}}
      % \psdots[dotstyle=*,linecolor=blue](1.,3.)
      \rput[bl](0.64,3.08){\blue{$D$}}
    \end{scriptsize}
  \end{pspicture*}\caption{}\label{fig:1}
\end{figure}

1878年,Clifford发表了著作\textit{Elements of Dynamic}.在这部著作
的\href{http://ebooks.library.cornell.edu/cgi/t/text/pageviewer-idx?c=math;cc=math;idno=04370002;view=image;seq=104;size=100;page=root}{第三章},Clifford把平面区域的面积看成一个有方向的量.他用描述性
的语言说, \blockquote{Every plane area is to be regarded as a directed
  quantity.It is represented by a vector drawn perpendicular to its
  plane,containing as many linear centimeters as there are square
  centimeters in the area.The vector must be drawn towards that side
  of the plane from which the area appears to be gone round
  counter-clockwise.}  \blockquote{每个平面区域都将被视作一个有方向的
  量,由和区域垂直的向量表示.其中向量的长度在数值上等于区域的面积,且区域
  在向量所朝向的那一面是逆时针走向.}\\\\

Clifford的表达模糊而朴素.但是我们可以将其意图表达的观念精确地应用到平行四边
形区域的情形.如图\eqref{fig:1},平行四边形的邻边为$AB$和$AD$.平行四边形区域的
边界有两种走向,它们分别是$A-B-C-D-A$,以及,$A-D-C-B-A$.当平行四边形
以$A-B-C-D-A$为走向时,用右手四指环绕$A-B-C-D-A$,右手大拇指的方向为平行
四边形对应的向量的方向,此时,向量的长度在数值上等于平行四边形的面积,为$|\ov{AB}||\ov{AD}|\sin\angle
BAD$,向量记为$\ov{AB}\times \ov{AD}$.当平行四边形区域的边界以$A-D-C-B-A$为走向时,用
右手四指环绕$A-D-C-B-A$,右手大拇指的方向为平行四边形对应的向量的方向,此
时向量的长度依然为$|\ov{AB}||\ov{AD}|\sin\angle DAB$,向量记为
$\ov{AD}\times\ov{AD}$.易得
$$
\ov{AB}\times \ov{AD}=-\ov{AD}\times \ov{AB}.
$$


Clifford在第三章着重证明了向量外积关于向量加法的分配律.我们知道,向量外
积的其它基本性质都是平凡的,只有外积关于向量加法的分配律不是平凡的,我想
这就是他为什么只叙述并证明这个定理的原因.他的证明是几何的,我不打算在此
重叙之.取而代之的是,我在此叙述自己发现
的证明.我的证明表明,{\color{red}向量外积关于向量加法分配律的本质,是平行四边形区域和平行四边形
区域对应的向量之间存在对偶关系}.向量外积关于加法的分配律是说,
\begin{equation}
  \label{eq:1}
  \ov{A'A}\times \ov{A'B}+\ov{A'A}\times \ov{A'C}=\ov{A'A}\times(\ov{A'B}+\ov{A'C}),
\end{equation}
其中 $A'$ 是空间中的任意一点.下面我们用汉语来说明式\eqref{eq:1}的正确
性.我们看
图\eqref{fig:2}.图\eqref{fig:2}是一个平行六面体,其中平行四边形$ABDC$和
平行四边形$A'B'D'C'$是该平行六面体的两个底面.特别地,如果向
量$\ov{A'A},\ov{A'B},\ov{A'C}$共面,则平行六面体退化成为平面图形.为了证
明外积分配律,一个最重
要的洞察是,平行四边形区域$A'ADD',C'CDD',A'ACC'$之间存在着类似于三角形三边
的关系,确切地说,平行四边形$A'ADD',C'CDD',A'ACC'$围成了一个三角柱面,该
三角柱面的底面分别是$\triangle ACD$和$\triangle A'C'D'$.这个三角柱面相
当于一个三角形$W$,平行四边形区域$A'ADD',C'CDD',A'ACC'$相当于三角形$W$的三
条边,平行四边形区域的面积相当于三角形$W$的边长,三个平行四边形区域两两之间
形成的三个二面角相当于三角形$W$的三个内角.由于向量$\ov{A'A}\times
\ov{A'B}$,$\ov{A'A}\times \ov{A'C}$,$\ov{A'A}\times
(\ov{A'B}+\ov{A'C})$分别和平行四边形区域$C'CDD',A'ACC',A'ADD'$垂直,且
向量的长度和各自所对应的平行四边形区域面积相等,因此向量$\ov{A'A}\times
\ov{A'B}$,$\ov{A'A}\times \ov{A'C}$,$\ov{A'A}\times
(\ov{A'B}+\ov{A'C})$可以首尾相接封闭成为三角形的三条边,于是有式\eqref{eq:1}成立,
事实上,这三个向量首尾相接所封闭成的三角形正是三角形$W$的具体显化.这样我们就说明
了向量外积关于向量加法分配律是成立的.
\begin{figure}[h]
  \centering
\includegraphics[width=1\textwidth]{Clifford对向量外积和内积的解释-1+0_0.pdf}
  \caption{}
  \label{fig:2}
\end{figure}
\end{document}







由于向
量$\overrightarrow{A'A}\times \overrightarrow{A'C}$和平
面$A'ACC'$垂直,向量$\overrightarrow{A'A}\times \overrightarrow{A'B}$和
平面$A'ABB'$垂直,因此向量$\overrightarrow{A'A}\times
\ov{A'C}$和$\ov{A'A}\times \ov{A'B}$之间的夹角大小等于二面
角$A'ACC'-A'ABB'$的大小.图\eqref{fig:3}体现了这一点.









\begin{figure}[h]
  \centering \newrgbcolor{qqwuqq}{0. 0.392156862745 0.}
  \psset{xunit=1.0cm,yunit=1.0cm,algebraic=true,dimen=middle,dotstyle=o,dotsize=3pt
    0,linewidth=0.8pt,arrowsize=3pt 2,arrowinset=0.25}
  \begin{pspicture*}(-3.42,-5.64)(18.68,6.48)
    \psline{->}(3.,3.24)(3.24,-1.72) \psline{->}(3.,3.24)(7.14,-0.84)
    \rput[tl](0.46,1.1){$\overrightarrow{A'A}\times
      \overrightarrow{A'C}$} \rput[tl](5.46,1.66){$\ov{A'A}\times
      \ov{A'B}$}
    \pscustom[linecolor=qqwuqq,fillcolor=qqwuqq,fillstyle=solid,opacity=0.1]{
      \parametricplot{-1.5224469401429936}{-0.7780990229537511}{0.6*cos(t)+3.|0.6*sin(t)+3.24}
      \lineto(3.,3.24)\closepath} \rput[tl](3.3,3.74){$\alpha=\mbox{二
        面角}A'AC'C-A'AB'B.$}
    \begin{scriptsize}
      \rput[bl](3.08,2.82){\qqwuqq{$\alpha$}}
    \end{scriptsize}
  \end{pspicture*}
  \caption{}
  \label{fig:3}
\end{figure}