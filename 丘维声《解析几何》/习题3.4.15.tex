\documentclass[a4paper]{article}
\usepackage{amsmath,amsfonts,amsthm,amssymb}
\usepackage{bm}
\usepackage{hyperref}
\usepackage{geometry}
\usepackage{yhmath}
\usepackage{pstricks-add}
\usepackage{framed,mdframed}
\usepackage{graphicx,color} 
\usepackage{mathrsfs,xcolor} 
\usepackage[all]{xy}
\usepackage{fancybox} 
\usepackage{xeCJK}
\newtheorem*{theo}{定理}
\newtheorem*{exe}{题目}
\newtheorem*{rem}{评论}
\newtheorem*{lemma}{引理}
\newtheorem*{coro}{推论}
\newtheorem*{exa}{例}
\newenvironment{corollary}
{\bigskip\begin{mdframed}\begin{coro}}
    {\end{coro}\end{mdframed}\bigskip}
\newenvironment{theorem}
{\bigskip\begin{mdframed}\begin{theo}}
    {\end{theo}\end{mdframed}\bigskip}
\newenvironment{exercise}
{\bigskip\begin{mdframed}\begin{exe}}
    {\end{exe}\end{mdframed}\bigskip}
\newenvironment{example}
{\bigskip\begin{mdframed}\begin{exa}}
    {\end{exa}\end{mdframed}\bigskip}
\newenvironment{remark}
{\bigskip\begin{mdframed}\begin{rem}}
    {\end{rem}\end{mdframed}\bigskip}
\geometry{left=2.5cm,right=2.5cm,top=2.5cm,bottom=2.5cm}
\setCJKmainfont[BoldFont=SimHei]{SimSun}
\renewcommand{\today}{\number\year 年 \number\month 月 \number\day 日}
\newcommand{\D}{\displaystyle}\newcommand{\ri}{\Rightarrow}
\newcommand{\ds}{\displaystyle} \renewcommand{\ni}{\noindent}
\newcommand{\ov}{\overrightarrow}
\newcommand{\pa}{\partial} \newcommand{\Om}{\Omega}
\newcommand{\om}{\omega} \newcommand{\sik}{\sum_{i=1}^k}
\newcommand{\vov}{\Vert\omega\Vert} \newcommand{\Umy}{U_{\mu_i,y^i}}
\newcommand{\lamns}{\lambda_n^{^{\scriptstyle\sigma}}}
\newcommand{\chiomn}{\chi_{_{\Omega_n}}}
\newcommand{\ullim}{\underline{\lim}} \newcommand{\bsy}{\boldsymbol}
\newcommand{\mvb}{\mathversion{bold}} \newcommand{\la}{\lambda}
\newcommand{\La}{\Lambda} \newcommand{\va}{\varepsilon}
\newcommand{\be}{\beta} \newcommand{\al}{\alpha}
\newcommand{\dis}{\displaystyle} \newcommand{\R}{{\mathbb R}}
\newcommand{\N}{{\mathbb N}} \newcommand{\cF}{{\mathcal F}}
\newcommand{\gB}{{\mathfrak B}} \newcommand{\eps}{\epsilon}
\renewcommand\refname{参考文献}\renewcommand\figurename{图}
\usepackage[]{caption2} 
\renewcommand{\captionlabeldelim}{}
\setlength\parindent{0pt}
\begin{document}
\title{\huge{\bf{习题3.4.15}}} \author{\small{叶卢庆\footnote{叶卢庆(1992---),男,杭州师范大学理学院数学与应用数学专业本科在读,E-mail:yeluqingmathematics@gmail.com}}}
\maketitle
\begin{exercise}
设三条直线$l_1,l_2,l_3$两两异面,且平行于同一平面,证明:与$l_1,l_2,l_3$
都相交的直线组成的曲面是马鞍面.
\end{exercise}
\begin{proof}[\textbf{证明}]
设直线$l_1$的方程是
$$
\frac{x}{1}=\frac{y}{0}=\frac{z}{0},
$$
直线$l_2$的方程是
$$
\frac{x}{a}=\frac{y}{1}=\frac{z-p}{0},
$$
其中$a\neq 1$.直线$l_3$的方程是
$$
\frac{x}{a'}=\frac{y}{1}=\frac{z-p'}{0},
$$
设与$l_1,l_2,l_3$都相交的直线方程为
$$
\frac{x-q_1}{t_1}=\frac{y-q_2}{t_2}=\frac{z-q_3}{t_3},
$$
则我们有
\begin{align*}
  \begin{vmatrix}
    q_1&q_2&q_3\\
1&0&0\\
t_1&t_2&t_3
  \end{vmatrix}=0,
\end{align*}
即
\begin{equation}\label{eq:1}
q_2t_3=q_3t_2.
\end{equation}
\begin{align*}
  \begin{vmatrix}
      q_1&q_2&q_3-p\\
a&1&0\\
t_1&t_2&t_3
  \end{vmatrix}=0,
\end{align*}
结合方程\eqref{eq:1},即
\begin{equation}\label{eq:2}
t_1q_3-t_1p+at_2p-q_1t_3=0.\iff t_1q_3-q_1t_3=t_1p-at_2p.
\end{equation}
\begin{align*}
  \begin{vmatrix}
    q_1&q_2&q_3-p'\\
a'&1&0\\
t_1&t_2&t_3
  \end{vmatrix}=0.
\end{align*}
结合方程\eqref{eq:1},即
\begin{equation}
  \label{eq:3}
  t_1q_3-t_1p'+a't_2p'-q_1t_3=0.\iff t_1q_3-q_1t_3=t_1p'-a't_2p'.
\end{equation}
由方程\eqref{eq:2}和方程\eqref{eq:3}可得
\begin{equation}
  \label{eq:4}
  t_1=\frac{at_2p-a't_2p'}{p-p'}.
\end{equation}
将方程\eqref{eq:4}代入方程\eqref{eq:2},可得
\begin{equation}
  \label{eq:5}
  q_3 \frac{at_2p-a't_2p'}{p-p'}-q_1t_3=p\frac{at_2p-a't_2p'}{p-p'} -at_2p.
\end{equation}
将方程\eqref{eq:5}两边同时乘以$q_3$,可得
\begin{equation}
  \label{eq:6}
  q_3^2\frac{at_2p-a't_2p'}{p-p'}-q_1q_3t_3=p \frac{at_2q_3p-a't_2q_3p'}{p-p'}-at_2q_3p.
\end{equation}
将方程\eqref{eq:1}代入方程\eqref{eq:6},可得
\begin{equation}
  \label{eq:7}
    q_3q_{2}\frac{apt_{3}-a'p't_{3}}{p-p'}-q_1q_3t_3=p \frac{aq_{2}t_{3}p-a'q_{2}t_{3}p'}{p-p'}-aq_{2}t_3p.
\end{equation}
当$t_3\neq 0$时,在方程\eqref{eq:7}两边同时除以$t_3$,可得
\begin{equation}
  \label{eq:8}
      q_3q_{2}\frac{ap-a'p'}{p-p'}-q_1q_3=p \frac{aq_{2}p-a'q_{2}p'}{p-p'}-aq_{2}p.
\end{equation}
令 $\frac{ap-a'p'}{p-p'}=-m,ap=n,mp-n=r$,则方程\eqref{eq:8}即
\begin{equation}
  \label{eq:9}
  q_1q_3+mq_{2}q_{3}+rq_2=0.
\end{equation}
这是个双曲抛物面(具体证明放在以后).
\end{proof}
\end{document}
