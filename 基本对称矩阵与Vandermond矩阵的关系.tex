\documentclass[a4paper]{article} 
\usepackage{amsmath,amsfonts,bm}
\usepackage{hyperref}
\usepackage{amsthm} 
\usepackage{geometry}
\usepackage{amssymb,epigraph}
\usepackage{pstricks-add}
\usepackage{framed,mdframed}
\usepackage{graphicx,color} 
\usepackage{mathrsfs,xcolor} 
\usepackage[all]{xy}
\usepackage{fancybox} 
% \usepackage{CJKutf8}
\usepackage{xeCJK}
\newtheorem{theorem}{定理}
\newtheorem{lemma}{引理}
\newtheorem{corollary}{推论}
\newtheorem*{exercise}{习题}
\newtheorem{example}{例}
\setlength{\epigraphwidth}{0.6\textwidth}
\geometry{left=2.5cm,right=2.5cm,top=2.5cm,bottom=2.5cm}
\setCJKmainfont[BoldFont=Adobe Heiti Std R]{Adobe Song Std L}
\renewcommand{\today}{\number\year 年 \number\month 月 \number\day 日}
\newcommand{\D}{\displaystyle}
\newcommand{\ds}{\displaystyle} \renewcommand{\ni}{\noindent}
\newcommand{\pa}{\partial} \newcommand{\Om}{\Omega}
\newcommand{\om}{\omega} \newcommand{\sik}{\sum_{i=1}^k}
\newcommand{\vov}{\Vert\omega\Vert} \newcommand{\Umy}{U_{\mu_i,y^i}}
\newcommand{\lamns}{\lambda_n^{^{\scriptstyle\sigma}}}
\newcommand{\chiomn}{\chi_{_{\Omega_n}}}
\newcommand{\ullim}{\underline{\lim}} \newcommand{\bsy}{\boldsymbol}
\newcommand{\mvb}{\mathversion{bold}} \newcommand{\la}{\lambda}
\newcommand{\La}{\Lambda} \newcommand{\va}{\varepsilon}
\newcommand{\be}{\beta} \newcommand{\al}{\alpha}
\newcommand{\dis}{\displaystyle} \newcommand{\R}{{\mathbb R}}
\newcommand{\N}{{\mathbb N}} \newcommand{\cF}{{\mathcal F}}
\newcommand{\gB}{{\mathfrak B}} \newcommand{\eps}{\epsilon}
\renewcommand\refname{参考文献}
\begin{document}
\title{\huge{\bf{基本对称矩阵与Vandermond矩阵的关系}}}
\author{\small{叶卢庆\footnote{叶卢庆(1992---),男,杭州师范大学理学院数
      学与应用数学专业本科在读,E-mail:h5411167@gmail.com}}\\{\small{杭
      州师范大学理学院,浙江~杭州~310036}}}
\maketitle
\noindent 我们把行列式
\begin{equation}
  \label{eq:1}
\det  \begin{bmatrix}
    \frac{\pa \sigma_1}{\pa r_1}&\frac{\pa \sigma_1}{\pa
      r_2}&\cdots&\frac{\pa\sigma_1}{\pa r_n}\\
    \frac{\pa \sigma_2}{\pa r_1}&\frac{\pa \sigma_2}{\pa
      r_2}&\cdots&\frac{\pa\sigma_2}{\pa r_n}\\
    \vdots&\vdots&\cdots&\vdots\\
    \frac{\pa \sigma_n}{\pa r_1}&\frac{\pa \sigma_n}{\pa
      r_2}&\cdots&\frac{\pa\sigma_n}{\pa r_n}
  \end{bmatrix}=\prod_{1\leq i<j\leq n} (r_i-r_j)
\end{equation}
叫基本对称行列式,其中 $\sigma_1,\cdots,\sigma_n$ 为关
于$r_1,\cdots,r_n$ 的 $n$ 个基本对称多项式,也即,$$
  \begin{cases}
    \sigma_1=r_1+r_2+\cdots+r_n,\\
\sigma_2=r_1r_2+r_1r_3+\cdots+r_{n-1}r_n,\\
\sigma_3=r_1r_2r_3+r_1r_2r_4+\cdots+r_{n-2}r_{n-1}r_n,\\
\vdots\\
\sigma_n=r_1r_2\cdots r_n.
  \end{cases}
$$
即,对于两两不等的 $p_1,\cdots,p_i$,
$$
\sigma_i=\sum_{1\leq p_1<p_2<\cdots<p_i\leq n}r_{p_1}r_{p_2}\cdots r_{p_i}.
$$
我们发现,基本对称行列式和Vandermond 行列式
\begin{equation}
  \label{eq:2}
\det\begin{bmatrix}
r_1^{n-1}&r_2^{n-1}&\cdots&r_n^{n-1}\\
\vdots&\vdots&\cdots&\vdots\\
r_1&r_2&\cdots&r_n\\
  1&1&\cdots&1\\
\end{bmatrix}=\prod_{1\leq i<j\leq n} (r_i-r_j)
\end{equation}
的值相等.因此我们来寻找矩阵
\begin{equation}
  \label{eq:3}
   \begin{bmatrix}
    \frac{\pa \sigma_1}{\pa r_1}&\frac{\pa \sigma_1}{\pa
      r_2}&\cdots&\frac{\pa\sigma_1}{\pa r_n}\\
    \frac{\pa \sigma_2}{\pa r_1}&\frac{\pa \sigma_2}{\pa
      r_2}&\cdots&\frac{\pa\sigma_2}{\pa r_n}\\
    \vdots&\vdots&\cdots&\vdots\\
    \frac{\pa \sigma_n}{\pa r_1}&\frac{\pa \sigma_n}{\pa
      r_2}&\cdots&\frac{\pa\sigma_n}{\pa r_n}
  \end{bmatrix}
\end{equation}
和矩阵
\begin{equation}
  \label{eq:4}
  \begin{bmatrix}
r_1^{n-1}&r_2^{n-1}&\cdots&r_n^{n-1}\\
\vdots&\vdots&\cdots&\vdots\\
r_1&r_2&\cdots&r_n\\
  1&1&\cdots&1\\
\end{bmatrix}
\end{equation}
的关系.我们先来考察简单的情形,当 $n=2$ 时,我们尝试发现
\begin{equation}
  \label{eq:5}
  \begin{bmatrix}
    1&1\\
r_2&r_1\\
  \end{bmatrix}
\end{equation}
和
\begin{equation}
  \label{eq:6}
\begin{bmatrix}
r_1&r_2\\
    1&1\\
\end{bmatrix}
\end{equation}
的关系.设
$$
\begin{bmatrix}
  r_{1}&r_{2}\\
1&1
\end{bmatrix}=\begin{bmatrix}
  a_{11}&a_{12}\\
a_{21}&a_{22}
\end{bmatrix}\begin{bmatrix}
  1&1\\
r_{2}&r_{1}
\end{bmatrix},
$$
则
$$
\begin{bmatrix}
  a_{11}&a_{12}\\
a_{21}&a_{22}
\end{bmatrix}=\begin{bmatrix}
  r_1+r_2&-1\\
1&0
\end{bmatrix}.
$$
当 $n=3$ 时,我们尝试发现
\begin{equation}
  \label{eq:11.01}
  \begin{bmatrix}
      1&1&1\\
r_2+r_3&r_1+r_3&r_1+r_2\\
r_2r_3&r_1r_3&r_1r_2
  \end{bmatrix}
\end{equation}
和
\begin{equation}
  \label{eq:11.02}
  \begin{bmatrix}
      r_1^2&r_2^2&r_3^2\\
r_1&r_2&r_3\\
1&1&1
  \end{bmatrix}
\end{equation}
的关系.设
\begin{equation}
  \label{eq:11.06}
  \begin{bmatrix}
      r_1^2&r_2^2&r_3^2\\
r_1&r_2&r_3\\
1&1&1
  \end{bmatrix}=\begin{bmatrix}
    a_{11}&a_{12}&a_{13}\\
a_{21}&a_{22}&a_{23}\\
a_{31}&a_{32}&a_{33}
  \end{bmatrix}\begin{bmatrix}
          1&1&1\\
r_2+r_3&r_1+r_3&r_1+r_2\\
r_2r_3&r_1r_3&r_1r_2
  \end{bmatrix},
\end{equation}
则
$$
\begin{bmatrix}
  a_{11}&a_{12}&a_{13}\\
a_{21}&a_{22}&a_{23}\\
a_{31}&a_{32}&a_{33}
\end{bmatrix}=\begin{bmatrix}
 r_1^2+r_2^2+r_3^2+r_1r_2+r_2r_3+r_3r_1&-(r_1+r_2+r_3)&1\\
r_1+r_2+r_3&-1&0\\
1&0&0
\end{bmatrix}.
$$

\end{document}








