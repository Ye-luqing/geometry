\documentclass[a4paper]{article}
\usepackage{amsmath,amsfonts,amsthm,amssymb}
\usepackage{bm}
\usepackage{hyperref}
\usepackage{geometry}
\usepackage{yhmath}
\usepackage{pstricks-add}
\usepackage{framed,mdframed}
\usepackage{graphicx,color} 
\usepackage{mathrsfs,xcolor} 
\usepackage[all]{xy}
\usepackage{fancybox} 
\usepackage{xeCJK}
\newtheorem{theo}{定理}
\newtheorem{exam}{例}
\newmdtheoremenv{lemma}{引理}
\newmdtheoremenv{corollary}{推论}
\newmdtheoremenv{exercise}{习题}
\newenvironment{theorem}
{\bigskip\begin{mdframed}\begin{theo}}
    {\end{theo}\end{mdframed}\bigskip}
\newenvironment{example}
{\bigskip\begin{mdframed}\begin{exam}}
    {\end{exam}\end{mdframed}\bigskip}
\geometry{left=2.5cm,right=2.5cm,top=2.5cm,bottom=2.5cm}
\setCJKmainfont[BoldFont=FZHei-B01S]{FZFangSong-Z02S}
\renewcommand{\today}{\number\year 年 \number\month 月 \number\day 日}
\newcommand{\D}{\displaystyle}\newcommand{\ri}{\Rightarrow}
\newcommand{\ds}{\displaystyle} \renewcommand{\ni}{\noindent}
\newcommand{\pa}{\partial} \newcommand{\Om}{\Omega}
\newcommand{\om}{\omega} \newcommand{\sik}{\sum_{i=1}^k}
\newcommand{\vov}{\Vert\omega\Vert} \newcommand{\Umy}{U_{\mu_i,y^i}}
\newcommand{\lamns}{\lambda_n^{^{\scriptstyle\sigma}}}
\newcommand{\chiomn}{\chi_{_{\Omega_n}}}
\newcommand{\ullim}{\underline{\lim}} \newcommand{\bsy}{\boldsymbol}
\newcommand{\mvb}{\mathversion{bold}} \newcommand{\la}{\lambda}
\newcommand{\La}{\Lambda} \newcommand{\va}{\varepsilon}
\newcommand{\ov}{\overrightarrow} \newcommand{\al}{\alpha}
\newcommand{\dis}{\displaystyle} \newcommand{\R}{{\mathbb R}}
\newcommand{\N}{{\mathbb N}} \newcommand{\cF}{{\mathcal F}}
\newcommand{\gB}{{\mathfrak B}} \newcommand{\eps}{\epsilon}
\renewcommand\refname{参考文献}\renewcommand\figurename{图}
\usepackage[]{caption2} 
\renewcommand{\captionlabeldelim}{}
\begin{document}\date{}
\title{\huge{\bf{数形结合在中学数学中的应用}}}
\maketitle

\section{引言}
\label{sec:1}


著名法国数学家拉格朗日曾经说过:“只要代数同几何分道扬镳,它们的进展就缓慢,
它们的应用就狭窄.但是当这两门学科结合时,它们就互相吸取对方的力量,然后
趋于完善."\\

我国著名数学家华罗庚曾赋诗:“数与形,本是相倚依,焉能分作两边飞.数缺形时少直
观,形少数时难入微.形数结合百般好,隔离分家万事休.切莫忘,几何代数统一体,
永远联系,切莫分离!"\\

什么是数?什么是形?这两个词汇都没有准确的数学定义,因为现代数学里的概念,归根
结底,大都基于 Cantor 所创立的集合论,至于论证方式,是基于公理化的推理,而
集合论和公理体系是没有什么数与形的区分的.而
且,据笔者所知,数形结合这种说法,大都也只出现在中学数学的教
育领域里,在真正的数学研究里,很少有人说什么“我们利用数形结合的思想”,因
为数学里其实到处都在数形结合,没必要分明地指出来.大体上来说,所谓数,就
是数量关系,是抽象的符号,能用来运算.所谓形,就是空间形式,在我们的大脑里
存在直观的图景.我们的大脑经过漫长的自然进化,能感受到直观,想象得出形状.数
形结合的巨大威力,本质上就是我们利用我们的大脑的这种先天优势,对数学进行
整体上的直观洞察,得出靠单纯的代数运算很难得到的结论,从而为我们的数学研
究指明方向.可以说,在某些方面,形可以作为数的指南针.在靠形指明了方向之后,我
们就可以再靠数来进行精确化.所以,更一般地来说,数形结合其实是我们的大脑
利用自己所熟悉的材料,来指导学习和发现自己所不熟悉的材料.\\

就中学数学而言,数形结合其实无处不在.比如,实数集合 $\mathbf{R}$ 和实数轴之间的一一
对应,就是数形结合.实数对集合 $\mathbf{R}^2$ 和平面直角坐标系之间的一
一对应,也是数形结合.我们在平面直角坐标系上画出函数的图像,仍然是数形结
合.我们把复数集合 $\mathbf{C}$ 和平面直角坐标系上的点一一对应,从而把一
个复数看作一个平面向量,还是数形结合.在几百年前,人们的心里还是不大容易
接受虚数 $a+bi$ 的,后来高斯等人把 $a+bi$ 解释为平面直角坐标系上的向量
后,人们就接受虚数了,可见图形对于人们心理上的巨大作用.甚
至在某些学数学的人的心里,理解数学在很多时候意味着找到一个几何解释,
只要找到了一个几何解释,在自己的脑海里产生直观,他们便能心安理得地接受所
学的数学知识.\\

以前的高中教材,在教授空间几何的时候会介绍所谓的“三垂线定理”,现在已经没
了,大家都喜欢用向量.利用向量的和,差,垂直,内积等概念一算,便能计算出各种
角度,长度,从而免除了作辅助线的烦恼,把空间几何题目的解决程序化,降低了学
习空间几何的难度,以至于现在的某些学生一看到这些题目,就会下意识地建立空
间直角坐标系.这样子做对于学生的发展利弊姑且不论,反正这也是数形结合在中
学数学里的一个巨大用处,是数帮助了人们解决形的问题.而整个中学的解析几何,都
是在谈怎么用数来帮助人们解决形的问题,这也是笛卡尔发明解析几何的初衷.\\


下面,我们举出几个具体的数形结合的例子.
\section{例子}
\begin{example}[一个不等式的几何证明]
我们来看不等式
\begin{equation}\label{eq:1}
\sqrt{a^2+b^2}+\sqrt{b^2+c^2}+\sqrt{c^2+a^2}\geq \sqrt{2}(a+b+c)
\end{equation}
的几何证明,其中 $a,b,c$ 为非负实数.
\end{example}
\begin{proof}[\textbf{证明}]
构造图形如下.图 \eqref{fig:1} 中凡四边形都是正方形.边长已经标记.则 
$$
|AB|+|BC|+|CD|=\sqrt{a^2+b^2}+\sqrt{b^2+c^2}+\sqrt{c^2+a^2},
$$
$$
|AD|=\sqrt{2}(a+b+c).
$$
由于平面上两点之间线段最短,因此 $|AB|+|BC|+|CD|\geq |AD|$.因此不等式成
立.\\
\begin{figure}[h]
\newrgbcolor{zzttqq}{0.6 0.2 0}
\psset{xunit=1.0cm,yunit=1.0cm,algebraic=true,dotstyle=o,dotsize=3pt 0,linewidth=0.8pt,arrowsize=3pt 2,arrowinset=0.25}
\begin{pspicture*}(-3,-6.12)(18.32,6.06)
\pspolygon[linecolor=zzttqq,fillcolor=zzttqq,fillstyle=solid,opacity=0.1](-1,-4)(1,-4)(1,-2)(-1,-2)
\pspolygon[linecolor=zzttqq,fillcolor=zzttqq,fillstyle=solid,opacity=0.1](1,-2)(5,-2)(5,2)(1,2)
\pspolygon[linecolor=zzttqq,fillcolor=zzttqq,fillstyle=solid,opacity=0.1](5,2)(6,2)(6,3)(5,3)
\pspolygon[linecolor=zzttqq,fillcolor=zzttqq,fillstyle=solid,opacity=0.1](6,3)(8,3)(8,5)(6,5)
\pspolygon[linecolor=zzttqq,fillcolor=zzttqq,fillstyle=solid,opacity=0.1](-1,-4)(8,-3.98)(7.98,5.02)(-1.02,5)
\psline[linecolor=zzttqq](-1,-4)(1,-4)
\psline[linecolor=zzttqq](1,-4)(1,-2)
\psline[linecolor=zzttqq](1,-2)(-1,-2)
\psline[linecolor=zzttqq](-1,-2)(-1,-4)
\psline[linecolor=zzttqq](1,-2)(5,-2)
\psline[linecolor=zzttqq](5,-2)(5,2)
\psline[linecolor=zzttqq](5,2)(1,2)
\psline[linecolor=zzttqq](1,2)(1,-2)
\psline[linecolor=zzttqq](5,2)(6,2)
\psline[linecolor=zzttqq](6,2)(6,3)
\psline[linecolor=zzttqq](6,3)(5,3)
\psline[linecolor=zzttqq](5,3)(5,2)
\psline[linecolor=zzttqq](6,3)(8,3)
\psline[linecolor=zzttqq](8,3)(8,5)
\psline[linecolor=zzttqq](8,5)(6,5)
\psline[linecolor=zzttqq](6,5)(6,3)
\rput[tl](-0.26,-4.2){$ a $}
\rput[tl](-1.56,-2.64){$ a $}
\rput[tl](2.72,-1.7){$ b $}
\rput[tl](0.5,0.56){$ b $}
\rput[tl](5.58,1.9){$ c $}
\rput[tl](4.78,2.8){$ c $}
\rput[tl](7.06,2.9){$ a $}
\rput[tl](5.68,4.38){$ a $}
\psline(1,-4)(5,-2)
\psline(5,-2)(6,2)
\psline(6,2)(8,3)
\psline[linecolor=zzttqq](-1,-4)(8,-3.98)
\psline[linecolor=zzttqq](8,-3.98)(7.98,5.02)
\psline[linecolor=zzttqq](7.98,5.02)(-1.02,5)
\psline[linecolor=zzttqq](-1.02,5)(-1,-4)
\psline(8,3)(1,-4)
\begin{scriptsize}
%\psdots[dotstyle=*](5,-2)
\rput[bl](5.08,-1.88){$B$}
%\psdots[dotstyle=*](6,2)
\rput[bl](6.08,2.12){$C$}
%\psdots[dotstyle=*,linecolor=darkgray](8,3)
\rput[bl](8.08,3.12){\darkgray{$D$}}
%\psdots[dotstyle=*,linecolor=darkgray](1,-4)
\rput[bl](0.6,-3.88){\darkgray{$A$}}
%\psdots[dotstyle=*,linecolor=darkgray](8,-3.98)
\rput[bl](8.08,-3.86){\darkgray{$E$}}
\end{scriptsize}
\end{pspicture*}    
  \caption{}
  \label{fig:1}
\end{figure}
\end{proof}

不等式 \eqref{eq:1} 其实也可以用柯西不等式来证明.而柯西不等式除了可以
用数学归纳法来做之外,本身其实是一个几何的不等式.

\begin{example}[柯西不等式的几何证明]
柯西不等式说,如果 $a_1,a_2,\cdots,a_n$,$b_1,b_2,\cdots,b_n$ 是实数,则
\begin{equation}
  \label{eq:2}
  (a_1^2+a_2^2+\cdots+a_n^2)(b_1^2+b_2^2+\cdots+b_n^2)\geq (a_1b_1+a_2b_2+\cdots+a_nb_n)^2.
\end{equation}
等号成立当且仅当存在不全为 $0$ 的实数 $\lambda_1,\lambda_2$,使得 $\lambda_1(a_1,\cdots,a_n)+\lambda_2(b_1,\cdots,b_n)=(0,\cdots,0)$.
\end{example}
\begin{proof}[\textbf{证明}]
设 $\ov{OA}=(a_1,\cdots,a_n)$ 和 $\ov{OB}=(b_1,\cdots,b_n)$ 为实数域上的 $n$ 维向量空
间 $\mathbf{R}^n$ 中的任意两个向量,其中 $O$ 为坐标原点 $(0,\cdots,0)$.则结合勾股定理,向量 $\ov{OA}$ 的长度与向量
$\ov{OB}$ 的长度可以归纳地定义为
$$
|\ov{OA}|=\sqrt{a_1^2+\cdots+a_n^2},|\ov{OB}|=\sqrt{b_1^2+\cdots+b_n^2}.
$$
如图 \eqref{fig:2} 所示,根据向量内积的定义和内积的分配律,可得
\begin{align*}
|\ov{AB}|^{2}=\ov{AB}\cdot \ov{AB}&=(\ov{OB}-\ov{OA})\cdot
(\ov{OB}-\ov{OA})\\&=\ov{OB}\cdot(\ov{OB}-\ov{OA})-\ov{OA}\cdot(\ov{OB}-\ov{OA})\\&=\ov{OB}^2-2\ov{OB}\cdot
\ov{OA}+\ov{OA}^2\\&=|\ov{OB}|^2-2\ov{OB}\cdot \ov{OA}+|\ov{OA}|^2
\end{align*}
由此可得
\begin{align*}
  \ov{OB}\cdot \ov{OA}&=\frac{|\ov{OB}|^{2}+|\ov{OA}|^{2}-|\ov{AB}|^2}{2}.
\end{align*}
这正是余弦定理.由于
\begin{align*}
  \frac{|\ov{OB}|^{2}+|\ov{OA}|^{2}-|\ov{AB}|^2}{2}&=a_1b_1+\cdots+a_nb_n,
\end{align*}
因此可得 $\ov{OB}\cdot \ov{OA}=a_1b_1+\cdots+a_nb_n$.于是,
$$
|\ov{OB}|^{2}|\ov{OA}|^{2}\cos^{2} \alpha=(a_1b_1+\cdots+a_nb_n)^2.
$$
由于 $\cos\alpha^2\leq 1$,因此便可得柯西不等式 \eqref{eq:2}.且等号成立
当且仅当 $\cos\alpha=\pm 1$,当且仅当 $\alpha=0$ 或者 $\pi$,当且仅当向量
$\ov{OA}$ 与向量 $\ov{OB}$ 线性相关.
\begin{figure}[h]
\newrgbcolor{qqwuqq}{0 0.39 0}
\psset{xunit=1.0cm,yunit=1.0cm,algebraic=true,dotstyle=o,dotsize=3pt 0,linewidth=0.8pt,arrowsize=3pt 2,arrowinset=0.25}
\begin{pspicture*}(-2.72,-4.68)(20.6,7.5)
\psline(1.98,2.86)(11.64,4.34)
\psline(11.64,4.34)(5.78,-1.2)
\psline(5.78,-1.2)(1.98,2.86)
\pscustom[linecolor=qqwuqq,fillcolor=qqwuqq,fillstyle=solid,opacity=0.1]{\parametricplot{0.7573353569642691}{2.32312766697659}{0.6*cos(t)+5.78|0.6*sin(t)+-1.2}\lineto(5.78,-1.2)\closepath}
\pscustom[linecolor=qqwuqq,fillcolor=qqwuqq,fillstyle=solid,opacity=0.1]{\parametricplot{-0.8184649866132031}{0.1520269541174477}{0.6*cos(t)+1.98|0.6*sin(t)+2.86}\lineto(1.98,2.86)\closepath}
\pscustom[linecolor=qqwuqq,fillcolor=qqwuqq,fillstyle=solid,opacity=0.1]{\parametricplot{-2.9895656994723456}{-2.384257296625524}{0.6*cos(t)+11.64|0.6*sin(t)+4.34}\lineto(11.64,4.34)\closepath}
\begin{scriptsize}
\psdots[dotstyle=*,linecolor=blue](5.78,-1.2)
\rput[bl](5.64,-1.62){\blue{$O$}}
\psdots[dotstyle=*,linecolor=blue](1.98,2.86)
\rput[bl](1.7,2.98){\blue{$A$}}
\psdots[dotstyle=*,linecolor=blue](11.64,4.34)
\rput[bl](11.72,4.46){\blue{$B$}}
\rput[bl](5.72,-0.94){\qqwuqq{$\alpha$}}
\rput[bl](2.24,2.66){\qqwuqq{$\beta$}}
\rput[bl](11.26,4.1){\qqwuqq{$\gamma$}}
\end{scriptsize}
\end{pspicture*}
  \caption{}
  \label{fig:2}
\end{figure}
\end{proof}
\end{document}








