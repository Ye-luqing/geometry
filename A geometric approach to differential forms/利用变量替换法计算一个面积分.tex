\documentclass[a4paper]{article}
\usepackage{amsmath,amsfonts,amsthm,amssymb}
\usepackage{bm}
\usepackage{hyperref}
\usepackage{geometry}
\usepackage{yhmath}
\usepackage{pstricks-add}
\usepackage{framed,mdframed}
\usepackage{graphicx,color} 
\usepackage{mathrsfs,xcolor} 
\usepackage[all]{xy}
\usepackage{fancybox} 
\usepackage{xeCJK}
\newtheorem*{theo}{定理}
\newtheorem*{exe}{题目}
\newtheorem*{rem}{评论}
\newmdtheoremenv{lemma}{引理}
\newmdtheoremenv{corollary}{推论}
\newtheorem*{exa}{例}
\newenvironment{theorem}
{\bigskip\begin{mdframed}\begin{theo}}
    {\end{theo}\end{mdframed}\bigskip}
\newenvironment{exercise}
{\bigskip\begin{mdframed}\begin{exe}}
    {\end{exe}\end{mdframed}\bigskip}
\newenvironment{example}
{\bigskip\begin{mdframed}\begin{exa}}
    {\end{exa}\end{mdframed}\bigskip}
\geometry{left=2.5cm,right=2.5cm,top=2.5cm,bottom=2.5cm}
\setCJKmainfont[BoldFont=SimHei]{SimSun}
\renewcommand{\today}{\number\year 年 \number\month 月 \number\day 日}
\newcommand{\D}{\displaystyle}\newcommand{\ri}{\Rightarrow}
\newcommand{\ds}{\displaystyle} \renewcommand{\ni}{\noindent}
\newcommand{\ov}{\overrightarrow}
\newcommand{\pa}{\partial} \newcommand{\Om}{\Omega}
\newcommand{\om}{\omega} \newcommand{\sik}{\sum_{i=1}^k}
\newcommand{\vov}{\Vert\omega\Vert} \newcommand{\Umy}{U_{\mu_i,y^i}}
\newcommand{\lamns}{\lambda_n^{^{\scriptstyle\sigma}}}
\newcommand{\chiomn}{\chi_{_{\Omega_n}}}
\newcommand{\ullim}{\underline{\lim}} \newcommand{\bsy}{\boldsymbol}
\newcommand{\mvb}{\mathversion{bold}} \newcommand{\la}{\lambda}
\newcommand{\La}{\Lambda} \newcommand{\va}{\varepsilon}
\newcommand{\be}{\beta} \newcommand{\al}{\alpha}
\newcommand{\dis}{\displaystyle} \newcommand{\R}{{\mathbb R}}
\newcommand{\N}{{\mathbb N}} \newcommand{\cF}{{\mathcal F}}
\newcommand{\gB}{{\mathfrak B}} \newcommand{\eps}{\epsilon}
\renewcommand\refname{参考文献}\renewcommand\figurename{图}
\usepackage[]{caption2} 
\renewcommand{\captionlabeldelim}{}
\setlength\parindent{0pt}
\begin{document}
\title{\huge{\bf{利用变量替换法计算一个面积分}}} \author{\small{叶卢庆\footnote{叶卢庆(1992---),男,杭州师范大学理学院数学与应用数学专业本科在读,E-mail:yeluqingmathematics@gmail.com}}}
\maketitle
\begin{exercise}
计算面积分
$$
\int_S f(x,y,z)dS,
$$
其中曲面$S$代表球面$x^2+y^2+z^2=1$的上半面,$f(x,y,z)=z^2$.
\end{exercise}
\begin{proof}[\textbf{解}]
  变量替换.令$x=r\cos\theta,y=r\sin\theta,z=\sqrt{1-r^2}$,其中
  $\theta\in [0,2\pi)$,$0\leq r\leq 1$.则曲面$S$上的任意一点都可以表示
  为$P(\theta,r)=(r\cos\theta,r\sin\theta,\sqrt{1-r^2})$.且
  \begin{align*}
    \int_S f(x,y,z)dS&=\int_S (1-r^2)dS\\&=\int_{0}^{1}\int_{0}^{2\pi}F(\theta,r)d\theta dr.
  \end{align*}
其中$T=[0,2\pi)\times [0,1]$ 是$\theta-O-r$平面直角坐标系上的一个长方
形区域.令
$$
F(\theta,r)d\theta dr=(1-r^2)dS.
$$
则
$$
F(\theta,r)=(1-r^2) \frac{dS}{d\theta dr}.
$$
易得
$$
\frac{\pa P}{\pa \theta}=(-r\sin\theta,r\cos\theta,0),\frac{\pa P}{\pa
r}=(\cos\theta,\sin\theta,\frac{-r}{\sqrt{1-r^2}}).
$$
因此
$$
\left|\frac{\pa P}{\pa\theta}\times \frac{\pa P}{\pa r}\right|=\frac{r}{\sqrt{1-r^2}}.
$$
因此
$$
\frac{dS}{d\theta dr}=\frac{r}{\sqrt{1-r^2}}.
$$
可见,$F(\theta,r)=r \sqrt{1-r^2}$.于是我们之用计算
$$
\int_0^1\int_0^{2\pi}r \sqrt{1-r^2}d\theta dr=2\pi\int_{0}^{1}r
\sqrt{1-r^2}dr=\frac{2\pi}{3}.
$$
完毕.
\end{proof}
\end{document}
