\documentclass[a4paper]{article}
\usepackage{amsmath,amsfonts,amsthm,amssymb}
\usepackage{bm}
\usepackage{hyperref}
\usepackage{geometry}
\usepackage{yhmath}
\usepackage{pstricks-add}
\usepackage{framed,mdframed}
\usepackage{graphicx,color} 
\usepackage{mathrsfs,xcolor} 
\usepackage[all]{xy}
\usepackage{fancybox} 
\usepackage{xeCJK}
\newtheorem*{theo}{定理}
\newtheorem*{exe}{题目}
\newtheorem*{rem}{评论}
\newtheorem*{lemma}{引理}
\newtheorem*{coro}{推论}
\newtheorem*{exa}{例}
\newenvironment{corollary}
{\bigskip\begin{mdframed}\begin{coro}}
    {\end{coro}\end{mdframed}\bigskip}
\newenvironment{theorem}
{\bigskip\begin{mdframed}\begin{theo}}
    {\end{theo}\end{mdframed}\bigskip}
\newenvironment{exercise}
{\bigskip\begin{mdframed}\begin{exe}}
    {\end{exe}\end{mdframed}\bigskip}
\newenvironment{example}
{\bigskip\begin{mdframed}\begin{exa}}
    {\end{exa}\end{mdframed}\bigskip}
\newenvironment{remark}
{\bigskip\begin{mdframed}\begin{rem}}
    {\end{rem}\end{mdframed}\bigskip}
\geometry{left=2.5cm,right=2.5cm,top=2.5cm,bottom=2.5cm}
\setCJKmainfont[BoldFont=SimHei]{SimSun}
\renewcommand{\today}{\number\year 年 \number\month 月 \number\day 日}
\newcommand{\D}{\displaystyle}\newcommand{\ri}{\Rightarrow}
\newcommand{\ds}{\displaystyle} \renewcommand{\ni}{\noindent}
\newcommand{\ov}{\overrightarrow}
\newcommand{\pa}{\partial} \newcommand{\Om}{\Omega}
\newcommand{\om}{\omega} \newcommand{\sik}{\sum_{i=1}^k}
\newcommand{\vov}{\Vert\omega\Vert} \newcommand{\Umy}{U_{\mu_i,y^i}}
\newcommand{\lamns}{\lambda_n^{^{\scriptstyle\sigma}}}
\newcommand{\chiomn}{\chi_{_{\Omega_n}}}
\newcommand{\ullim}{\underline{\lim}} \newcommand{\bsy}{\boldsymbol}
\newcommand{\mvb}{\mathversion{bold}} \newcommand{\la}{\lambda}
\newcommand{\La}{\Lambda} \newcommand{\va}{\varepsilon}
\newcommand{\be}{\beta} \newcommand{\al}{\alpha}
\newcommand{\dis}{\displaystyle} \newcommand{\R}{{\mathbb R}}
\newcommand{\N}{{\mathbb N}} \newcommand{\cF}{{\mathcal F}}
\newcommand{\gB}{{\mathfrak B}} \newcommand{\eps}{\epsilon}
\renewcommand\refname{参考文献}\renewcommand\figurename{图}
\usepackage[]{caption2} 
\renewcommand{\captionlabeldelim}{}
\setlength\parindent{0pt}
\begin{document}
\title{\huge{\bf{$\mathbf{R}^4$中二次形式可分解为一次形式楔积的必
      要条件}}} \author{\small{叶卢庆\footnote{叶卢庆(1992---),男,杭州师范大学理学院数学与应用数学专业本科在读,E-mail:yeluqingmathematics@gmail.com}}}
\maketitle
我们知道,$\mathbf{R}^4$中的任意二次形式都可以写作
\begin{align*}
a_1dx\wedge dy+a_2dy\wedge dz+a_3dz\wedge dw+a_4dw\wedge dx&=
dx\wedge (a_{1}dy-a_{4}dw)+dz\wedge (a_3dw-a_2dy).
\end{align*}
可见,$\mathbf{R}^4$里的任意二次形式$\omega$都可以写为两个二次形式
$\omega_1$与$\omega_2$的和,$\omega=\omega_1+\omega_2$.其中这两
个二次形式都是都是一次形式的楔积,也即
$$
\omega_1=\nu_1\wedge \nu_2,\omega_2=\xi_1\wedge \xi_2.
$$
其中$\nu_1,\nu_2,\xi_1,\xi_2$都是$\mathbf{R}^4$中的一次形式.
我们来证明如下结论:
\begin{theorem}
若$\langle \nu_1\rangle$,$\langle\nu_2\rangle$张成的平面和$\langle
\xi_1\rangle$,$\langle\xi_2\rangle$张成的平面仅有一个交点,则二次形式
$\omega$不能表达成一次形式的楔积.
\end{theorem}
\begin{proof}[\textbf{证明}]
  $\langle \nu_1\rangle$,$\langle\nu_2\rangle$张成的平面和$\langle
\xi_1\rangle$,$\langle\xi_2\rangle$张成的平面仅有一个交点,说明$\det
(\langle\nu_1\rangle,\langle\nu_2\rangle,\langle\xi_1\rangle,\langle\xi_2\rangle)\neq
0$.于是,对于任意$\mathbf{R}^4$中线性无关的向量$V_1,V_2,V_3,V_{4}$
$$
  \omega\wedge\omega(V_1,V_2,V_3,V_{4})=(\omega_1+\omega_2)\wedge(\omega_1+\omega_2)(V_1,V_2,V_{3},V_{4})
=2\nu_1\wedge\nu_2\wedge\xi_1\wedge\xi_2(V_1,V_2,V_3,V_{4})\neq 0.
$$
假设$\omega=\delta_1\wedge\delta_2$,其中$\delta_1,\delta_2$是
$\mathbf{R}^4$中的两个一次形式,则应该有
$\omega\wedge\omega=\delta_1\wedge \delta_2\wedge \delta_1\wedge \delta_2=0$.矛盾.说
明假设错误.于是二次形式$\omega$不能表达成一次形式的楔积.
\end{proof}
根据这个定理,我们有推论:
\begin{corollary}
若$\mathbf{R}^{4}$中的二次形式$\omega$能表达成
一次形式的楔积,则$\langle\nu_1\rangle$,$\langle\nu_2\rangle$张成的平面和
$\langle\xi_1\rangle$,$\langle\xi_2\rangle$张成的平面或者只有一条交线,或者平
行,或者重合.
\end{corollary}
\end{document}
