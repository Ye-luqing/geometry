\documentclass[a4paper]{article}
\usepackage{amsmath,amsfonts,amsthm,amssymb}
\usepackage{bm}
\usepackage{hyperref}
\usepackage{geometry}
\usepackage{yhmath}
\usepackage{pstricks-add}
\usepackage{framed,mdframed}
\usepackage{graphicx,color} 
\usepackage{mathrsfs,xcolor} 
\usepackage[all]{xy}
\usepackage{fancybox} 
\usepackage{xeCJK}
\newtheorem*{theo}{定理}
\newtheorem*{exe}{Exercise}
\newtheorem*{rem}{评论}
\newmdtheoremenv{lemma}{引理}
\newmdtheoremenv{corollary}{推论}
\newtheorem*{exa}{例}
\newenvironment{theorem}
{\bigskip\begin{mdframed}\begin{theo}}
    {\end{theo}\end{mdframed}\bigskip}
\newenvironment{exercise}
{\bigskip\begin{mdframed}\begin{exe}}
    {\end{exe}\end{mdframed}\bigskip}
\newenvironment{example}
{\bigskip\begin{mdframed}\begin{exa}}
    {\end{exa}\end{mdframed}\bigskip}
\geometry{left=2.5cm,right=2.5cm,top=2.5cm,bottom=2.5cm}
\setCJKmainfont[BoldFont=SimHei]{SimSun}
\newcommand{\D}{\displaystyle}\newcommand{\ri}{\Rightarrow}
\newcommand{\ds}{\displaystyle} \renewcommand{\ni}{\noindent}
\newcommand{\ov}{\overrightarrow}
\newcommand{\pa}{\partial} \newcommand{\Om}{\Omega}
\newcommand{\om}{\omega} \newcommand{\sik}{\sum_{i=1}^k}
\newcommand{\vov}{\Vert\omega\Vert} \newcommand{\Umy}{U_{\mu_i,y^i}}
\newcommand{\lamns}{\lambda_n^{^{\scriptstyle\sigma}}}
\newcommand{\chiomn}{\chi_{_{\Omega_n}}}
\newcommand{\ullim}{\underline{\lim}} \newcommand{\bsy}{\boldsymbol}
\newcommand{\mvb}{\mathversion{bold}} \newcommand{\la}{\lambda}
\newcommand{\La}{\Lambda} \newcommand{\va}{\varepsilon}
\newcommand{\be}{\beta} \newcommand{\al}{\alpha}
\newcommand{\dis}{\displaystyle} \newcommand{\R}{{\mathbb R}}
\newcommand{\N}{{\mathbb N}} \newcommand{\cF}{{\mathcal F}}
\newcommand{\gB}{{\mathfrak B}} \newcommand{\eps}{\epsilon}
\renewcommand\refname{参考文献}\renewcommand\figurename{图}
\usepackage[]{caption2} 
\renewcommand{\captionlabeldelim}{}
\setlength\parindent{0pt}
\begin{document}
\title{\huge{\bf{Exercise2.6}}} \author{\small{Luqing Ye\footnote{叶卢庆(1992---),男,杭州师范大学理学院数学与应用数学专业本科在读,E-mail:yeluqingmathematics@gmail.com}}}
\maketitle
\begin{exercise}
Let 
$$
D(x,y)=
\begin{vmatrix}
\frac{\pa^2f}{\pa x^2}&\frac{\pa^2f}{\pa x\pa y}\\
\frac{\pa^2f}{\pa y\pa x}& \frac{\pa^2f}{\pa y^2}
\end{vmatrix}.
$$
If,for some point $(x_0,y_0)$,you know $D(x_0,y_0)>0$,then show that
the signs of $\frac{\pa^2f}{\pa x^2}(x_0,y_0)$ and $\frac{\pa^2f}{\pa
  y^2}(x_0,y_0)$ are the same.
\end{exercise}
\begin{proof}
Let $(a,b)$ be a vector,and $t$ be a real number.Let $g(t)=f(x_0+ta,y_0+tb)$.According to Taylor's formula,
$$
g(t)=g(0)+\frac{g'(0)}{1!}t+\frac{g''(0)}{2!}t^2+\cdots
$$
It is easy to verify that 
$$
g'(h)=a\frac{\pa f}{\pa x}(x_0+ha,y_0+hb)+b \frac{\pa f}{\pa y}(x_0+ha,y_0+hb),
$$
and
\begin{align*}
  g''(h)&=a\left[a\frac{\pa }{\pa x}\frac{\pa f}{\pa x}(x_0+ha,y_{0}+hb)+b
  \frac{\pa}{\pa y}\frac{\pa f}{\pa
    x}(x_0+ha,y_{0}+hb)\right]\\&+b\left[a\frac{\pa}{\pa x}\frac{\pa
    f}{\pa y}(x_{0}+ha,y_{0}+hb)+b \frac{\pa}{\pa y} \frac{\pa f}{\pa
    y}(x_{0}+ha,y_{0}+hb)\right]
\\&=a^2 \frac{\pa^2f}{\pa
  x^2}(x_0+ha,y_0+hb)+2ab\frac{\pa^2f}{\pa y\pa
    x}(x_{0}+ha,y_{0}+hb)+b^2 \frac{\pa^2f}{\pa
  y^2}(x_0+ha,y_0+hb).
\end{align*}
Let $b=1$,then
$$
g''(0)=a^2 \frac{\pa^2f}{\pa
  x^2}(x_0,y_0)+2a\frac{\pa^2f}{\pa y\pa
    x}(x_{0},y_{0})+\frac{\pa^2f}{\pa
  y^2}(x_0,y_0)
$$
Regard $g''(0)$ as a quadratic equation of $a$,then $\Delta<0$.So  $\frac{\pa^2f}{\pa
  x^2}(x_0,y_0)\frac{\pa^2f}{\pa
  y^2}(x_0,y_0)>0$.
\end{proof}
\end{document}
