\documentclass[a4paper]{article}
\usepackage{amsmath,amsfonts,amsthm,amssymb}
\usepackage{bm}
\usepackage{hyperref}
\usepackage{geometry}
\usepackage{yhmath}
\usepackage{pstricks-add}
\usepackage{framed,mdframed}
\usepackage{graphicx,color} 
\usepackage{mathrsfs,xcolor} 
\usepackage[all]{xy}
\usepackage{fancybox} 
\usepackage{xeCJK}
\newtheorem*{theo}{定理}
\newtheorem*{exe}{题目}
\newtheorem*{rem}{评论}
\newtheorem*{lemma}{引理}
\newtheorem*{coro}{推论}
\newtheorem*{exa}{例}
\newenvironment{corollary}
{\bigskip\begin{mdframed}\begin{coro}}
    {\end{coro}\end{mdframed}\bigskip}
\newenvironment{theorem}
{\bigskip\begin{mdframed}\begin{theo}}
    {\end{theo}\end{mdframed}\bigskip}
\newenvironment{exercise}
{\bigskip\begin{mdframed}\begin{exe}}
    {\end{exe}\end{mdframed}\bigskip}
\newenvironment{example}
{\bigskip\begin{mdframed}\begin{exa}}
    {\end{exa}\end{mdframed}\bigskip}
\newenvironment{remark}
{\bigskip\begin{mdframed}\begin{rem}}
    {\end{rem}\end{mdframed}\bigskip}
\geometry{left=2.5cm,right=2.5cm,top=2.5cm,bottom=2.5cm}
\setCJKmainfont[BoldFont=SimHei]{SimSun}
\renewcommand{\today}{\number\year 年 \number\month 月 \number\day 日}
\newcommand{\D}{\displaystyle}\newcommand{\ri}{\Rightarrow}
\newcommand{\ds}{\displaystyle} \renewcommand{\ni}{\noindent}
\newcommand{\ov}{\overrightarrow}
\newcommand{\pa}{\partial} \newcommand{\Om}{\Omega}
\newcommand{\om}{\omega} \newcommand{\sik}{\sum_{i=1}^k}
\newcommand{\vov}{\Vert\omega\Vert} \newcommand{\Umy}{U_{\mu_i,y^i}}
\newcommand{\lamns}{\lambda_n^{^{\scriptstyle\sigma}}}
\newcommand{\chiomn}{\chi_{_{\Omega_n}}}
\newcommand{\ullim}{\underline{\lim}} \newcommand{\bsy}{\boldsymbol}
\newcommand{\mvb}{\mathversion{bold}} \newcommand{\la}{\lambda}
\newcommand{\La}{\Lambda} \newcommand{\va}{\varepsilon}
\newcommand{\be}{\beta} \newcommand{\al}{\alpha}
\newcommand{\dis}{\displaystyle} \newcommand{\R}{{\mathbb R}}
\newcommand{\N}{{\mathbb N}} \newcommand{\cF}{{\mathcal F}}
\newcommand{\gB}{{\mathfrak B}} \newcommand{\eps}{\epsilon}
\renewcommand\refname{参考文献}\renewcommand\figurename{图}
\usepackage[]{caption2} 
\renewcommand{\captionlabeldelim}{}
\setlength\parindent{0pt}
\begin{document}
\title{\huge{\bf{基变换与坐标变换}}} \author{\small{叶卢庆\footnote{叶卢庆(1992---),男,杭州师范大学理学院数学与应用数学专业本科在读,E-mail:yeluqingmathematics@gmail.com}}}
\maketitle
设向量$V\in \mathbf{R}^n$,且$\alpha=(w_1,\cdots,w_n)$是$\mathbf{R}^n$
的一组有序
基,则存在$a_1,\cdots,a_n\in \mathbf{R}$,使得
$$
V=a_1w_1+\cdots+a_nw_n.
$$
$(a_1,\cdots,a_n)$叫做$V$在有序基$\alpha$下的坐标.随着$V$的变
化,$(a_1,\cdots,a_n)$也会随之变化,可见,向量在某个基下的各个坐标是关于向量
的函数.\\\\

在此,我们揭示基变换和坐标变换之间的区别和联系.所谓的基变换,指的是两个基之间的关系.设
$\beta=(v_1,\cdots,v_n)$是$\mathbf{R}^n$的另一组有序基,设
$$
[I]_{\beta}^{\alpha}=
\begin{pmatrix}
  a_{11}&a_{12}&\cdots&a_{1n}\\
  a_{21}&a_{22}&\cdots&a_{2n}\\
  \vdots&\vdots&\ddots&\vdots\\
a_{n1}&a_{n2}&\cdots&a_{nn}
\end{pmatrix},
$$
即$\forall 1\leq i\leq n$,
$$
v_i=a_{1i}w_1+a_{2i}w_2+\cdots+a_{ni}w_n.
$$
也就是,
$$
\begin{pmatrix}
v_1\\
v_2\\
\vdots\\
v_n
\end{pmatrix}=\begin{pmatrix}
  a_{11}&a_{12}&\cdots&a_{1n}\\
  a_{21}&a_{22}&\cdots&a_{2n}\\
  \vdots&\vdots&\ddots&\vdots\\
a_{n1}&a_{n2}&\cdots&a_{nn}
\end{pmatrix}^T
\begin{pmatrix}
  w_1\\
w_2\\
\vdots\\
w_n
\end{pmatrix}.
$$
而且,在基$\beta$下坐标为$(b_1,\cdots,b_n)$的向量在$\alpha$下的坐标会成
为
$$
\begin{pmatrix}
  a_{11}&a_{12}&\cdots&a_{1n}\\
  a_{21}&a_{22}&\cdots&a_{2n}\\
  \vdots&\vdots&\ddots&\vdots\\
a_{n1}&a_{n2}&\cdots&a_{nn}
\end{pmatrix}
\begin{pmatrix}
  b_1\\
b_2\\
\vdots\\
b_n
\end{pmatrix}.
$$
这是很容易想到的.
\end{document}
