\documentclass{article}
\usepackage{amsmath}
\usepackage{amsthm,euler}
\usepackage{amssymb}
\usepackage{framed,mdframed}
\usepackage{amsfonts}
\usepackage{graphicx,color}
\usepackage{mathrsfs}
\usepackage[all]{xy}
\usepackage{fancybox}
\newtheorem*{theo}{Theorem}
\newtheorem*{exe}{Exercise}
\newtheorem*{rem}{Remark}
\newtheorem*{lemma}{Lemma}
\newtheorem*{coro}{Corollary}
\newtheorem*{exa}{Example}
\newenvironment{corollary}
{\bigskip\begin{mdframed}\begin{coro}}
    {\end{coro}\end{mdframed}\bigskip}
\newenvironment{theorem}
{\bigskip\begin{mdframed}\begin{theo}}
    {\end{theo}\end{mdframed}\bigskip}
\newenvironment{exercise}
{\bigskip\begin{mdframed}\begin{exe}}
    {\end{exe}\end{mdframed}\bigskip}
\newenvironment{example}
{\bigskip\begin{mdframed}\begin{exa}}
    {\end{exa}\end{mdframed}\bigskip}
\newenvironment{remark}
{\bigskip\begin{mdframed}\begin{rem}}
    {\end{rem}\end{mdframed}\bigskip}
\newcommand{\D}{\displaystyle}\newcommand{\ri}{\Rightarrow}
\newcommand{\ds}{\displaystyle} \renewcommand{\ni}{\noindent}
\newcommand{\ov}{\overrightarrow}
\newcommand{\pa}{\partial} \newcommand{\Om}{\Omega}
\newcommand{\om}{\omega} \newcommand{\sik}{\sum_{i=1}^k}
\newcommand{\vov}{\Vert\omega\Vert} \newcommand{\Umy}{U_{\mu_i,y^i}}
\newcommand{\lamns}{\lambda_n^{^{\scriptstyle\sigma}}}
\newcommand{\chiomn}{\chi_{_{\Omega_n}}}
\newcommand{\ullim}{\underline{\lim}} \newcommand{\bsy}{\boldsymbol}
\newcommand{\mvb}{\mathversion{bold}} \newcommand{\la}{\lambda}
\newcommand{\La}{\Lambda} \newcommand{\va}{\varepsilon}
\newcommand{\be}{\beta} \newcommand{\al}{\alpha}
\newcommand{\dis}{\displaystyle} \newcommand{\R}{{\mathbb R}}
\newcommand{\N}{{\mathbb N}} \newcommand{\cF}{{\mathcal F}}
\newcommand{\gB}{{\mathfrak B}} \newcommand{\eps}{\epsilon}
\usepackage[]{caption2} 
\renewcommand{\captionlabeldelim}{}
\begin{document}
\title{Exercise3.5.3}\author{Luqing Ye\footnote{An undergraduate at Hangzhou Normal University,Email:yeluqingmathematics@gmail.com}}
\maketitle\noindent
\begin{exercise}
  Let $\omega$ and $v$ be the following $1$-forms:
$$
\omega(\langle dx,dy\rangle)=2dx-3dy.
$$
$$
v(\langle dx,dy\rangle)=dx+dy.
$$
Find a constant $c$ such that $\omega\wedge v=c dx\wedge dy$.
\end{exercise}
\begin{proof}[Solve]
For any $V_1=\langle a_1,b_1\rangle$,$V_2=\langle a_2,b_2\rangle$.
$$
dx\wedge dy(V_1,V_2)=
\begin{vmatrix}
  a_1&b_1\\
a_2&b_2
\end{vmatrix}=a_1b_2-a_2b_1.
$$
$$
\omega\wedge v(V_1,V_2)=
\begin{vmatrix}
  \omega(V_1)&v(V_1)\\
\omega(V_2)&v(V_2)
\end{vmatrix}=5(a_1b_2-a_2b_1).
$$
So $c=5$.
\end{proof}






% \bibliographystyle{unsrt}
% \bibliography{}
\end{document}
