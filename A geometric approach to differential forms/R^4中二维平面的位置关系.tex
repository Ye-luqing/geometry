\documentclass[a4paper]{article}
\usepackage{amsmath,amsfonts,amsthm,amssymb}
\usepackage{bm}
\usepackage{hyperref}
\usepackage{geometry}
\usepackage{yhmath}
\usepackage{pstricks-add}
\usepackage{framed,mdframed}
\usepackage{graphicx,color} 
\usepackage{mathrsfs,xcolor} 
\usepackage[all]{xy}
\usepackage{fancybox} 
\usepackage{xeCJK}
\newtheorem*{theo}{定理}
\newtheorem*{exe}{题目}
\newtheorem*{rem}{评论}
\newtheorem*{lemma}{引理}
\newtheorem*{coro}{推论}
\newtheorem*{exa}{例}
\newenvironment{corollary}
{\bigskip\begin{mdframed}\begin{coro}}
    {\end{coro}\end{mdframed}\bigskip}
\newenvironment{theorem}
{\bigskip\begin{mdframed}\begin{theo}}
    {\end{theo}\end{mdframed}\bigskip}
\newenvironment{exercise}
{\bigskip\begin{mdframed}\begin{exe}}
    {\end{exe}\end{mdframed}\bigskip}
\newenvironment{example}
{\bigskip\begin{mdframed}\begin{exa}}
    {\end{exa}\end{mdframed}\bigskip}
\newenvironment{remark}
{\bigskip\begin{mdframed}\begin{rem}}
    {\end{rem}\end{mdframed}\bigskip}
\geometry{left=2.5cm,right=2.5cm,top=2.5cm,bottom=2.5cm}
\setCJKmainfont[BoldFont=SimHei]{SimSun}
\renewcommand{\today}{\number\year 年 \number\month 月 \number\day 日}
\newcommand{\D}{\displaystyle}\newcommand{\ri}{\Rightarrow}
\newcommand{\ds}{\displaystyle} \renewcommand{\ni}{\noindent}
\newcommand{\ov}{\overrightarrow}
\newcommand{\pa}{\partial} \newcommand{\Om}{\Omega}
\newcommand{\om}{\omega} \newcommand{\sik}{\sum_{i=1}^k}
\newcommand{\vov}{\Vert\omega\Vert} \newcommand{\Umy}{U_{\mu_i,y^i}}
\newcommand{\lamns}{\lambda_n^{^{\scriptstyle\sigma}}}
\newcommand{\chiomn}{\chi_{_{\Omega_n}}}
\newcommand{\ullim}{\underline{\lim}} \newcommand{\bsy}{\boldsymbol}
\newcommand{\mvb}{\mathversion{bold}} \newcommand{\la}{\lambda}
\newcommand{\La}{\Lambda} \newcommand{\va}{\varepsilon}
\newcommand{\be}{\beta} \newcommand{\al}{\alpha}
\newcommand{\dis}{\displaystyle} \newcommand{\R}{{\mathbb R}}
\newcommand{\N}{{\mathbb N}} \newcommand{\cF}{{\mathcal F}}
\newcommand{\gB}{{\mathfrak B}} \newcommand{\eps}{\epsilon}
\renewcommand\refname{参考文献}\renewcommand\figurename{图}
\usepackage[]{caption2} 
\renewcommand{\captionlabeldelim}{}
\setlength\parindent{0pt}
\begin{document}
\title{\huge{\bf{$\mathbf{R}^4$中二维平面的位置关系}}} \author{\small{叶卢庆\footnote{叶卢庆(1992---),男,杭州师范大学理学院数学与应用数学专业本科在读,E-mail:yeluqingmathematics@gmail.com}}}
\maketitle
四维空间直角坐标系$\mathbf{R}^4$中的任意一个二维平面的方程形如
$$
\begin{cases}
  A_1x+B_1y+C_1z+D_1w+E_1=0,\\
  A_2x+B_2y+C_2z+D_2w+E_2=0.
\end{cases}
$$
也就是说,$\mathbf{R}^{4}$中的任意一个二维平面定义为$\mathbf{R}^{4}$中
两个三维体的交.实际上,$\mathbf{R}^4$中任何两个不同的三维体不是平行,就
是相交.对于$\mathbf{R}^4$中的两个二维平面
\begin{equation}\label{eq:1}
\begin{cases}
  A_1x+B_1y+C_1z+D_1w+E_1=0,\\
  A_2x+B_2y+C_2z+D_2w+E_2=0.
\end{cases}
\end{equation}
和
\begin{equation}
  \label{eq:2}
\begin{cases}
  M_1x+N_1y+P_1z+Q_1w+T_1=0,\\
M_2x+N_2y+P_2z+Q_2w+T_2=0.
\end{cases}
\end{equation}
当行列式
$$
\begin{vmatrix}
  A_1&B_1&C_1&D_1\\
A_2&B_2&C_2&D_2\\
M_1&N_1&P_1&Q_1\\
M_2&N_2&P_2&Q_2
\end{vmatrix}\neq 0
$$
时,平面\eqref{eq:1}和平面\eqref{eq:2}交于一个点.否则,平面\eqref{eq:1}
和\eqref{eq:2}的方程联立之后,可能无解,也可能有无限个解.当无解时,说明两
平面平行.当有无限解时,说明两平面交于一直线.注意$\mathbf{R}^4$中的两平
面不可能异体.因为异体至少要求维数达到$2+2+1=5$.
\end{document}
