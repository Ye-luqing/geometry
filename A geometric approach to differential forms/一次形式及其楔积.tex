\documentclass[a4paper]{article}
\usepackage{amsmath,amsfonts,amsthm,amssymb}
\usepackage{bm}
\usepackage{hyperref}
\usepackage{geometry}
\usepackage{yhmath}
\usepackage{pstricks-add}
\usepackage{framed,mdframed}
\usepackage{graphicx,color} 
\usepackage{mathrsfs,xcolor} 
\usepackage[all]{xy}
\usepackage{fancybox} 
\usepackage{xeCJK}
\newtheorem*{theo}{定理}
\newtheorem*{exe}{题目}
\newtheorem*{rem}{评论}
\newtheorem*{lemma}{引理}
\newtheorem*{coro}{推论}
\newtheorem*{exa}{例}
\newenvironment{corollary}
{\bigskip\begin{mdframed}\begin{coro}}
    {\end{coro}\end{mdframed}\bigskip}
\newenvironment{theorem}
{\bigskip\begin{mdframed}\begin{theo}}
    {\end{theo}\end{mdframed}\bigskip}
\newenvironment{exercise}
{\bigskip\begin{mdframed}\begin{exe}}
    {\end{exe}\end{mdframed}\bigskip}
\newenvironment{example}
{\bigskip\begin{mdframed}\begin{exa}}
    {\end{exa}\end{mdframed}\bigskip}
\newenvironment{remark}
{\bigskip\begin{mdframed}\begin{rem}}
    {\end{rem}\end{mdframed}\bigskip}
\geometry{left=2.5cm,right=2.5cm,top=2.5cm,bottom=2.5cm}
\setCJKmainfont[BoldFont=SimHei]{SimSun}
\renewcommand{\today}{\number\year 年 \number\month 月 \number\day 日}
\newcommand{\D}{\displaystyle}\newcommand{\ri}{\Rightarrow}
\newcommand{\ds}{\displaystyle} \renewcommand{\ni}{\noindent}
\newcommand{\ov}{\overrightarrow}
\newcommand{\pa}{\partial} \newcommand{\Om}{\Omega}
\newcommand{\om}{\omega} \newcommand{\sik}{\sum_{i=1}^k}
\newcommand{\vov}{\Vert\omega\Vert} \newcommand{\Umy}{U_{\mu_i,y^i}}
\newcommand{\lamns}{\lambda_n^{^{\scriptstyle\sigma}}}
\newcommand{\chiomn}{\chi_{_{\Omega_n}}}
\newcommand{\ullim}{\underline{\lim}} \newcommand{\bsy}{\boldsymbol}
\newcommand{\mvb}{\mathversion{bold}} \newcommand{\la}{\lambda}
\newcommand{\La}{\Lambda} \newcommand{\va}{\varepsilon}
\newcommand{\be}{\beta} \newcommand{\al}{\alpha}
\newcommand{\dis}{\displaystyle} \newcommand{\R}{{\mathbb R}}
\newcommand{\N}{{\mathbb N}} \newcommand{\cF}{{\mathcal F}}
\newcommand{\gB}{{\mathfrak B}} \newcommand{\eps}{\epsilon}
\renewcommand\refname{参考文献}\renewcommand\figurename{图}
\usepackage[]{caption2} 
\renewcommand{\captionlabeldelim}{}
\setlength\parindent{0pt}
\begin{document}
\title{\huge{\bf{一次形式及其楔积}}} \author{\small{叶卢庆\footnote{叶卢庆(1992---),男,杭州师范大学理学院数学与应用数学专业本科在读,E-mail:yeluqingmathematics@gmail.com}}}
\maketitle
设$w=a_{1}dx_{1}+\cdots+a_{n}dx_{n}$,其中$a_1,\cdots,a_n$
是常数,且$dx_{i}$是坐标函数,表示$n$维向量$V$的第$i$个坐标.可见,$w$可
以被一个$1\times n$的矩阵
$$
\begin{pmatrix}
  a_1&
\cdots&
a_n
\end{pmatrix}.
$$
表示.$w$叫做一个一次形式,且这个一次形式是一个把向量变成实数的线性变换$T:\mathbf{R}^{n}\to \mathbf{R}$.\\

我们再看另外一个一次形式$v=b_1dx_1+\cdots+b_ndx_n$,其中
$b_1,\cdots,b_n$是常数,且$dx_i$是坐标函数.两个一次形式$w,v$形成一个整体时,是从$\mathbf{R}^{n}$到$\mathbf{R}^{2}$的线
性变换,可以用一个$2\times n$的矩阵表示
$$
\begin{pmatrix}
  a_1&\cdots&a_n\\
b_1&\cdots&b_n
\end{pmatrix}.
$$
任意一个$\mathbf{R}^n$中的向量经过这个矩阵的作用后,会变成一个
$\mathbf{R}^2$中的向量.任意两个$\mathbf{R}^n$中的向量经过这个矩阵作用
后,会变成两个$\mathbf{R}^2$中的向量.这两个$\mathbf{R}^2$中的向量可以按
照顺序形成一个行列式,这个行列式就是两个一次形式的楔积对于两个向量的作
用.
\end{document}






二次形式是把两个向量变成数字的双线性变换,同时,当两个向量交换地位时,数
字的值会相反.我们将两个向量中的一个向量固
定,让另一个向量不固定,那么二次形
式形如
$$
a_1dx_1+\cdots+a_ndx_n,
$$
其中$dx_i$为坐标函数,表示$n$维向量$V$的第$i$个坐标.
