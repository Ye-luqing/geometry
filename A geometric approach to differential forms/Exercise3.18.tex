\documentclass{article}
\usepackage{amsmath}
\usepackage{amsthm,euler}
\usepackage{amssymb}
\usepackage{framed,mdframed}
\usepackage{amsfonts}
\usepackage{graphicx,color}
\usepackage{mathrsfs}
\usepackage[all]{xy}
\usepackage{fancybox}
\newtheorem*{theo}{Theorem}
\newtheorem*{exe}{Exercise}
\newtheorem*{rem}{Remark}
\newtheorem*{lemma}{Lemma}
\newtheorem*{coro}{Corollary}
\newtheorem*{exa}{Example}
\newenvironment{corollary}
{\bigskip\begin{mdframed}\begin{coro}}
    {\end{coro}\end{mdframed}\bigskip}
\newenvironment{theorem}
{\bigskip\begin{mdframed}\begin{theo}}
    {\end{theo}\end{mdframed}\bigskip}
\newenvironment{exercise}
{\bigskip\begin{mdframed}\begin{exe}}
    {\end{exe}\end{mdframed}\bigskip}
\newenvironment{example}
{\bigskip\begin{mdframed}\begin{exa}}
    {\end{exa}\end{mdframed}\bigskip}
\newenvironment{remark}
{\bigskip\begin{mdframed}\begin{rem}}
    {\end{rem}\end{mdframed}\bigskip}
\newcommand{\D}{\displaystyle}\newcommand{\ri}{\Rightarrow}
\newcommand{\ds}{\displaystyle} \renewcommand{\ni}{\noindent}
\newcommand{\ov}{\overrightarrow}
\newcommand{\pa}{\partial} \newcommand{\Om}{\Omega}
\newcommand{\om}{\omega} \newcommand{\sik}{\sum_{i=1}^k}
\newcommand{\vov}{\Vert\omega\Vert} \newcommand{\Umy}{U_{\mu_i,y^i}}
\newcommand{\lamns}{\lambda_n^{^{\scriptstyle\sigma}}}
\newcommand{\chiomn}{\chi_{_{\Omega_n}}}
\newcommand{\ullim}{\underline{\lim}} \newcommand{\bsy}{\boldsymbol}
\newcommand{\mvb}{\mathversion{bold}} \newcommand{\la}{\lambda}
\newcommand{\La}{\Lambda} \newcommand{\va}{\varepsilon}
\newcommand{\be}{\beta} \newcommand{\al}{\alpha}
\newcommand{\dis}{\displaystyle} \newcommand{\R}{{\mathbb R}}
\newcommand{\N}{{\mathbb N}} \newcommand{\cF}{{\mathcal F}}
\newcommand{\gB}{{\mathfrak B}} \newcommand{\eps}{\epsilon}
\usepackage[]{caption2} 
\renewcommand{\captionlabeldelim}{}
\begin{document}
\title{Exercise 3.18}\author{Luqing Ye\footnote{An undergraduate at Hangzhou Normal University,Email:yeluqingmathematics@gmail.com}}
\maketitle\noindent
\begin{exercise}
  Find a $2$- form which is not the product of $1$- forms.
\end{exercise}
\begin{proof}[Solve]
We first prove that any $2$-form on $T_p\mathbf{R}^2$ can be expressed
as the product of two $1$-forms.Any $2$- form on $T_p\mathbf{R}^2$ can
be expressed as $a dx\wedge dy$,where $a$ is a real number.
$$
(a_{1}dx+b_{1}dy)\wedge (a_2dx+b_2dy)=(a_1b_2-b_{1}a_{2})dx\wedge dy.
$$
So let $a_1=a,b_2=1,b_1=0$ then we are done.\\\\

Next we prove that any $2$- form on $T_p\mathbf{R}^3$ can be expressed
as the product of two $1$-forms.Any $2$-form on $T_p\mathbf{R}^3$ can
be expressed as $a dx\wedge dy+b dy\wedge dz+c dz\wedge dx$,where
$a,b,c$ are real numbers.
\begin{align*}
&(a_1dx+b_1dy+c_1dz)\wedge (a_2dx+b_2dy+c_2dz)\\&=(a_1b_2-a_2b_1)dx\wedge
dy+(b_1c_2-c_1b_2)dy\wedge dz+(c_1a_2-a_1c_2)dz\wedge dx.
\end{align*}
Let 
$$
\begin{cases}
  a=a_1b_2-a_2b_1,\\
b=b_1c_2-b_2c_1,\\
c=c_1a_2-c_2a_1
\end{cases}
$$
then 
$$
(b,c,a)=(a_1,b_1,c_1)\times (a_2,b_2,c_2).
$$
According to the geometrical interpretion of the vector product,we
know that the solution $(a_1,b_1,c_1),(a_2,b_2,c_2)$ exists.\\\\

Next we prove that there exist $2$- forms on $T_p\mathbf{R}^4$ such
that it can not be expressed as the product of two $1$- forms.
\begin{align*}
&(a_1dx+b_1dy+c_1dz+d_1dw)\wedge
(a_2dx+b_2dy+c_2dz+d_2dw)\\&=(a_1b_2-a_2b_1)dx\wedge
dy+(a_1c_2-a_2c_1)dx\wedge dz+(a_1d_2-a_2d_1)dx\wedge
dw\\&+(b_1c_2-b_2c_1)dy\wedge dz+(b_1d_2-d_1b_2)dy\wedge
dw+(c_1d_2-c_2d_1)dz\wedge dw,
\end{align*}
let $a_1b_2-a_2b_1=0$,then
$$a_1c_2-a_2c_1:b_1c_2-b_2c_1=a_1d_2-a_2d_1:b_1d_2-d_1b_2,$$So the two
form $dx\wedge dz+dx\wedge dw+dy\wedge dz+2dy\wedge dw+dz\wedge dw$ in
$T_p\mathbf{R}^4$ can not be expressed as the product of two $1$- forms.
\end{proof}
\end{document}
