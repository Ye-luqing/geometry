\documentclass[a4paper]{article}
\usepackage{amsmath,amsfonts,amsthm,amssymb}
\usepackage{bm}
\usepackage{hyperref}
\usepackage{geometry}
\usepackage{yhmath}
\usepackage{pstricks-add}
\usepackage{framed,mdframed}
\usepackage{graphicx,color} 
\usepackage{mathrsfs,xcolor} 
\usepackage[all]{xy}
\usepackage{fancybox} 
\usepackage{xeCJK}
\newtheorem*{theo}{定理}
\newtheorem*{exe}{Exercise}
\newtheorem*{rem}{评论}
\newtheorem*{lemma}{引理}
\newtheorem*{coro}{推论}
\newtheorem*{exa}{例}
\newenvironment{corollary}
{\bigskip\begin{mdframed}\begin{coro}}
    {\end{coro}\end{mdframed}\bigskip}
\newenvironment{theorem}
{\bigskip\begin{mdframed}\begin{theo}}
    {\end{theo}\end{mdframed}\bigskip}
\newenvironment{exercise}
{\bigskip\begin{mdframed}\begin{exe}}
    {\end{exe}\end{mdframed}\bigskip}
\newenvironment{example}
{\bigskip\begin{mdframed}\begin{exa}}
    {\end{exa}\end{mdframed}\bigskip}
\newenvironment{remark}
{\bigskip\begin{mdframed}\begin{rem}}
    {\end{rem}\end{mdframed}\bigskip}
\geometry{left=2.5cm,right=2.5cm,top=2.5cm,bottom=2.5cm}
\setCJKmainfont[BoldFont=SimHei]{SimSun}
\renewcommand{\today}{\number\year 年 \number\month 月 \number\day 日}
\newcommand{\D}{\displaystyle}\newcommand{\ri}{\Rightarrow}
\newcommand{\ds}{\displaystyle} \renewcommand{\ni}{\noindent}
\newcommand{\ov}{\overrightarrow}
\newcommand{\pa}{\partial} \newcommand{\Om}{\Omega}
\newcommand{\om}{\omega} \newcommand{\sik}{\sum_{i=1}^k}
\newcommand{\vov}{\Vert\omega\Vert} \newcommand{\Umy}{U_{\mu_i,y^i}}
\newcommand{\lamns}{\lambda_n^{^{\scriptstyle\sigma}}}
\newcommand{\chiomn}{\chi_{_{\Omega_n}}}
\newcommand{\ullim}{\underline{\lim}} \newcommand{\bsy}{\boldsymbol}
\newcommand{\mvb}{\mathversion{bold}} \newcommand{\la}{\lambda}
\newcommand{\La}{\Lambda} \newcommand{\va}{\varepsilon}
\newcommand{\be}{\beta} \newcommand{\al}{\alpha}
\newcommand{\dis}{\displaystyle} \newcommand{\R}{{\mathbb R}}
\newcommand{\N}{{\mathbb N}} \newcommand{\cF}{{\mathcal F}}
\newcommand{\gB}{{\mathfrak B}} \newcommand{\eps}{\epsilon}
\renewcommand\refname{参考文献}\renewcommand\figurename{图}
\usepackage[]{caption2} 
\renewcommand{\captionlabeldelim}{}
\setlength\parindent{0pt}
\begin{document}
\begin{exercise}
Let $D$ be some region in the $xy$-plane. Let $M$ denote the portion of the
graph of $z = g(x, y)$ that lies above $D$.
\begin{itemize}
\item Let $\omega=f(x,y)dx\wedge dy$ be a differential $2$-form on
  $\mathbf{R}^3$.Show that 
$$
\int_M\omega=\int_Df(x,y)dxdy.
$$ 
\item Now suppose $\omega =f(x,y,z)dx\wedge dy$.Show that 
$$
\int_M\omega=\int_Df(x,y,g(x,y))dxdy.
$$
\end{itemize}
\end{exercise}
\begin{itemize}
\item The surface can be parameterized as
$$
p:(x,y)\to (x,y,g(x,y)).
$$
$$
\frac{\pa p}{\pa x}=\left(1,0,\frac{\pa g}{\pa x}\right),\frac{\pa p}{\pa
  y}=\left(0,1,\frac{\pa g}{\pa y}\right).
$$
$$
\omega\left(\frac{\pa p}{\pa x},\frac{\pa p}{\pa y}\right)=f(x,y).
$$
So 
$$
\int_M\omega=\int_Df(x,y)dxdy.
$$
\item 
$$
\omega \left(\frac{\pa p}{\pa x},\frac{\pa p}{\pa y}\right)=f(x,y,z)=f(x,y,g(x,y)).
$$
So
$$
\int_M\omega=\int_Df(x,y,g(x,y))dxdy.
$$
\end{itemize}
\end{document}
