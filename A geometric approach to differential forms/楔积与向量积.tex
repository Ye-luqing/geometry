\documentclass[a4paper]{article}
\usepackage{amsmath,amsfonts,amsthm,amssymb}
\usepackage{bm}
\usepackage{hyperref}
\usepackage{geometry}
\usepackage{yhmath}
\usepackage{pstricks-add}
\usepackage{framed,mdframed}
\usepackage{graphicx,color} 
\usepackage{mathrsfs,xcolor} 
\usepackage[all]{xy}
\usepackage{fancybox} 
\usepackage{xeCJK}
\newtheorem*{theo}{定理}
\newtheorem*{exe}{题目}
\newtheorem*{rem}{评论}
\newtheorem*{lemma}{引理}
\newtheorem*{coro}{推论}
\newtheorem*{exa}{例}
\newenvironment{corollary}
{\bigskip\begin{mdframed}\begin{coro}}
    {\end{coro}\end{mdframed}\bigskip}
\newenvironment{theorem}
{\bigskip\begin{mdframed}\begin{theo}}
    {\end{theo}\end{mdframed}\bigskip}
\newenvironment{exercise}
{\bigskip\begin{mdframed}\begin{exe}}
    {\end{exe}\end{mdframed}\bigskip}
\newenvironment{example}
{\bigskip\begin{mdframed}\begin{exa}}
    {\end{exa}\end{mdframed}\bigskip}
\newenvironment{remark}
{\bigskip\begin{mdframed}\begin{rem}}
    {\end{rem}\end{mdframed}\bigskip}
\geometry{left=2.5cm,right=2.5cm,top=2.5cm,bottom=2.5cm}
\setCJKmainfont[BoldFont=SimHei]{SimSun}
\renewcommand{\today}{\number\year 年 \number\month 月 \number\day 日}
\newcommand{\D}{\displaystyle}\newcommand{\ri}{\Rightarrow}
\newcommand{\ds}{\displaystyle} \renewcommand{\ni}{\noindent}
\newcommand{\ov}{\overrightarrow}
\newcommand{\pa}{\partial} \newcommand{\Om}{\Omega}
\newcommand{\om}{\omega} \newcommand{\sik}{\sum_{i=1}^k}
\newcommand{\vov}{\Vert\omega\Vert} \newcommand{\Umy}{U_{\mu_i,y^i}}
\newcommand{\lamns}{\lambda_n^{^{\scriptstyle\sigma}}}
\newcommand{\chiomn}{\chi_{_{\Omega_n}}}
\newcommand{\ullim}{\underline{\lim}} \newcommand{\bsy}{\boldsymbol}
\newcommand{\mvb}{\mathversion{bold}} \newcommand{\la}{\lambda}
\newcommand{\La}{\Lambda} \newcommand{\va}{\varepsilon}
\newcommand{\be}{\beta} \newcommand{\al}{\alpha}
\newcommand{\dis}{\displaystyle} \newcommand{\R}{{\mathbb R}}
\newcommand{\N}{{\mathbb N}} \newcommand{\cF}{{\mathcal F}}
\newcommand{\gB}{{\mathfrak B}} \newcommand{\eps}{\epsilon}
\renewcommand\refname{参考文献}\renewcommand\figurename{图}
\usepackage[]{caption2} 
\renewcommand{\captionlabeldelim}{}
\setlength\parindent{0pt}
\begin{document}
\title{\huge{\bf{楔积与向量积}}} \author{\small{叶卢庆\footnote{叶卢庆(1992---),男,杭州师范大学理学院数学与应用数学专业本科在读,E-mail:yeluqingmathematics@gmail.com}}}
\maketitle
设
$\langle
\omega\rangle=a_1\mathbf{i}+a_2\mathbf{j}+a_3\mathbf{k}$,$\langle \nu\rangle=b_1\mathbf{i}+b_2\mathbf{j}+b_3\mathbf{k}$,则
\begin{align*}
  \langle\omega\rangle\times
  \langle\nu\rangle&=(a_1\mathbf{i}+a_2\mathbf{j}+a_3\mathbf{k})\times (b_1\mathbf{i}+b_2\mathbf{j}+b_3\mathbf{k})
\\&=(a_2b_3-b_2a_3)\mathbf{i}+(b_1a_3-a_1b_3)\mathbf{j}+(a_1b_2-b_1a_2)\mathbf{k}.
\end{align*}
我们还知道,设
$\omega=a_1dx+a_2dy+a_3dz$,$\nu=b_1dx+b_2dy+b_3dz$,
则
\begin{align*}
  \omega\wedge \nu&=(a_1b_2-a_2b_1)dx\wedge dy+(a_2b_3-a_3b_2)dy\wedge
  dz+(a_3b_1-a_1b_3)dz\wedge dx.
\end{align*}
可见,在映射$T(dx\wedge dy)=\mathbf{k}$,$T(dy\wedge
dz)=\mathbf{i}$,$T(dz\wedge dx)=\mathbf{j}$下,楔积和向量积之间建立了同构
的关系.且
\begin{equation}
  T(\omega\wedge v)=\langle\omega\rangle\times\langle \nu\rangle.
\end{equation}
根据Lagrange恒等式,
\begin{align*}
\omega\wedge \nu(V_1,V_2)&=
\begin{vmatrix}
  \omega(V_1)&\nu(V_1)\\
\omega(V_2)&\nu(V_2)
\end{vmatrix}\\&=
\begin{vmatrix}
  \langle \omega\rangle\cdot V_1&\langle \nu\rangle \cdot V_1\\
\langle \omega\rangle\cdot V_2&\langle \nu\rangle\cdot V_2
\end{vmatrix}\\&=(V_1\times V_2)\cdot \left(\langle \omega\rangle\times
\langle \nu\rangle\right),
\end{align*}
可见,$\omega\wedge \nu(V_1,V_2)$的几何意义,就是以向量$V_1,V_2$为邻边的
有向平行四边形,投影到$\langle\omega\rangle,\langle\nu\rangle$展成的平面上所得
的平行四边形的有向面积,再乘以以$\langle
\omega\rangle$,$\langle\nu\rangle$为邻边的平行四边形的有向面积.相当于
两个有向平行四边形之间进行的“内积”.
\end{document}
