\documentclass[a4paper]{article}
\usepackage{amsmath,amsfonts,amsthm,amssymb}
\usepackage{bm}
\usepackage{hyperref}
\usepackage{geometry}
\usepackage{yhmath}
\usepackage{pstricks-add}
\usepackage{framed,mdframed}
\usepackage{graphicx,color} 
\usepackage{mathrsfs,xcolor} 
\usepackage[all]{xy}
\usepackage{fancybox} 
\usepackage{xeCJK}
\newtheorem*{theo}{定理}
\newtheorem*{exe}{题目}
\newtheorem*{rem}{评论}
\newtheorem*{lemma}{引理}
\newtheorem*{coro}{推论}
\newtheorem*{exa}{例}
\newenvironment{corollary}
{\bigskip\begin{mdframed}\begin{coro}}
    {\end{coro}\end{mdframed}\bigskip}
\newenvironment{theorem}
{\bigskip\begin{mdframed}\begin{theo}}
    {\end{theo}\end{mdframed}\bigskip}
\newenvironment{exercise}
{\bigskip\begin{mdframed}\begin{exe}}
    {\end{exe}\end{mdframed}\bigskip}
\newenvironment{example}
{\bigskip\begin{mdframed}\begin{exa}}
    {\end{exa}\end{mdframed}\bigskip}
\newenvironment{remark}
{\bigskip\begin{mdframed}\begin{rem}}
    {\end{rem}\end{mdframed}\bigskip}
\geometry{left=2.5cm,right=2.5cm,top=2.5cm,bottom=2.5cm}
\setCJKmainfont[BoldFont=SimHei]{SimSun}
\renewcommand{\today}{\number\year 年 \number\month 月 \number\day 日}
\newcommand{\D}{\displaystyle}\newcommand{\ri}{\Rightarrow}
\newcommand{\ds}{\displaystyle} \renewcommand{\ni}{\noindent}
\newcommand{\ov}{\overrightarrow}
\newcommand{\pa}{\partial} \newcommand{\Om}{\Omega}
\newcommand{\om}{\omega} \newcommand{\sik}{\sum_{i=1}^k}
\newcommand{\vov}{\Vert\omega\Vert} \newcommand{\Umy}{U_{\mu_i,y^i}}
\newcommand{\lamns}{\lambda_n^{^{\scriptstyle\sigma}}}
\newcommand{\chiomn}{\chi_{_{\Omega_n}}}
\newcommand{\ullim}{\underline{\lim}} \newcommand{\bsy}{\boldsymbol}
\newcommand{\mvb}{\mathversion{bold}} \newcommand{\la}{\lambda}
\newcommand{\La}{\Lambda} \newcommand{\va}{\varepsilon}
\newcommand{\be}{\beta} \newcommand{\al}{\alpha}
\newcommand{\dis}{\displaystyle} \newcommand{\R}{{\mathbb R}}
\newcommand{\N}{{\mathbb N}} \newcommand{\cF}{{\mathcal F}}
\newcommand{\gB}{{\mathfrak B}} \newcommand{\eps}{\epsilon}
\renewcommand\refname{参考文献}\renewcommand\figurename{图}
\usepackage[]{caption2} 
\renewcommand{\captionlabeldelim}{}
\setlength\parindent{0pt}
\begin{document}
\title{\huge{\bf{$\mathbf{R}^n$中平行四边形的面积}}} \author{\small{叶卢庆\footnote{叶卢庆(1992---),男,杭州师范大学理学院数学与应用数学专业本科在读,E-mail:yeluqingmathematics@gmail.com}}}
\maketitle
在此我们探讨$\mathbf{R}^n$中平行四边形的面积.设
$\mathbf{a}=(a_1,a_2,\cdots,a_n)\in
\mathbf{R}^n$,$\mathbf{b}=(b_1,b_2,\cdots,b_n)\in \mathbf{R}^n$,我们来
计算$\mathbf{R}^n$中以$\mathbf{a},\mathbf{b}$为邻边张成的平行四边形$S$的
面积.易得其面积的平方为
{\small\begin{equation}\label{eq:1}
||\mathbf{a}||^{2}||\mathbf{b}||^2-(\mathbf{a}\cdot
\mathbf{b})^2=(a_1^2+a_2^2+\cdots+a_n^2)(b_1^2+b_2^2+\cdots+b_n^2)-(a_1b_1+a_2b_2+\cdots+a_nb_n)^2=\sum_{1\leq
  i<j\leq n}(a_ib_j-a_jb_i)^2.
\end{equation}}
我们惊讶地发现,$(a_ib_j-a_jb_i)^2$是$\mathbf{a},\mathbf{b}$张成的平行
四边形在$x_iOx_{j}$坐标平面的正投影得到的平行四边形的面积的平方,这样,我们
可以认为式\eqref{eq:1}表达的,和勾股定理很类似,可以看作是面积之间的“勾股定理”.\\

下面我们从几何角度来阐述式\eqref{eq:1}.设平行四边形$S$所在的平面与坐标
平面$x_iOx_j$的夹角为$\alpha_{i,j}$.平行四边形$S$正投影到$x_iOx_{j}$平
面上,得到平行四边形$S_{i,j}$.然后平行四边形$S_{i,j}$反过来又正投影到平行
四边形$S$所在平面上.易得$S_{i,j}$在$S$平面上正投影所得的图形$S_{i,j}'$
是完全含于$S$的,且$S_{i,j}'$也是个平行四边形.且$S_{i,j}'$的面积为
$S\cos^2\alpha_{i,j}$.且由于$\mathbf{R}^{n}$中过二维平面外一点有且仅有一条
直线与平面垂直,因此当$m\neq i$,$m\neq
j$,$w\neq i$,$w\neq j$这四种情况只要发生一种,便有图形$S_{i,j}'\cap
S_{m,n}'$的面积为$0$.而且$\cup_{1\leq i<j\leq n}S'_{i,j}=S$.于是,
$$
\sum_{1\leq i<j\leq n}S\cos^{2}\alpha_{ij}=S.
$$
也即,
$$
\sum_{1\leq i<j\leq n}\left(S\cos\alpha_{ij}\right)^{2}=S^2.
$$
这就是式\eqref{eq:1}.这样我们就给出了式\eqref{eq:1}的几何解释.
\end{document}




事实上,设平行四边形$S$与坐标平面$x_iOx_j$之间的夹角为$\alpha_{i,j}$,则
为了从几何上证明式\eqref{eq:1},我们只用证明
\begin{equation}
  \label{eq:2}
  \sum_{1\leq i<j\leq n}\cos^2\alpha_{i,j}=1.
\end{equation}
为了从几何上证明式\eqref{eq:2}.我们只用证明$\{\cos\alpha_{i,j}:1\leq i<j\leq
n\}$是某个向量$\mathbf{ON}$的所有方向余弦形成的集合即可.下面我们证
明,$\mathbf{ON}$可以这样构造:在平行四边形$S$所在的平面上选取点$N$,连接
原点$O$和$N$,使得$\ov{ON}$与平行四边形$S$正交,令$\ov{ON}=\mathbf{ON}$
即可.这是因为,设平行四边形$S$所在的平面与坐标平面$x_iOx_j$交于线
$L_{i,j}$.再过原点作$L_{i,j}$的垂线$T_{i,j}$.则线$T_{i,j}$和坐标轴
$x_k$形成的平面$P$垂直于直线$L_{i,j}$.由初等几何可得线段$ON$位于平面
$P$内,且$\mathbf{ON}$与$x_k$轴形成的方向角等于角$\alpha_{i,j}$.这样就
证明了$\{\cos\alpha_{i,j}:1\leq i<j\leq
n\}$是某个向量$\mathbf{ON}$的所有方向余弦形成的集合.\\

当然,平行四边行$S$所在的平面可能与$x_k$轴不相交,也可能与$x_iOx_j$平面
不相交.但是容易验证在这些极端情形,$\mathbf{ON}$与$x_k$轴形成的方向角仍
然等于角$\alpha_{i,j}$.\\












下面我们将式\eqref{eq:1}与两个一次形式的楔积联系起来.我们知道,当
$\omega=a_1dx_1+a_2dx_2+\cdots+a_ndx_n$,$\nu=b_1dx_1+b_2dx_2+\cdots+b_ndx_n$
时,
$$
\omega\wedge\nu=\sum_{1\leq i<j\leq n}(a_ib_j-a_jb_i) dx_i\wedge dx_j.
$$
这和式\eqref{eq:1}应该有比较大的关系.我们知道,