\documentclass[a4paper]{article}
\usepackage{amsmath,amsfonts,amsthm,amssymb}
\usepackage{bm}
\usepackage{hyperref}
\usepackage{geometry}
\usepackage{yhmath}
\usepackage{pstricks-add}
\usepackage{framed,mdframed}
\usepackage{graphicx,color} 
\usepackage{mathrsfs,xcolor} 
\usepackage[all]{xy}
\usepackage{fancybox} 
\usepackage{xeCJK}
\newtheorem*{theo}{定理}
\newtheorem*{exe}{题目}
\newtheorem*{rem}{评论}
\newtheorem*{lemma}{引理}
\newtheorem*{coro}{推论}
\newtheorem*{exa}{例}
\newenvironment{corollary}
{\bigskip\begin{mdframed}\begin{coro}}
    {\end{coro}\end{mdframed}\bigskip}
\newenvironment{theorem}
{\bigskip\begin{mdframed}\begin{theo}}
    {\end{theo}\end{mdframed}\bigskip}
\newenvironment{exercise}
{\bigskip\begin{mdframed}\begin{exe}}
    {\end{exe}\end{mdframed}\bigskip}
\newenvironment{example}
{\bigskip\begin{mdframed}\begin{exa}}
    {\end{exa}\end{mdframed}\bigskip}
\newenvironment{remark}
{\bigskip\begin{mdframed}\begin{rem}}
    {\end{rem}\end{mdframed}\bigskip}
\geometry{left=2.5cm,right=2.5cm,top=2.5cm,bottom=2.5cm}
\setCJKmainfont[BoldFont=SimHei]{SimSun}
\renewcommand{\today}{\number\year 年 \number\month 月 \number\day 日}
\newcommand{\D}{\displaystyle}\newcommand{\ri}{\Rightarrow}
\newcommand{\ds}{\displaystyle} \renewcommand{\ni}{\noindent}
\newcommand{\ov}{\overrightarrow}
\newcommand{\pa}{\partial} \newcommand{\Om}{\Omega}
\newcommand{\om}{\omega} \newcommand{\sik}{\sum_{i=1}^k}
\newcommand{\vov}{\Vert\omega\Vert} \newcommand{\Umy}{U_{\mu_i,y^i}}
\newcommand{\lamns}{\lambda_n^{^{\scriptstyle\sigma}}}
\newcommand{\chiomn}{\chi_{_{\Omega_n}}}
\newcommand{\ullim}{\underline{\lim}} \newcommand{\bsy}{\boldsymbol}
\newcommand{\mvb}{\mathversion{bold}} \newcommand{\la}{\lambda}
\newcommand{\La}{\Lambda} \newcommand{\va}{\varepsilon}
\newcommand{\be}{\beta} \newcommand{\al}{\alpha}
\newcommand{\dis}{\displaystyle} \newcommand{\R}{{\mathbb R}}
\newcommand{\N}{{\mathbb N}} \newcommand{\cF}{{\mathcal F}}
\newcommand{\gB}{{\mathfrak B}} \newcommand{\eps}{\epsilon}
\renewcommand\refname{参考文献}\renewcommand\figurename{图}
\usepackage[]{caption2} 
\renewcommand{\captionlabeldelim}{}
\setlength\parindent{0pt}
\begin{document}
\title{\huge{\bf{一次形式楔积的几何解释}}} \author{\small{叶卢庆\footnote{叶卢庆(1992---),男,杭州师范大学理学院数学与应用数学专业本科在读,E-mail:yeluqingmathematics@gmail.com}}}
\maketitle
在此我们从几何的角度来解释一次形式的楔积.两个一次形式分别形如
$$
w=a_1dx_1+a_2dx_2+\cdots+a_ndx_n,v=b_1dx_1+b_2dx_2+\cdots+b_ndx_n.
$$
两个一次形式的楔积$w\wedge v$的几何意义,必须要通过作用于具体的向量才能
看出来.设$V_1,V_2$是$T_p\mathbf{R}^n$中两个线性无关的向量,我们来看
$$
w\wedge v(V_1,V_2)=
\begin{vmatrix}
  w(V_1)&v(V_1)\\
w(V_2)&v(V_2)
\end{vmatrix}
$$
的几何解释.首先,$T_p\mathbf{R}^n$中的向量$V_1$在$\langle w\rangle$上的
投影再乘以$\langle w\rangle$的长度就是$w(V_1)$,$V_1$在$\langle
v\rangle$上的投影再乘以$\langle v\rangle$的长度就是$w(V_2)$.向量
$\langle w\rangle$和向量$\langle v\rangle$未必是正交的,如图所
示.$OA=w(V_1),OB=v(V_1)$,则点$P$的坐标是$(w(V_1),v(V_1))$,其中$OAPB$
是平行四边形.类似地,$P'$的坐标是$(w(V_2),v(V_2))$.\\

以向量$OP',OP$为邻边,张成了一个平行四边形.该平行四边形的面积,再除以向
量$\frac{\langle w\rangle}{|\langle w\rangle|}$和$\frac{\langle
  v\rangle}{|\langle v\rangle|}$张成的平行四边形的面积,就是$w\wedge v(V_1,V_2)$.\\
\begin{figure}[h]
\newrgbcolor{xdxdff}{0.490196078431 0.490196078431 1.}
\newrgbcolor{zzttqq}{0.6 0.2 0.}
\psset{xunit=1.0cm,yunit=1.0cm,algebraic=true,dimen=middle,dotstyle=o,dotsize=3pt 0,linewidth=0.8pt,arrowsize=3pt 2,arrowinset=0.25}
\begin{pspicture*}(0,-5.82)(22.8,4.3)
\pspolygon[linecolor=zzttqq,fillcolor=zzttqq,fillstyle=solid,opacity=0.1](3.64,0.58)(6.36375433438,1.81169773209)(11.12,0.28)(7.80130856527,-0.936409053446)
\pspolygon[linecolor=zzttqq,fillcolor=zzttqq,fillstyle=solid,opacity=0.1](3.64,0.58)(7.7995583136,2.4609767374)(9.5,1.92)(5.25188613453,-0.00738223546567)
\pspolygon[linecolor=zzttqq,fillcolor=zzttqq,fillstyle=solid,opacity=0.1](3.64,0.58)(9.5,1.92)(16.04,1.74)(11.12,0.28)
\psline{->}(3.64,0.58)(9.08,3.04)
\psline{->}(3.64,0.58)(10.72,-2.)
\rput[tl](5.24,2.36){$\langle w\rangle$}
\rput[tl](6.42,-0.72){$\langle v\rangle$}
\psline[linecolor=zzttqq](3.64,0.58)(6.36375433438,1.81169773209)
\psline[linecolor=zzttqq](6.36375433438,1.81169773209)(11.12,0.28)
\psline[linecolor=zzttqq](11.12,0.28)(7.80130856527,-0.936409053446)
\psline[linecolor=zzttqq](7.80130856527,-0.936409053446)(3.64,0.58)
\psline{->}(3.64,0.58)(11.12,0.28)
\psline[linecolor=zzttqq](3.64,0.58)(7.7995583136,2.4609767374)
\psline[linecolor=zzttqq](7.7995583136,2.4609767374)(9.5,1.92)
\psline[linecolor=zzttqq](9.5,1.92)(5.25188613453,-0.00738223546567)
\psline[linecolor=zzttqq](5.25188613453,-0.00738223546567)(3.64,0.58)
\psline{->}(3.64,0.58)(9.5,1.92)
\psline[linecolor=zzttqq](3.64,0.58)(9.5,1.92)
\psline[linecolor=zzttqq](9.5,1.92)(16.04,1.74)
\psline[linecolor=zzttqq](16.04,1.74)(11.12,0.28)
\psline[linecolor=zzttqq](11.12,0.28)(3.64,0.58)
\begin{scriptsize}
\psdots[dotstyle=*,linecolor=blue](3.64,0.58)
\rput[bl](3.72,0.7){\blue{$O$}}
\psdots[dotstyle=*,linecolor=xdxdff](6.36375433438,1.81169773209)
\rput[bl](6.44,1.94){\xdxdff{$A$}}
\psdots[dotstyle=*,linecolor=xdxdff](7.80130856527,-0.936409053446)
\rput[bl](7.88,-0.82){\xdxdff{$B$}}
\psdots[dotstyle=*,linecolor=blue](11.12,0.28)
\rput[bl](11.2,0.4){\blue{$P$}}
\psdots[dotstyle=*,linecolor=xdxdff](7.7995583136,2.4609767374)
\rput[bl](7.88,2.58){\xdxdff{$A'$}}
\psdots[dotstyle=*,linecolor=blue](9.5,1.92)
\rput[bl](9.58,2.04){\blue{$P'$}}
\psdots[dotstyle=*,linecolor=xdxdff](5.25188613453,-0.00738223546567)
\rput[bl](5.08,-0.34){\xdxdff{$B'$}}
\end{scriptsize}
\end{pspicture*}
  \caption{}
  \label{fig:1}
\end{figure}

下面我们根据对一次形式楔积的几何解释,来说明为什么一次形式的楔积具有分
配律.也就是,为什么会有
$$
w\wedge (v_1+v_2)=w\wedge v_1+w\wedge v_2,
$$
其中$v_2=c_1dx_1+c_2dx_2+\cdots+c_ndx_n$.
\end{document}
