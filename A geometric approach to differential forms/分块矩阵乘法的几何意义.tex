\documentclass[a4paper]{article}
\usepackage{amsmath,amsfonts,amsthm,amssymb}
\usepackage{bm}
\usepackage{hyperref}
\usepackage{geometry}
\usepackage{yhmath}
\usepackage{pstricks-add}
\usepackage{framed,mdframed}
\usepackage{graphicx,color} 
\usepackage{mathrsfs,xcolor} 
\usepackage[all]{xy}
\usepackage{fancybox} 
\usepackage{xeCJK}
\newtheorem*{theo}{定理}
\newtheorem*{exe}{题目}
\newtheorem*{rem}{评论}
\newtheorem*{lemma}{引理}
\newtheorem*{coro}{推论}
\newtheorem*{exa}{例}
\newenvironment{corollary}
{\bigskip\begin{mdframed}\begin{coro}}
    {\end{coro}\end{mdframed}\bigskip}
\newenvironment{theorem}
{\bigskip\begin{mdframed}\begin{theo}}
    {\end{theo}\end{mdframed}\bigskip}
\newenvironment{exercise}
{\bigskip\begin{mdframed}\begin{exe}}
    {\end{exe}\end{mdframed}\bigskip}
\newenvironment{example}
{\bigskip\begin{mdframed}\begin{exa}}
    {\end{exa}\end{mdframed}\bigskip}
\newenvironment{remark}
{\bigskip\begin{mdframed}\begin{rem}}
    {\end{rem}\end{mdframed}\bigskip}
\geometry{left=2.5cm,right=2.5cm,top=2.5cm,bottom=2.5cm}
\setCJKmainfont[BoldFont=SimHei]{SimSun}
\renewcommand{\today}{\number\year 年 \number\month 月 \number\day 日}
\newcommand{\D}{\displaystyle}\newcommand{\ri}{\Rightarrow}
\newcommand{\ds}{\displaystyle} \renewcommand{\ni}{\noindent}
\newcommand{\ov}{\overrightarrow}
\newcommand{\pa}{\partial} \newcommand{\Om}{\Omega}
\newcommand{\om}{\omega} \newcommand{\sik}{\sum_{i=1}^k}
\newcommand{\vov}{\Vert\omega\Vert} \newcommand{\Umy}{U_{\mu_i,y^i}}
\newcommand{\lamns}{\lambda_n^{^{\scriptstyle\sigma}}}
\newcommand{\chiomn}{\chi_{_{\Omega_n}}}
\newcommand{\ullim}{\underline{\lim}} \newcommand{\bsy}{\boldsymbol}
\newcommand{\mvb}{\mathversion{bold}} \newcommand{\la}{\lambda}
\newcommand{\La}{\Lambda} \newcommand{\va}{\varepsilon}
\newcommand{\be}{\beta} \newcommand{\al}{\alpha}
\newcommand{\dis}{\displaystyle} \newcommand{\R}{{\mathbb R}}
\newcommand{\N}{{\mathbb N}} \newcommand{\cF}{{\mathcal F}}
\newcommand{\gB}{{\mathfrak B}} \newcommand{\eps}{\epsilon}
\renewcommand\refname{参考文献}\renewcommand\figurename{图}
\usepackage[]{caption2} 
\renewcommand{\captionlabeldelim}{}
\setlength\parindent{0pt}
\begin{document}
\title{\huge{\bf{分块矩阵乘法的几何意义}}} \author{\small{叶卢庆\footnote{叶卢庆(1992---),男,杭州师范大学理学院数学与应用数学专业本科在读,E-mail:yeluqingmathematics@gmail.com}}}
\maketitle
在此我们探讨分块矩阵乘法的几何意义.设$T_{1}:\mathbf{R}^n\to
\mathbf{R}^m$,$T_2:\mathbf{R}^m\to \mathbf{R}^l$是两个线性变换.其
中$n,m,l$都是正整数.$\alpha=(v_1,\cdots,v_n)$是$\mathbf{R}^n$中的一组
有序基,$\beta=(w_1,\cdots,w_m)$是$\mathbf{R}^m$中的一组有序
基,$\gamma=(r_1,\cdots,r_l)$是$\mathbf{R}^l$中的一组有序基.则可得
$[T_1]_{\alpha}^{\beta}$是一个$m\times n$矩阵,$[T_2]_{\beta}^{\gamma}$
是一个$l\times m$矩阵.\\

将矩阵$[T_1]_{\alpha}^{\beta}$同时位于第$q,\cdots,q+i-1$行($1\leq q\leq
m,1\leq q+i-1\leq m)$,第
$k,\cdots,k+j-1$列的元素取出($1\leq k\leq n,1\leq k+j-1\leq n)$,按照这
些元素在矩阵$[T_1]_{\alpha}^{\beta}$中原本的顺序重新排成一个$i\times
j$矩阵,则这个矩阵是$[T_1]_{\alpha}^{\beta}$的子矩阵.虽然仅凭这个子矩阵,我
们无法知道线性映射$T_1$如何将有序基$\alpha$中的$j$个向量$v_k,\cdots,v_{k+j-1}$张成的位于$\mathbf{R}^n$中
的平行体$S$映射成$\mathbf{R}^m$中的平行体$S'$.但是我们可以确定
$\mathbf{R}^m$中的平行体$S'$在向量$w_q,\cdots,w_{q+i-1}$张成的子空间中的投影.



\end{document}
