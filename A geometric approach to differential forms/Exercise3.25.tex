\documentclass[a4paper]{article}
\usepackage{amsmath,amsfonts,amsthm,amssymb}
\usepackage{bm}
\usepackage{hyperref}
\usepackage{geometry}
\usepackage{yhmath}
\usepackage{pstricks-add}
\usepackage{framed,mdframed}
\usepackage{graphicx,color} 
\usepackage{mathrsfs,xcolor} 
\usepackage[all]{xy}
\usepackage{fancybox} 
\usepackage{xeCJK}
\newtheorem*{theo}{定理}
\newtheorem*{exe}{Exercise}
\newtheorem*{rem}{评论}
\newtheorem*{lemma}{引理}
\newtheorem*{coro}{推论}
\newtheorem*{exa}{例}
\newenvironment{corollary}
{\bigskip\begin{mdframed}\begin{coro}}
    {\end{coro}\end{mdframed}\bigskip}
\newenvironment{theorem}
{\bigskip\begin{mdframed}\begin{theo}}
    {\end{theo}\end{mdframed}\bigskip}
\newenvironment{exercise}
{\bigskip\begin{mdframed}\begin{exe}}
    {\end{exe}\end{mdframed}\bigskip}
\newenvironment{example}
{\bigskip\begin{mdframed}\begin{exa}}
    {\end{exa}\end{mdframed}\bigskip}
\newenvironment{remark}
{\bigskip\begin{mdframed}\begin{rem}}
    {\end{rem}\end{mdframed}\bigskip}
\geometry{left=2.5cm,right=2.5cm,top=2.5cm,bottom=2.5cm}
\setCJKmainfont[BoldFont=SimHei]{SimSun}
\newcommand{\D}{\displaystyle}\newcommand{\ri}{\Rightarrow}
\newcommand{\ds}{\displaystyle} \renewcommand{\ni}{\noindent}
\newcommand{\ov}{\overrightarrow}
\newcommand{\pa}{\partial} \newcommand{\Om}{\Omega}
\newcommand{\om}{\omega} \newcommand{\sik}{\sum_{i=1}^k}
\newcommand{\vov}{\Vert\omega\Vert} \newcommand{\Umy}{U_{\mu_i,y^i}}
\newcommand{\lamns}{\lambda_n^{^{\scriptstyle\sigma}}}
\newcommand{\chiomn}{\chi_{_{\Omega_n}}}
\newcommand{\ullim}{\underline{\lim}} \newcommand{\bsy}{\boldsymbol}
\newcommand{\mvb}{\mathversion{bold}} \newcommand{\la}{\lambda}
\newcommand{\La}{\Lambda} \newcommand{\va}{\varepsilon}
\newcommand{\be}{\beta} \newcommand{\al}{\alpha}
\newcommand{\dis}{\displaystyle} \newcommand{\R}{{\mathbb R}}
\newcommand{\N}{{\mathbb N}} \newcommand{\cF}{{\mathcal F}}
\newcommand{\gB}{{\mathfrak B}} \newcommand{\eps}{\epsilon}
\renewcommand\refname{参考文献}\renewcommand\figurename{图}
\usepackage[]{caption2} 
\renewcommand{\captionlabeldelim}{}
\setlength\parindent{0pt}
\begin{document}
\begin{exercise}
  Show that any $3$- form on $T_p\mathbf{R}^4$ can be written as the
  product of three $1$-forms.
\end{exercise}
\begin{proof}
Any three form on $T_p\mathbf{R}^4$ can be expressed as
\begin{equation}\label{eq:1}
a_1dx\wedge dy\wedge dz+a_2dx\wedge dy\wedge dw+a_3dx\wedge dz\wedge
dw+a_4dy\wedge dz\wedge dw.
\end{equation}
\eqref{eq:1} is equivalent to
\begin{equation}
  \label{eq:2}
dx\wedge (a_1dy\wedge dz+a_2dy\wedge dw+a_3dz\wedge dw)+a_4dy\wedge
dz\wedge dw.
\end{equation}
We know that there exists $p_1,p_2,p_3$ and $q_1,q_2,q_3$,such that 
$$
a_1dy\wedge dz+a_2dy\wedge dw+a_3dz\wedge dw=(p_1dy+p_2dz+p_3dw)\wedge (q_1dy+q_2dz+q_3dw),
$$
so
\begin{align*}
&dx\wedge (a_1dy\wedge dz+a_2dy\wedge dw+a_3dz\wedge dw)+a_4dy\wedge
dz\wedge dw\\&=dx\wedge (p_1dy+p_2dz+p_3dw)\wedge
(q_1dy+q_2dz+q_3dw)+a_4dy\wedge dz\wedge dw.
\end{align*}
For the intersection of any two three dimensional subspace of a four dimensional linear
space exists and the intersection  is a two dimensional subspace of
the four dimensional space,so when $(p_1,p_2,p_3)$ and $(q_1,q_2,q_3)$
are linearly independent,then there exists $n_1,n_2,n_3$ such that 
$$
dy\wedge dz\wedge dw=(p_1dy+p_2dz+p_3dw)\wedge
(q_1dy+q_2dz+q_3dw)\wedge (n_1dy+n_2dz+n_3dw).
$$
So
\begin{align*}
&dx\wedge (a_1dy\wedge dz+a_2dy\wedge dw+a_3dz\wedge dw)+a_4dy\wedge
dz\wedge dw\\&=dx\wedge (p_1dy+p_2dz+p_3dw)\wedge
(q_1dy+q_2dz+q_3dw)+a_4dy\wedge dz\wedge dw
\\&=(p_1dy+p_2dz+p_3dw)\wedge
(q_1dy+q_2dz+q_3dw)\wedge (dx+a_4n_1dy+a_4n_2dz+a_4n_3dw).
\end{align*}
When $(q_1,q_2,q_3)$ and $(p_1,p_2,p_3)$ are linearly dependent,then 
\begin{align*}
&dx\wedge (a_1dy\wedge dz+a_2dy\wedge dw+a_3dz\wedge dw)+a_4dy\wedge
dz\wedge dw\\&=a_4dy\wedge dz\wedge dw.
\end{align*}
\end{proof}
\end{document}
