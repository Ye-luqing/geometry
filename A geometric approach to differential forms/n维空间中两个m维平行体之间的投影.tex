\documentclass[a4paper]{article}
\usepackage{amsmath,amsfonts,amsthm,amssymb}
\usepackage{bm,euler}
\usepackage{hyperref}
\usepackage{geometry}
\usepackage{yhmath}
\usepackage{pstricks-add}
\usepackage{framed,mdframed}
\usepackage{graphicx,color} 
\usepackage{mathrsfs,xcolor} 
\usepackage[all]{xy}
\usepackage{fancybox} 
\usepackage{xeCJK}
\newtheorem*{theo}{定理}
\newtheorem*{exe}{题目}
\newtheorem*{rem}{评论}
\newtheorem*{lemma}{引理}
\newtheorem*{coro}{推论}
\newtheorem*{exa}{例}
\newenvironment{corollary}
{\bigskip\begin{mdframed}\begin{coro}}
    {\end{coro}\end{mdframed}\bigskip}
\newenvironment{theorem}
{\bigskip\begin{mdframed}\begin{theo}}
    {\end{theo}\end{mdframed}\bigskip}
\newenvironment{exercise}
{\bigskip\begin{mdframed}\begin{exe}}
    {\end{exe}\end{mdframed}\bigskip}
\newenvironment{example}
{\bigskip\begin{mdframed}\begin{exa}}
    {\end{exa}\end{mdframed}\bigskip}
\newenvironment{remark}
{\bigskip\begin{mdframed}\begin{rem}}
    {\end{rem}\end{mdframed}\bigskip}
\geometry{left=2.5cm,right=2.5cm,top=2.5cm,bottom=2.5cm}
\setCJKmainfont[BoldFont=SimHei]{FZQingKeBenYueSongS-R-GB}
\renewcommand{\today}{\number\year 年 \number\month 月 \number\day 日}
\newcommand{\D}{\displaystyle}\newcommand{\ri}{\Rightarrow}
\newcommand{\ds}{\displaystyle} \renewcommand{\ni}{\noindent}
\newcommand{\ov}{\overrightarrow}
\newcommand{\pa}{\partial} \newcommand{\Om}{\Omega}
\newcommand{\om}{\omega} \newcommand{\sik}{\sum_{i=1}^k}
\newcommand{\vov}{\Vert\omega\Vert} \newcommand{\Umy}{U_{\mu_i,y^i}}
\newcommand{\lamns}{\lambda_n^{^{\scriptstyle\sigma}}}
\newcommand{\chiomn}{\chi_{_{\Omega_n}}}
\newcommand{\ullim}{\underline{\lim}} \newcommand{\bsy}{\boldsymbol}
\newcommand{\mvb}{\mathversion{bold}} \newcommand{\la}{\lambda}
\newcommand{\La}{\Lambda} \newcommand{\va}{\varepsilon}
\newcommand{\be}{\beta} \newcommand{\al}{\alpha}
\newcommand{\dis}{\displaystyle} \newcommand{\R}{{\mathbb R}}
\newcommand{\N}{{\mathbb N}} \newcommand{\cF}{{\mathcal F}}
\newcommand{\gB}{{\mathfrak B}} \newcommand{\eps}{\epsilon}
\renewcommand\refname{参考文献}\renewcommand\figurename{图}
\usepackage[]{caption2} 
\renewcommand{\captionlabeldelim}{}
\setlength\parindent{0pt}
\begin{document}
\title{\huge{\bf{$n$维空间中两个$m$维平行体之间的投影}}} \author{\small{叶卢庆\footnote{叶卢庆(1992---),男,杭州师范大学理学院数学与应用数学专业本科在读,E-mail:yeluqingmathematics@gmail.com}}}
\maketitle
设$\mathbf{a}_1,\mathbf{a}_2,\cdots,\mathbf{a}_m$是$\mathbf{R}^n$中的$m$个向量.集合
$$\left\{p_1\mathbf{a}_1+p_2\mathbf{a}_2+\cdots+p_m\mathbf{a}_m:p_1,p_2,\cdots,p_m\geq 0,p_1+p_2+\cdots+p_m\leq
1\right\}$$叫做$\mathbf{R}^{n}$中由向量
$\mathbf{a}_1,\mathbf{a}_2,\cdots,\mathbf{a}_m$张成的一个平行体.\\

设$A$是$\mathbf{R}^{n}$中由$m$个线性无关的向量
$\mathbf{a}_1,\mathbf{a}_2,\cdots,\mathbf{a}_m$张成的一个平行体.$B$是
$\mathbf{R}^n$中由$m$个线性无关的向量
$\mathbf{b}_1,\mathbf{b}_2,\cdots,\mathbf{b}_m$张成的一个平行体.其中
$m< n$.\\

当$\mathbf{b}_1,\mathbf{b}_2,\cdots,\mathbf{b}_m$是两两正
交的向量时.设向量$\mathbf{a}_j$在
$\mathbf{b}_1,\mathbf{b}_2,\cdots,\mathbf{b}_m$张成的子空间上的正投影为
$\mathbf{a}_j'$,易得$\mathbf{a}_j'$是存在且唯一的,而且$\forall 1\leq
i,j\leq m$,
$$
(\mathbf{a}_j-\mathbf{a}_j')\cdot \mathbf{b}_i=0.\iff \mathbf{a}_j\cdot\mathbf{b}_i=\mathbf{a}_j'\cdot\mathbf{b}_i.
$$
\iffalse
于是,
$$
\begin{cases}
  \mathbf{a}_1'\cdot \mathbf{b}_1=\mathbf{a}_1\cdot \mathbf{b}_1\\
\mathbf{a}_1'\cdot \mathbf{b}_2=\mathbf{a}_1\cdot \mathbf{b}_2,\\
\vdots\\
\mathbf{a}_1'\cdot \mathbf{b}_m=\mathbf{a}_1\cdot \mathbf{b}_m.
\end{cases}
$$
\fi
设
$\mathbf{a}_j'=x_{1j}\mathbf{b}_1+x_{2j}\mathbf{b}_2+\cdots+x_{mj}\mathbf{b}_m$,
代入上面的关系式,可得$x_{ij}||b_i||^2=\mathbf{a}_j\cdot
\mathbf{b}_i$.令$\mathbf{e}_i=\frac{\mathbf{b}_i}{||\mathbf{b}_i||}$,
可得$x_{ij}=\mathbf{a}_j\mathbf{e}_i$.于是
$$
\begin{cases}
  \mathbf{a}_1'=(\mathbf{a}_1\cdot
  \mathbf{e}_1)\mathbf{e}_1+(\mathbf{a}_1\cdot
  \mathbf{e}_2)\mathbf{e}_2+\cdots+(\mathbf{a}_1\cdot
  \mathbf{e}_m)\mathbf{e}_m,\\
  \mathbf{a}_2'=(\mathbf{a}_2\cdot
  \mathbf{e}_1)\mathbf{e}_1+(\mathbf{a}_2\cdot
  \mathbf{e}_2)\mathbf{e}_2+\cdots+(\mathbf{a}_2\cdot
  \mathbf{e}_m)\mathbf{e}_m,\\
\vdots\\
  \mathbf{a}_m'=(\mathbf{a}_m\cdot
  \mathbf{e}_1)\mathbf{e}_1+(\mathbf{a}_m\cdot
  \mathbf{e}_2)\mathbf{e}_2+\cdots+(\mathbf{a}_m\cdot
  \mathbf{e}_m)\mathbf{e}_m.
\end{cases}
$$
可见,由$\mathbf{a}_1',\mathbf{a}_2',\cdots,\mathbf{a}_m'$张成的平行体
的有向体积为行列式
\begin{equation}\label{eq:1}
\begin{vmatrix}
  \mathbf{a}_1\cdot \mathbf{e}_1&\mathbf{a}_1\cdot
  \mathbf{e}_2&\cdots&\mathbf{a}_1\cdot \mathbf{e}_m\\
  \mathbf{a}_2\cdot \mathbf{e}_1&\mathbf{a}_2\cdot
  \mathbf{e}_2&\cdots&\mathbf{a}_2\cdot \mathbf{e}_m\\
\vdots&\vdots&\ddots&\vdots\\
  \mathbf{a}_m\cdot \mathbf{e}_1&\mathbf{a}_m\cdot
  \mathbf{e}_2&\cdots&\mathbf{a}_m\cdot \mathbf{e}_m\\
\end{vmatrix}
\end{equation}
再乘以行列式$\det (\mathbf{e}_1,\mathbf{e}_2,\cdots,\mathbf{e}_m)$.于
是,
\begin{equation}
  \label{eq:2}
  \begin{vmatrix}
  \mathbf{a}_1\cdot \mathbf{b}_1&\mathbf{a}_1\cdot
  \mathbf{b}_2&\cdots&\mathbf{a}_1\cdot \mathbf{b}_m\\
  \mathbf{a}_2\cdot \mathbf{b}_1&\mathbf{a}_2\cdot
  \mathbf{b}_2&\cdots&\mathbf{a}_2\cdot \mathbf{b}_m\\
\vdots&\vdots&\ddots&\vdots\\
  \mathbf{a}_m\cdot \mathbf{b}_1&\mathbf{a}_m\cdot
  \mathbf{b}_2&\cdots&\mathbf{a}_m\cdot \mathbf{b}_m\\
\end{vmatrix}
\end{equation}
等于$\mathbf{a}_1',\mathbf{a}_2',\cdots,\mathbf{a}_m'$张成的平行体
的有向体积,再乘以$\mathbf{b}_1',\mathbf{b}_2',\cdots,\mathbf{b}_m'$张
成的平行体的有向体积.
\\

当$\mathbf{b}_1,\mathbf{b}_2,\cdots,\mathbf{b}_m$并非都是两两正交向量
时,根据行列式的性质,上述结论仍然成立.
\end{document}
