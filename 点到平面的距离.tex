\documentclass[xetex,mathserif,serif]{beamer}
\usetheme{AnnArbor}
\usepackage{amsmath,amsfonts,bm}
\usepackage{hyperref}
\usepackage{amsthm} 
\usepackage{geometry}
\usepackage{amssymb}
\usepackage{pstricks-add}
\usepackage{framed,mdframed}
\usepackage{graphicx,color} 
\usepackage{mathrsfs,xcolor} 
\usepackage[all]{xy}
\usepackage{fancybox} 
\usepackage{xeCJK}
%\newtheorem*{theorem}{定理}
%\newtheorem*{lemma}{引理}
%\newtheorem*{corollary}{推论}
%\newtheorem*{exercise}{习题}
%\newtheorem*{example}{例}
\setCJKmainfont[BoldFont=KaiTi_GB2312]{Adobe Fangsong Std}
\renewcommand{\today}{\number\year 年 \number\month 月 \number\day 日}
\newcommand{\D}{\displaystyle}\newcommand{\ri}{\Rightarrow}
\newcommand{\ds}{\displaystyle} \renewcommand{\ni}{\noindent}
\newcommand{\pa}{\partial} \newcommand{\Om}{\Omega}
\newcommand{\om}{\omega} \newcommand{\sik}{\sum_{i=1}^k}
\newcommand{\vov}{\Vert\omega\Vert} \newcommand{\Umy}{U_{\mu_i,y^i}}
\newcommand{\lamns}{\lambda_n^{^{\scriptstyle\sigma}}}
\newcommand{\chiomn}{\chi_{_{\Omega_n}}}
\newcommand{\ullim}{\underline{\lim}} \newcommand{\bsy}{\boldsymbol}
\newcommand{\mvb}{\mathversion{bold}} \newcommand{\la}{\lambda}
\newcommand{\La}{\Lambda} \newcommand{\va}{\varepsilon}
\newcommand{\be}{\beta} \newcommand{\al}{\alpha}
\newcommand{\dis}{\displaystyle} \newcommand{\R}{{\mathbb R}}
\newcommand{\N}{{\mathbb N}} \newcommand{\cF}{{\mathcal F}}
\newcommand{\gB}{{\mathfrak B}} \newcommand{\eps}{\epsilon}
\renewcommand\refname{参考文献}\newcommand{\degre}{\ensuremath{^\circ}}
\begin{document}
\title{\bf{点到平面的距离}}
\author{\texorpdfstring{叶卢
    庆~\\\url{yeluqingmathematics@gmail.com}~~}{叶卢庆}}
\institute{杭州师范大学理学院数学112}
\frame{\titlepage}
  \begin{frame}

\frametitle{\bf{点到直线的距离}}

  \end{frame}
  \begin{frame}
    \frametitle{\textbf{点到直线的距离}}
    \framesubtitle{\textbf{在平面直角坐标系中考虑}}
$$\overrightarrow{MA}\cdot \overrightarrow{AP}=0$$
\newrgbcolor{xdxdff}{0.49 0.49 1}
\newrgbcolor{qqwuqq}{0 0.39 0}
\psset{xunit=0.5cm,yunit=0.5cm,algebraic=true,dotstyle=o,dotsize=3pt 0,linewidth=0.8pt,arrowsize=3pt 2,arrowinset=0.25}
\begin{pspicture*}(0.17,-2.7)(15.11,3.95)
\pspolygon[linecolor=qqwuqq,fillcolor=qqwuqq,fillstyle=solid,opacity=0.1](5.84,-0.2)(5.62,-0.26)(5.69,-0.49)(5.91,-0.42)
\psplot{0.17}{15.11}{(-18.25--2.48*x)/8.6}
\psline(4.98,2.66)(1.74,-1.62)
\psline(4.98,2.66)(10.34,0.86)
\psline(4.98,2.66)(5.91,-0.42)
\psline{->}(5.91,-0.42)(4.98,2.66)
\psline(4.98,2.66)(3.12,-1.22)
\psline(4.98,2.66)(4.79,-0.74)
\psline(4.98,2.66)(7.7,0.1)
\psline(4.98,2.66)(13.05,1.64)
\begin{scriptsize}
\rput[bl](1.78,-1.56){\blue{$M$}}
\rput[bl](5.03,2.73){\blue{$P$}}
\rput[bl](5.96,-0.35){\xdxdff{$A$}}
\rput[bl](13.1,1.7){\xdxdff{$N$}}
\end{scriptsize}
\end{pspicture*}
$M(m_1,m_2)$,$N(n_1,n_2),P(p_1,p_2)$.\pause
$\overrightarrow{MN}=(n_1-m_1,n_2-m_2)$.\pause
$$\overrightarrow{MA}=\lambda \overrightarrow{MN}=(\lambda
(n_1-m_1),\lambda(n_2-m_2)).$$\pause
$$
\overrightarrow{AP}=\overrightarrow {MP}-\overrightarrow {MA}\pause
=(p_1-m_1,p_2-m_2)-\pause (\lambda(n_1-m_1),\lambda(n_2-m_2))
$$

  \end{frame}

\end{document}