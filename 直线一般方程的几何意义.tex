\documentclass[a4paper]{article} 
\usepackage{amsmath,amsfonts,bm}
\usepackage{hyperref}
\usepackage{amsthm} 
\usepackage{geometry}
\usepackage{amssymb}
\usepackage{pstricks-add}
\usepackage{framed,mdframed}
\usepackage{graphicx,color} 
\usepackage{mathrsfs,xcolor} 
\usepackage[all]{xy}
\usepackage{fancybox} 
\usepackage{xeCJK}
\newtheorem*{theorem}{定理}
\newtheorem*{lemma}{引理}
\newtheorem*{corollary}{推论}
\newtheorem*{exercise}{习题}
\newtheorem*{example}{例}
\geometry{left=2.5cm,right=2.5cm,top=2.5cm,bottom=2.5cm}
\setCJKmainfont[BoldFont=Adobe Heiti Std R]{Adobe Song Std L}
\renewcommand{\today}{\number\year 年 \number\month 月 \number\day 日}
\newcommand{\D}{\displaystyle}\newcommand{\ri}{\Rightarrow}
\newcommand{\ds}{\displaystyle} \renewcommand{\ni}{\noindent}
\newcommand{\pa}{\partial} \newcommand{\Om}{\Omega}
\newcommand{\om}{\omega} \newcommand{\sik}{\sum_{i=1}^k}
\newcommand{\vov}{\Vert\omega\Vert} \newcommand{\Umy}{U_{\mu_i,y^i}}
\newcommand{\lamns}{\lambda_n^{^{\scriptstyle\sigma}}}
\newcommand{\chiomn}{\chi_{_{\Omega_n}}}
\newcommand{\ullim}{\underline{\lim}} \newcommand{\bsy}{\boldsymbol}
\newcommand{\mvb}{\mathversion{bold}} \newcommand{\la}{\lambda}
\newcommand{\La}{\Lambda} \newcommand{\va}{\varepsilon}
\newcommand{\be}{\beta} \newcommand{\al}{\alpha}
\newcommand{\dis}{\displaystyle} \newcommand{\R}{{\mathbb R}}
\newcommand{\N}{{\mathbb N}} \newcommand{\cF}{{\mathcal F}}
\newcommand{\gB}{{\mathfrak B}} \newcommand{\eps}{\epsilon}
\renewcommand\refname{参考文献}
\begin{document}
\title{\huge{\bf{直线一般方程的几何意义}}} \author{\small{叶卢
    庆\footnote{叶卢庆(1992---),男,杭州师范大学理学院数学与应用数学专业
      本科在读,E-mail:yeluqingmathematics@gmail.com}}\\{\small{杭州师范大学理学院
      数学112}}}
\maketitle
我们知道,平面直角坐标系中直线的一般方程为
\begin{equation}
  \label{eq:1}
  ax+by+c=0.
\end{equation}
其中 $a,b,c\in \mathbf{R}$.\eqref{eq:1} 可以写为三维空间
$\mathbf{R}^3$ 中两个向量的点积. 
$$
\begin{pmatrix}
  a&b&c
\end{pmatrix}\begin{pmatrix}
  x\\
y\\
1
\end{pmatrix}=0.
$$
不妨将 $\mathbf{R}^3$ 中的向量 $\begin{pmatrix}
  a&b&c
\end{pmatrix}$ 和向量 $\begin{pmatrix}
  x&y&1
\end{pmatrix}$ 的起始点都固定在原点 $(0,0,0)$.则向量 $\begin{pmatrix}
  a&b&c
\end{pmatrix}$ 的终点在 $(a,b,c)$,向量 $\begin{pmatrix}
  x&y&1
\end{pmatrix}$ 的终点在 $(x,y,1)$.经过原点,且与向量 $\begin{pmatrix}
  a&b&c
\end{pmatrix}$ 垂直的平面 $A$,与平面 $z=1$ 的交线就是方程 \eqref{eq:1}.
\end{document}








