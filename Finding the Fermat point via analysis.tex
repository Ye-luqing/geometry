\documentclass{amsart}

\usepackage{amsmath}
\usepackage{amsthm}
\usepackage{amsfonts}
\usepackage{amssymb}

\usepackage{pstricks-add}

\parindent = 5 pt
\parskip = 12 pt
\theoremstyle{plain}

\newtheorem{theorem}{Theorem}
\newtheorem{conjecture}[theorem]{Conjecture}
\newtheorem{problem}[theorem]{Problem}
\newtheorem{assumption}[theorem]{Assumption}
\newtheorem{heuristic}[theorem]{Heuristic}
\newtheorem{proposition}[theorem]{Proposition}
\newtheorem{fact}[theorem]{Fact}
\newtheorem{lemma}[theorem]{Lemma}
\newtheorem{corollary}[theorem]{Corollary}
\newtheorem{claim}[theorem]{Claim}
\newtheorem{question}[theorem]{Question}

\theoremstyle{definition}
\newtheorem{definition}[theorem]{Definition}
\newtheorem{example}[theorem]{Example}
\newtheorem{remark}[theorem]{Remark}

\include{psfig}

\begin{document}
\title{Finding the Fermat Point via analysis}
\author{Luqing Ye}
\address{College of Science, Hangzhou Normal University,Hangzhou City,Zhejiang Province,China}
\email{yeluqingmathematics@gmail.com}
% \thanks{}

\maketitle

%\setcounter{tocdepth}{2}
% \tableofcontents
 Let $P_1,P_2,P_3$ be three given points in $\mathbf{R}^2$ ,and $P$ be an arbitrary
 point in $\mathbf{R}^2$.The famous Fermat's problem to
 Torricelli asks for the location of $P$,such that 
 \begin{equation*}
   |P_{0}P_1|+|P_{0}P_2|+|P_{0}P_3|
 \end{equation*}
is a minimum.Then $P$ is called the Fermat point of the triangle
$P_1P_2P_3$(Triangle $P_1P_2P_{3}$ is  nondegenerate).There exist several elegant geometrical solutions in the
literature.In this note,we consider finding the Fermat point by
using methods in advanced calculus.The main tools we use are the 
extreme value theorem,Fermat's theorem,and the intermediate value
theorem,which are listed below.
\begin{theorem}[The extreme value theorem]
Let $f:D\to \mathbf{R}$ be a continuous function,where $D$ is a
nonempty bounded closed set in $\mathbf{R}^2$.Then $f$ must attain a minimum
on $D$.That is,there exist point $\xi$ in $D$ such that $f(\xi)\leq
f(x)$ for all $x\in D$.
\end{theorem}
\begin{theorem}[Fermat's theorem]
  Let $f:C\to \mathbf{R}$ be a differentiable function,where $C$ is a
  nonempty open set in $\mathbf{R}^2$.Suppose $x_0\in D$ is a local
  extreme of $f$,then $f'(x_0)=\mathbf{0}$.
\end{theorem}
\begin{theorem}[The intermediate value theorem]
  Let $f:I\to \mathbf{R}$ be a continuous function,where $I=[a,b]$ is a
  closed interval of $\mathbf{R}$.For any real number $u$ between
  $f(a)$ and $f(b)$,there exist $\xi\in [a,b]$ such that $f(\xi)=u$.
\end{theorem}

And we need two lemmas.
\begin{lemma}
  Let $M_1M_2M_3$ be a nondegenerate triangle,let $M$ be an arbitrary
  point in the interior of the triangle region.As is shown in Figure
  \eqref{fig:0}. Then $$\hbox{rad}\angle M_1MM_3>\hbox{rad}\angle
  M_1M_2M_3$$,where $\hbox{rad}\angle M_1MM_3$ is the radian measure of
  the angle $\angle M_1MM_3$.
\begin{proof}
  Extend the segment $M_3M$ to $N$,where $N$ is a point on the segment $M_1M_2$.Then  $\hbox{rad}\angle M_1MM_3>\hbox{rad}\angle M_1NM_3$,and $\hbox{rad}\angle M_1NM_3>\hbox{rad}\angle M_1M_2M_3$.So $\hbox{rad}\angle M_1MM_3>\hbox{rad}\angle M_1M_2M_3$.
\end{proof}
\begin{figure}[h]
\psset{xunit=1.0cm,yunit=1.0cm,algebraic=true,dotstyle=o,dotsize=3pt 0,linewidth=0.8pt,arrowsize=3pt 2,arrowinset=0.25}
\begin{pspicture*}(1.3,-5.88)(23.02,6.3)
\psline(6.3,3.38)(3.42,-1.66)
\psline(3.42,-1.66)(12.08,-1.62)
\psline(12.08,-1.62)(6.3,3.38)
\psline(6.3,3.38)(7.08,0.46)
\psline(7.08,0.46)(3.42,-1.66)
\psline[linestyle=dashed,dash=5pt 5pt](8.63,1.36)(7.08,0.46)
\begin{scriptsize}
\psdots[dotstyle=*](6.3,3.38)
\rput[bl](6.06,3.7){{$M_1$}}
\psdots[dotstyle=*](3.42,-1.66)
\rput[bl](2.9,-1.62){{$M_3$}}
\psdots[dotstyle=*](12.08,-1.62)
\rput[bl](12.5,-1.68){{$M_2$}}
\psdots[dotstyle=*](7.08,0.46)
\rput[bl](7.16,0.58){{$M$}}
\psdots[dotstyle=*](8.63,1.36)
\rput[bl](8.6,1.6){{$N$}}
\end{scriptsize}
\end{pspicture*}
  \caption{}
  \label{fig:0}
\end{figure}
\end{lemma}


\begin{lemma}
  Let $f:\mathbf{R}^2\to \mathbf{R}$ be  continuous in the
  neighborhood $U$ of $P\in \mathbf{R}^2$,and continuously differentiable in the
  deleted neighborhood $U\backslash \{P\}$ of $P$\footnote{The deleted neighborhood of a
  point is the neighborhood of a point which exclude the point
  itself.}. On $U\backslash\{P\}$,when the diameter $\xi$ of $U$ is
small enough,the derivative of $f$ approximates a nonzero linear
map\footnote{A zero linear map maps any vectors into the zero vector.} from $\mathbf{R}^2$ to $\mathbf{R}$,i.e,
$$
\lim_{\xi\to 0;\mathbf{x}\in U\backslash\{P\}}f'(\mathbf{x})= T,
$$
where $T$ is a nonzero linear map from $\mathbf{R^2}$ to $\mathbf{R}$. Then $P$ is not a local extreme of $f$.
\end{lemma}




First we prove the existence of the Fermat point of the triangle
$P_1P_2P_3$.Zuo Quanru and Lin Bo already used a
sophisticated version of this method in \cite{zuo}.

Let $P=(x,y)$,$P_1=(x_1,y_1),P_2=(x_2,y_2),P_3=(x_3,y_3)$.Let 
\begin{equation*}
  f(x,y)=|PP_1|+|PP_2|+|PP_3|=\sum_{i=1}^3\sqrt{(x-x_i)^2+(y-y_i)^2}.
\end{equation*}


\begin{theorem}[Existence of the Fermat point]\label{theorem:3}
  Any triangle $P_1P_2P_3$ has a Fermat point.
\end{theorem}
\begin{proof}
Draw a circle $O_1$ centered at $P_1$,whose radius $r$ is large
enough.Then $$D_{1}=\{|P-P_1|\leq r:P\in \mathbf{R}^2\}$$ is a bounded
closed disk.According to the extreme value theorem,$f$ must
attain a minimum on $D_1$.When $r$ is large,the minimum point of $f$ on
$D_1$ is the minimum point of $f$ on the whole plane $\mathbf{R}^2$.Thus the existence of the Fermat
point of the triangle $P_1P_2P_3$ is guaranteed.
\end{proof}





Now we prove the uniqueness of the Fermat point of the triangle
$P_1P_2P_3$,in the mean time,we find the exact
location of the Fermat point.




If $P_0=(x_0,y_0)$ is a minimum point of $f$,and $P_0\neq P_1,P_2,P_3$,then
according to Fermat's theorem,we have
\begin{equation}\label{eq:1}
  \begin{cases}
          \displaystyle\frac{\partial f}{\partial x}(x_{0},y_{0})=\sum_{i=1}^3
  \frac{x_{0}-x_i}{\sqrt{(x_{0}-x_i)^2+(y_{0}-y_i)^2}}=0,\\
\displaystyle\frac{\partial f}{\partial
    y}(x_{0},y_{0})=\sum_{i=1}^3 \frac{y_{0}-y_i}{\sqrt{(x_{0}-x_i)^2+(y_{0}-y_i)^2}}=0.
  \end{cases}
\end{equation}
Let vectors
$$
\mathbf{L_{0}}=(x_0-x_1,y_0-y_1),\mathbf{M_{0}}=(x_0-x_2,y_0-y_2),\mathbf{N_{0}}=(x_0-x_3,y_0-y_3).
$$
Then the simultaneous equations \eqref{eq:1} is equivalent to
\begin{equation}
  \label{eq:2}
\mathbf{\frac{L_{0}}{|L_{0}|}+\frac{M_{0}}{|M_{0}|}+\frac{N_{0}}{|N_{0}|}}=\mathbf{0}.
\end{equation}

Notice that 
$\frac{\mathbf{L_0}}{|\mathbf{L_0}|},\frac{\mathbf{M_0}}{|\mathbf{M_0}|},\frac{\mathbf{N_0}}{|\mathbf{N_0}|}$
are unit vectors.It can be easily verified that equation
\eqref{eq:2} holds if and only if the point $P_0$ is in the interior of the
triangle region,and the radian measure of the angle between any
two of the unit vectors is $\frac{2\pi}{3}$.So the location of the Fermat point of the
triangle $P_1P_2P_3$ has two possibilities.
\begin{enumerate}
\item \label{item:1}When there
exist a point $P_0$ which satisfies the equation \eqref{eq:2},the Fermat point is in the interior
of the triangle region,or on the vertex of the triangle
$P_1P_2P_3$ whose corresponding interior angle is the
largest among the three interior angles.Be aware at present we haven't
proved the uniqueness of the Fermat point,we will do it later in theorem
8.

\item \label{item:2}When there is no point $P_0$ which satisfies
the equation \eqref{eq:2},the Fermat point must be on the vertex of
the triangle whose corresponding interior angle is the largest among
the three interior angles.In this case,the Fermat point is
unique.Later,in theorem 8,we will show this is only possible when the
radian measure of all the interior triangle are equal or more than $\frac{2\pi}{3}$.
\end{enumerate}

Combine the item \eqref{eq:2} with lemma 4 and theorem 5,we have the following corollary:
\begin{corollary}
When the radian measure of all the interior angle of the triangle $P_1P_2P_3$
is equal or larger than $\frac{2\pi}{3}$,then the Fermat point must be on the vertex of
the triangle whose corresponding interior angle is the largest among
the three interior angles.
\end{corollary}




Now we prove
\begin{theorem}\label{theorem:1.22}
When the radian measure of all the interior angles of the triangle
$P_1P_2P_3$ are less than $\frac{2\pi}{3}$,we can
find a unique point in the interior of the triangle region which satisfies
equation \eqref{eq:2}.  
\end{theorem}
\begin{proof}
As is shown in figure
\eqref{fig:1},draw a family of circles in which all of the circles
pass through $P_2$ and $P_3$.Suppose that an interior  point  of
the triangle region $F$ is on a given
circle,then $\hbox{rad}\angle P_3FP_2$ is a constant.Now we let this given
circle move while keep the property that this circle passes through
$P_2,P_3$.When the center of this circle moves downward to
infinity,$\hbox{rad}\angle P_3FP_2$ tends to $\pi$.When
the center of the circle moves from infinity to a location such that the circle passes through
$P_1,P_2$ and $P_3$,then $\hbox{rad}\angle P_3FP_2$ becomes
$\hbox{rad}\angle P_3P_1P_2$,which is less than $\frac{2\pi}{3}$.So according to
the intermediate value theorem,there exist a location such that when the center of the circle moves to
this location, $\hbox{rad}\angle P_3FP_2$ becomes
$\frac{2\pi}{3}$,denote this circle by $O'$.Now let $F$ moves on this
circle.When $F$ tends to the line $P_1P_{3}$,$\hbox{rad}\angle P_1FP_3$ tends to
$\pi$ while $\hbox{rad}\angle P_1FP_2$ tends to $2\pi-\pi-\frac{2\pi}{3}=\frac{\pi}{3}$.
.When $F$ tends to the line $P_1P_2$,
$\hbox{rad}\angle P_1FP_2$ tends to $\pi$ while $\hbox{rad}\angle P_1FP_3$ tends
to $2\pi-\pi-\frac{2\pi}{3}=\frac{\pi}{3}$.So according to the
intermediate value theorem,there exists a point $F'$ on the circle $O'$ such
that $\hbox{rad}\angle P_1F'P_3=\hbox{rad}\angle P_1F'P_{2}$,i.e,both of them are equal to
$\frac{2\pi-\frac{2\pi}{3}}{2}=\frac{2\pi}{3}$.So $F'$ satisfies equation \eqref{eq:2}.

And the uniqueness of the point which satisfies the equation
\eqref{eq:2} is obvious by lemma 4.
\end{proof}
\begin{figure}[h]
\psset{xunit=1.0cm,yunit=1.0cm,algebraic=true,dotstyle=o,dotsize=3pt 0,linewidth=0.8pt,arrowsize=3pt 2,arrowinset=0.25}
\begin{pspicture*}(2,-6.67)(21.62,4.3)
\psline(7.66,3.72)(5.22,-0.84)
\psline(5.22,-0.84)(12.22,-0.84)
\psline(12.22,-0.84)(7.66,3.72)
\psline(8.72,-6.67)(8.72,4.3)
\pscircle(8.72,0.21){3.65}
\pscircle(8.72,-0.45){3.52}
\pscircle(8.72,-2.56){3.9}
\psline(7.66,3.72)(7.93,1.26)
\psline(7.93,1.26)(5.22,-0.84)
\psline(7.93,1.26)(12.22,-0.84)
\begin{scriptsize}
\psdots[dotstyle=*](7.66,3.72)
\rput[bl](7.73,3.83){{$P_1$}}
\psdots[dotstyle=*](5.22,-0.84)
\rput[bl](4.44,-0.94){{$P_3$}}
\psdots[dotstyle=*](12.22,-0.84)
\rput[bl](12.29,-0.73){{$P_2$}}
\psdots[dotstyle=*](8.72,0.21)
\rput[bl](8.79,0.32){$C$}
\psdots[dotstyle=*](8.72,-0.45)
\rput[bl](8.79,-0.35){}
\psdots[dotstyle=*](8.72,-2.56)
\rput[bl](8.79,-2.45){}
\psdots[dotstyle=*](7.93,1.26)
\rput[bl](8,1.36){{$F$}}
\end{scriptsize}
\end{pspicture*}
\caption{}\label{fig:1}
\end{figure}

Next we prove
\begin{theorem}
  When the radian  measure of all the interior angles of the triangle
$P_1P_2P_3$ are less than $\frac{2\pi}{3}$,then item
\eqref{item:2} does not hold,which means the Fermat point is in the
interior of the triangle and is unique.
\end{theorem}
\begin{proof}

Let  $P=(x,y)$ be an arbitrary point in the small
deleted neighborhood of $P_1$.Let vectors
$$
\mathbf{L}=(x-x_1,y-y_1),\mathbf{M}=(x-x_2,y-y_2),\mathbf{N}=(x-x_3,y-y_3).
$$
When the diameter of the deleted neighborhood $\xi$ is small
enough,the radian measure of the
angle between the unit vectors
$$
\frac{\mathbf{M}}{|\mathbf{M}|},\frac{\mathbf{N}}{|\mathbf{N}|}
$$
approximates $\mbox{rad}\angle P_3P_1P_2$.To be more precise,
$$
\lim_{\xi\to 0} \hbox{rad}\angle P_3PP_2=\hbox{rad}\angle P_3P_1P_2< \frac{2\pi}{3},
$$
which implies that
$$
\lim_{\xi\to 0}\left|\frac{\mathbf{M}}{|\mathbf{M}|}+\frac{\mathbf{N}}{|\mathbf{N}|}\right|>1,
$$
So
$$
\lim_{\xi\to
  0}\left|\frac{\mathbf{L}}{|\mathbf{L}|}+\frac{\mathbf{M}}{|\mathbf{M}|}+\frac{\mathbf{N}}{|\mathbf{N}|}\right|>c,
$$
where $c\in \mathbf{R}^{+}$ is a constant.So according to lemma
5,$P_1$ is not a minimum point of $f$.Similarly,$P_2,P_3$ are not 
minimum points of $f$.So the theorem holds.
\end{proof}


\begin{thebibliography}{1}

\bibitem{zuo}Zuo Quanru,Lin Bo.Fermat Points of finite Point Sets in
  Metric Spaces.Journal of Math.(PRC),1997.V.17,No.3,pp359-364


\end{thebibliography}




\end{document} 




















When the radian of all  the interial angles of the triangle
$\bigtriangleup {P_1P_2P_3}$ are less than $\frac{2\pi}{3}$,and the point
$P$ is inside or on the boundary of the triangle $P_1P_2P_3$,as is
shown in figure \eqref{fig:1},we have
\begin{equation*}
  \begin{cases}
 \angle P_1P_3P_2< \angle P_1PP_2\leq \pi,\\
\angle P_2P_1P_3<\angle P_2PP_3\leq \pi,\\
\angle P_3P_2P_1<\angle P_3PP_1\leq \pi,\\
\angle P_1PP_2+\angle P_2PP_3+\angle P_3PP_1=2\pi.
  \end{cases}
\end{equation*}

The angles $\angle P_1PP_2,\angle P_2PP_3,\angle P_3PP_1$ changes
continuously as the point $P$ moves continuously in the plane, so according to the intermediate value theorem, there exists $P_0$ inside the triangle such that $\angle
P_1P_0P_2=\angle P_2P_{0}P_3=\angle P_3P_0P_{1}=\frac{2\pi}{3}$.

























Now we show that these two possibilities do not hold
simultaneously.When item \eqref{item:1} holds,which means that an interior point of the triangle region is Fermat point of the
triangle,then for any point $Q$ in the small deleted
neighborhood\footnote{The deleted neighborhood of a point $P$ is
  $U\backslash\{P\}$,where $U$ is a neighborhood of $P$. } of
any vertex,$f(Q)$ is larger than $f(P_0)$,and $f$ is continuous,so
$f(P_1)\geq f(P_0),f(P_2)\geq f(P_0),f(P_3)\geq f(P_0)$,which means item
\eqref{item:2} does not hold.


















