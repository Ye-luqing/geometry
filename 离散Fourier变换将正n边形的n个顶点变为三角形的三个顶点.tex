\documentclass[a4paper]{article} 
\usepackage{amsmath,amsfonts,bm}
\usepackage{hyperref}
\usepackage{geometry}
\usepackage{amsthm} 
\usepackage{amssymb}
\usepackage{framed,mdframed}
\usepackage{graphicx,color} 
\usepackage{mathrsfs,xcolor} 
\usepackage[all]{xy}
\usepackage{fancybox} 
\usepackage{xeCJK}
\newtheorem{theorem}{定理}
\newtheorem{lemma}{引理}
\newtheorem{corollary}{推论}
\newtheorem*{exercise}{习题}
\newtheorem*{example}{例}
\renewcommand{\today}{\number\year 年 \number\month 月 \number\day 日}
\setCJKmainfont[BoldFont=Adobe Heiti Std R]{Adobe Song Std L} 
\geometry{left=2.5cm,right=2.5cm,top=2.5cm,bottom=2.5cm}
\begin{document}
\title{{\huge{\bf{离散Fourier变换将正 $n$ 边形的 $n$ 个顶点变成三角形
        的三个顶点}}}}\author{叶卢庆
  \footnote{1992--,杭州师范大学在读本科生,Email:h5411167@gmail.com}}
\maketitle
我们来看复平面上的 $n(n\geq 3)$ 个点 $Z_0,\cdots,Z_{n-1}$,它们对应的复
数分别为$z_0,\cdots,z_{n-1}$.点 $Z_0,\cdots,Z_{n-1}$ 按照逆时针方向排
列,且$Z_0,\cdots,Z_{n-1}$ 是一个正 $n$ 边形的 $n$ 个顶
点.设点 $P$ 是 $n$ 边形 $Z_0\cdots Z_{n-1}$ 的中心,其对应的复数
为 $p$.易得 $$p=\frac{z_0+z_1+\cdots+z_n}{n}.$$将正$n$ 边形平
移,使得 $\forall 0\leq k\leq n-1$, $Z_k$ 平移到 $Z_k'$,其中$Z_k'$ 对应
的复数为 $z_k'=z_k-p$.易得存在复
数 $re^{i\theta}$,使得$z_k'=re^{i\theta}\omega_n^k$,其中
$\omega_n=e^{\frac{2\pi i}{n}}$ 是$n$ 次单位根,$r>0$.且向量
\begin{equation}
  \label{eq:1}\begin{pmatrix}
    u_0\\
    u_1\\
    \vdots\\
    u_{n-1}\\
  \end{pmatrix}=
  \begin{pmatrix}
    \omega_n^0 &\omega_n^0&\cdots&\omega_n^0\\
    \omega_n^0&\omega_n^1&\cdots&\omega_n^{n-1}\\
    \vdots&\vdots&\vdots&\vdots\\
    \omega_n^0&\omega_n^{n-1}&\cdots&\omega_n^{(n-1)(n-1)}\\
  \end{pmatrix}\begin{pmatrix}
    z_0\\
    z_1\\
    \vdots\\
    z_{n-1}
  \end{pmatrix}.
\end{equation}
对于不理解线性代数的读者来说,可以把表达式 \eqref{eq:1} 理解成如下,
$$
\begin{cases}
  u_0=\omega_n^0z_0+\omega_n^0z_1+\cdots+\omega_n^0z_{n-1},\\
  u_1=\omega_n^0z_0+\omega_n^1z_1+\cdots+\omega_n^{n-1}z_{n-1},\\
  \vdots\\
  u_{n-1}=\omega_n^0z_0+\omega_n^{n-1}z_1+\cdots+\omega_n^{(n-1)(n-1)}z_{n-1}.
\end{cases}
$$
这是一个离散Fourier变换.如上叙述的内容作为本文中所有定理的条件.易得
$$
u_0=z_0+z_1+\cdots+z_{n-1}=np.
$$
下面我们来证明
\begin{theorem}
  当 $1\leq j<n-1$ 时,
$$
u_j=\omega_n^0z_0+\omega_n^jz_1+\cdots+\omega_n^{j(n-1)}z_{n-1}=0,
$$
当 $j=n-1$ 时,
$$
u_j=\omega_n^0z_0+\omega_n^jz_1+\cdots+\omega_n^{j(n-1)}z_{n-1}=nre^{i\theta}.
$$
\end{theorem}
\begin{proof}[\textbf{证明}]
  由于当 $1\leq j\leq n$ 时,根据等比数列求和公式,
$$
\omega_n^0p+\omega_n^jp+\cdots+\omega_n^{j(n-1)}p=0,
$$
因此
\begin{align*}
  u_j&=\omega_n^0z_0+\omega_n^jz_1+\cdots+\omega_n^{j(n-1)}z_{n-1}\\&=\omega_{n}^{0}z_{0}'+\omega_{n}^{j}z_{1}'+\cdots+\omega_{n}^{j(n-1)}z_{n-1}'\\&=re^{i\theta}(\omega_{n}^0+\omega_{n}^{j+1}+\cdots+\omega_{n}^{(j+1)(n-1)}).
\end{align*}
当 $\omega_n^{j+1}=1$ ,也就
是 $j=n-1$ 时,可得$u_j=nre^{i\theta}.$否则,根据等比数列求和公式,易
得
$re^{i\theta}(\omega_n^0+\omega_{n}^{j+1}+\cdots+\omega_n^{(j+1)(n-1)})=0$.
\end{proof}
\bigskip
\noindent由上面的定理可见,
$$
\begin{pmatrix}
  u_0\\
  u_1\\
u_2\\
  \vdots\\
  u_{n-1}
\end{pmatrix}=\begin{pmatrix}
  np\\
  0\\
  \vdots\\
  0\\
  nre^{i\theta}
\end{pmatrix}.
$$
于是,离散Fourier变换把一个正 $n$ 边形$(n\geq
3)$ 的顶点变成一个三角形的顶点,三角形的三个顶点对应的复数分别为
$0,np,nre^{i\theta}$,当然当这三个顶点有重合时,那个三角形
是退化的三角形.\\

\noindent特别的,当 $n=3$ 时,离散 Fourier 变换将正三角形的三个顶点变为
三角形的三个顶点.易得
\end{document}








