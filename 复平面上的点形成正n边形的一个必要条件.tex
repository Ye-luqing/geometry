\documentclass[twoside,11pt]{article} 
\usepackage{amsmath,amsfonts,bm}
\usepackage{hyperref}
\usepackage{amsthm} 
\usepackage{amssymb}
\usepackage{framed,mdframed}
\usepackage{graphicx,color} 
\usepackage{mathrsfs,xcolor} 
\usepackage[all]{xy}
\usepackage{fancybox} 
% \usepackage{CJKutf8}
\usepackage{xeCJK}\usepackage{pstricks-add}
\newtheorem{lemma}{定理}
\newtheorem{corollary}{推论}
\newtheorem{remarklemma}{注}[lemma]
\newtheorem*{exercise}{习题}
\newtheorem*{example}{例}
\setCJKmainfont[BoldFont=Adobe Heiti Std R]{Adobe Song Std L}
% \usepackage{latexdef}
\def\ZZ{\mathbb{Z}} \topmargin -0.40in \oddsidemargin 0.08in
\evensidemargin 0.08in \marginparwidth 0.00in \marginparsep 0.00in
\textwidth 16cm \textheight 24cm \newcommand{\D}{\displaystyle}
\newcommand{\ds}{\displaystyle} \renewcommand{\ni}{\noindent}
\newcommand{\pa}{\partial} \newcommand{\Om}{\Omega}
\newcommand{\om}{\omega} \newcommand{\sik}{\sum_{i=1}^k}
\newcommand{\vov}{\Vert\omega\Vert} \newcommand{\Umy}{U_{\mu_i,y^i}}
\newcommand{\lamns}{\lambda_n^{^{\scriptstyle\sigma}}}
\newcommand{\chiomn}{\chi_{_{\Omega_n}}}
\newcommand{\ullim}{\underline{\lim}} \newcommand{\bsy}{\boldsymbol}
\newcommand{\mvb}{\mathversion{bold}} \newcommand{\la}{\lambda}
\newcommand{\La}{\Lambda} \newcommand{\va}{\varepsilon}
\newcommand{\be}{\beta} \newcommand{\al}{\alpha}
\newcommand{\dis}{\displaystyle} \newcommand{\R}{{\mathbb R}}
\newcommand{\N}{{\mathbb N}} \newcommand{\cF}{{\mathcal F}}
\newcommand{\gB}{{\mathfrak B}} \newcommand{\eps}{\epsilon}
\renewcommand\refname{参考文献} \def \qed {\hfill \vrule height6pt
  width 6pt depth 0pt} \topmargin -0.40in \oddsidemargin 0.08in
\evensidemargin 0.08in \marginparwidth0.00in \marginparsep 0.00in
\textwidth 15.5cm \textheight 24cm \pagestyle{myheadings}
\markboth{\rm \centerline{}} {\rm \centerline{}}
\begin{document}
\title{\huge{\bf{复平面上的点形成正$n$ 边形的一个必要条件}}} \author{\small{叶卢
    庆\footnote{叶卢庆(1992---),男,杭州师范大学理学院数学与应用数学专业
      本科在读,E-mail:h5411167@gmail.com}}\\{\small{杭州师范大学理学院,浙
      江~杭州~310036}}} \date{}
\maketitle
\vspace{30pt} 
\begin{lemma}
  设 $z_1,z_2,z_3$ 为复平面上的三个点 $Z_1,Z_2,Z_3$ 所代表的复
  数,且$Z_1,Z_2,Z_3$ 如果不共线,则按照逆时针方向放置.则这三个点形成正三
  角形的充要条件是
  \begin{equation}\label{eq:1}
    z_1+\omega_{3} z_2+\omega_{3}^2z_3=0,
  \end{equation}
  其中 $\omega_{3}=e^{\frac{2\pi}{3}i}$.
\end{lemma}
\begin{proof}[\textbf{证明}]
$$
z_1+\omega_{3} z_2+\omega_{3}^2z_3=0 \iff z_1+\omega_{3}
z_2+(-1-\omega_3)z_3=0 \iff z_1-z_3=\omega_3(z_3-z_2).
$$
得证.
\end{proof}
\begin{lemma}
  设 $z_1,z_2,z_3,z_{4}$ 为复平面上的四个点 $Z_1,Z_2,Z_3,Z_4$ 所代表的
  复数,且$Z_1,Z_2,Z_3,Z_{4}$ 如果不共线,则按照逆时针方向放置.若这四个点
  是正方形的四个顶点,则
  \begin{equation}\label{eq:1}
    z_1+\omega_{4} z_2+\omega_{4}^2z_3+\omega_4^3z_{4}=0,
  \end{equation}
  其中 $\omega_4=e^{\frac{2\pi i}{4}}$.
\end{lemma}
\begin{proof}[\bf{证明}]
  \begin{align*}
    z_1+\omega_{4} z_2+\omega_{4}^2z_3+\omega_4^3z_{4}=0 \iff
    (z_1-z_{3})=i(z_4-z_2).
  \end{align*}
  得证.
\end{proof}
\begin{remarklemma}
  注意式 \ref{eq:1} 不是四个点形成正方形顶点的充要条件.因为方程
  \ref{eq:1} 还可以是如下情形:\\

  \psset{xunit=1.0cm,yunit=1.0cm}
  \begin{pspicture*}(0.02,-2.52)(15.45,4.36)
    \psgrid[subgriddiv=0,gridlabels=0,gridcolor=lightgray](0,0)(0.02,-2.52)(15.45,4.36)
    \psset{xunit=1.0cm,yunit=1.0cm,algebraic=true,dotstyle=o,dotsize=3pt
      0,linewidth=0.8pt,arrowsize=3pt 2,arrowinset=0.25}
    \psline(3,1)(9,1) \psline(4,4)(4,-2) \psline(4,4)(3,1)
    \psline(3,1)(4,-2) \psline(4,-2)(9,1) \psline(9,1)(4,4)
    \begin{scriptsize}
      \psdots[dotstyle=*,linecolor=blue](3,1)
      \rput[bl](3.05,1.07){\blue{$Z_1$}}
      \psdots[dotstyle=*,linecolor=blue](9,1)
      \rput[bl](9.04,1.07){\blue{$Z_3$}}
      \psdots[dotstyle=*,linecolor=blue](4,4)
      \rput[bl](4.04,4.06){\blue{$Z_4$}}
      \psdots[dotstyle=*,linecolor=blue](4,-2)
      \rput[bl](4.04,-1.93){\blue{$Z_2$}}
    \end{scriptsize}
  \end{pspicture*}
\end{remarklemma}
\begin{lemma}
  设 $z_1,z_2,z_3,z_4,z_5$ 为复平面上的五个点 $Z_1,Z_2,Z_3,Z_4,Z_5$ 所
  代表的复数,且 $Z_1,Z_2,Z_3,Z_4,Z_5$ 不共线时,则按照逆时针方向放置.若
  这五个点形成正五边形的五个顶点,则
$$
z_1+\omega_5 z_2+\omega_5^2 z_3+\omega_5^3 z_4+\omega_5^4 z_5=0.
$$
其中 $\omega_5=e^{\frac{2\pi i}{5}}$ 是5次单位根.
\end{lemma}
\begin{proof}[\bf{证明}]
  由于五个点形成正五边形的五个顶点,因此我们把形成的正五边形的中心平移到
  原点.也就是说,令 $z_i'=z_i+p$,其中 $p$ 是一个复数,
$$
p=-\frac{z_1+z_2+z_3+z_4+z_5}{5}.
$$
易得
$$
z_1+\omega_5 z_2+\omega_5^2 z_3+\omega_5^3 z_4+\omega_5^4 z_5=0 \iff
z_1'+\omega_5 z_2'+\omega_5^2 z_3'+\omega_5^3 z_4'+\omega_5^4 z_5'=0,
$$
这是因为 $p+\omega_5 p+\omega_5^2 p+\omega_5^3 p+\omega_5^4
p=0$.然后,令 $z_i=re^{\theta i}$,其中 $r>0$ 是一个实数.这样,我们就只用
证明
$$
r(e^{\theta i}+\omega_5^2e^{\theta i}+\omega_5^4e^{\theta
  i}+\omega_5^6e^{\theta i}+\omega_5^8e^{\theta i})=0,
$$
也就是证明
$$
1+\omega_5^2+\omega_5^4+\omega_5^6+\omega_5^8=0,
$$
这是容易的.
\end{proof}
\begin{lemma}
  设 $z_1,\cdots,z_n$ 为复平面上的 $n$ 个点($n\geq 3$)
  $Z_1,\cdots,Z_n$ 所对应的复数. 且 $Z_1,\cdots,Z_n$ 不共线时,按照逆时
  针方向放置.若这 $n$ 个点形成正 $n$ 边形的 $n$ 个顶点,则
$$
z_1+\omega_nz_2+\omega_n^2z_3+\cdots+\omega_n^{n-1}z_n=0.
$$
其中 $\omega_n=e^{\frac{2\pi i}{n}}$.
\end{lemma}
\begin{proof}[\textbf{证明}]
  可以完全仿照 $n=5$ 的情形(引理2.3)来证明,最后划归为要证明的一个等式为
$$
1+\omega_n^2+\omega_n^4+\cdots+\omega_n^{2n-2}=0,
$$
根据等比数列求和公式,也就是证明,
$$
\frac{1-\omega_n^{2n}}{1-\omega_n^2}=0.
$$
得证.
\end{proof}
\begin{lemma}
  设 $z_1,z_2,z_3,z_{4}$ 为复平面上的四个点 $Z_1,Z_2,Z_3,Z_4$ 所代表的
  复数,且$Z_1,Z_2,Z_3,Z_{4}$ 如果不共线,则按照逆时针方向放置.则这四个点
  是以 $p$ 为中心的正方形的四个顶点当且仅当
  \begin{eqnarray}
    \begin{cases}
z_1+z_2+z_3+z_4=4p,\\
      z_1+\omega_{4} z_2+\omega_{4}^2z_3+\omega_4^3z_{4}=0,\\
      z_1+\omega_4^2z_2+\omega_4^4z_3+\omega_4^6z_4=0,\\
     z_1+\omega_4^3z_2+\omega_4^6z_3+\omega_4^9z_4=2z_1.
    \end{cases}
  \end{eqnarray}
  其中 $\omega_4=e^{\frac{2\pi i}{4}}$.
\end{lemma}
\begin{proof}[\textbf{证明}]
很简单,请自行验证.
\end{proof}
\iffalse
\begin{lemma}
设 $z_1,z_2,z_3,z_4,z_5$ 为复平面上的五个点 $Z_1,Z_2,Z_3,Z_4,Z_5$ 所
  代表的复数,且 $Z_1,Z_2,Z_3,Z_4,Z_5$ 不共线时,则按照逆时针方向放置.则
  这五个点形成以 $p$ 为中心的正五边形的五个顶点,当且仅当
  \begin{equation}
    \label{eq:12.08}
    \begin{cases}
      z_1+z_2+z_3+z_4+z_5=5p,\\
z_1+\omega_5z_2+\omega_5^2z_3+\omega_5^3z_4+\omega_5^4z_5=0,\\
z_1+\omega_5^2z_2+\omega_5^4z_3+\omega_5^6z_4+\omega_5^8z_5=0,\\

    \end{cases}
  \end{equation}
其中 $\omega_5=e^{\frac{2\pi i}{5}}$ 是5次单位根.
\end{lemma}\fi

\iffalse探索了这么多铺垫性的内容,现在我们来探索复平面上的 $n$ 个点 $z_1,z_2,\cdots,z_n$ 形成正 $n$ 边形
的 $n$ 个顶点的充要条件.$z_1,z_2,\cdots,z_n$ 形成正 $n$ 边形的 $n$ 个
顶点,当且仅当 $z_1,z_2,\cdots,z_n$ 经过同样的平移后,形成中心在原点的正 $n$
边形的 $n$ 个顶点,也即存在复数 $p$,使得 $z_1+p,z_2+p,\cdots,z_n+p$ 是
中心在原点的正 $n$ 边形的 $n$ 个顶点.而中心在原点的正 $n$ 边形的 $n$
个顶点可以表示为
$$
e^{i\theta},e^{i\theta}\omega_n,e^{i\theta}\omega_n^2,\cdots,e^{i\theta}\omega_n^{n-1}.
$$\fi

% ----------------------------------------------------------------------------------------
\end{document}





















